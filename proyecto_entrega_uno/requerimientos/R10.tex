%\documentclass[10pt,a4paper]{article}
%
%
%\usepackage[spanish]{babel}
%\usepackage[utf8]{inputenc}
%\usepackage{geometry}
%\usepackage{colortbl}
%\usepackage{longtable}
%
%\geometry{tmargin=1cm,bmargin=2cm,lmargin=2cm,rmargin=2cm}
%\begin{document}
\begin{center}


\begin{longtable}{|p{3cm}|p{3cm}|p{3cm}|p{3cm}|}

\hline
\multicolumn{2}{|p{6cm}|}{{\bf {Descripción del requerimiento:}}
     Eliminar documentos y su información. } & 
     {\bf{ Estado:}} & Análisis \\
\hline
\bf {Creado por:} & 
	Maria Andrea Cruz   & \bf {Actualizado por:} & Felipe Vargas  \\
\hline
\bf {Fecha Creación } & Marzo 31 2011 & \bf {Fecha de  Actualización }& Abril 2 del 2011\\
\hline 
\multicolumn{2}{|p{6cm}|}{\bf Identificador} & \multicolumn{2}{|p{6cm}|}{R10} \\
\hline
\bf {Tipo de requerimiento:} & No Crítico &  \bf{Tipo de requerimiento:} & Funcional\\     
\hline
\bf Descripcion &\multicolumn{3}{|p{10cm}|}
{El sistema debe proporcionar la posibilidad de que los usuarios que tengan como perfil administrador o catalogador puedan eliminar cualquier documento que se encuentre referenciado en la base de datos y toda la información relacionada con el documento. El sistema debe de proveer por medio de una búsqueda en la base de datos el o los documentos que se desean eliminar.} \\
\hline
\bf Datos de salida &\multicolumn{3}{|p{10cm}|}
{El sistema actualiza en la base de datos lo referente a la tabla que mantiene la información de los documentos eliminando de esta la tupla del documento en cuestión y elimina del sistema host el documento, notificando al usuario que este realizando la eliminación de la operación exitosa.} \\
\hline
\bf resultados esperados &\multicolumn{3}{|p{10cm}|}
{El sistema se ve estimulado con un cambio en la base de datos con nueva información en la cual se tiene un documento menos en lo registros.} \\
\hline
\bf Origen &\multicolumn{3}{|p{10cm}|}{Documento de descripción del problema.} \\
\hline
\bf Dirigido a  &\multicolumn{3}{|p{10cm}|}
{Catalogador y administrador} \\
\hline
\bf Prioridad &\multicolumn{3}{|p{10cm}|}{3} \\
\hline
\bf Requerimientos Asociados &\multicolumn{3}{|p{10cm}|}
{\begin{itemize}
	\item R06
	\item R07
	\item R08
\end{itemize}} \\
\hline
\multicolumn{4}{|>{\columncolor[rgb]{0.8,0.8,0.8}}c|}{\bf Especificación}\\
\hline


\bf Precondiciones &\multicolumn{3}{|p{10cm}|}
{El sistema debe estar conectado a la base de datos, también debe estar logueado un usuario bajo el perfil de administrador o de catalogador lo que le proporcionará una interfaz para esta operación de eliminación de documentos.} \\
\hline
\hline
\bf Poscondiciones &\multicolumn{3}{|p{10cm}|}
{El sistema cambia en su base de datos el registro correspondiente  a documentos eliminando el documento solicitado lo que en un futuro no permitirá que usuarios en su resultado de consulta tengan en la respuesta alguna coincidencia con este documento} \\
\hline
\bf Criterios de Aceptación &\multicolumn{3}{|p{10cm}|}
{Criterio de aceptación del requerimiento} \\
\hline

\end{longtable}
\end{center}

%\end{document} 