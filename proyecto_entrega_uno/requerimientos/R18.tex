%\documentclass[10pt,a4paper]{article}
%
%
%\usepackage[spanish]{babel}
%\usepackage[utf8]{inputenc}
%\usepackage{geometry}
%\usepackage{colortbl}
%\usepackage{longtable}
%
%\geometry{tmargin=1cm,bmargin=2cm,lmargin=2cm,rmargin=2cm}
%\begin{document}
\begin{center}


\begin{longtable}{|p{3cm}|p{3cm}|p{3cm}|p{3cm}|}

\hline
\multicolumn{2}{|p{6cm}|}{{\bf {Descripción del requerimiento:}}
  Almacenar áreas de interés con sus respectivas subáreas.    } & 
     {\bf{ Estado:}} & Análisis \\
\hline
\bf {Creado por:} & 
	Maria Andrea Cruz   & \bf {Actualizado por:} & Felipe Vargas  \\
\hline
\bf {Fecha Creación } & Marzo 31 2011 & \bf {Fecha de  Actualización }& Abril 2 del 2011\\
\hline 
\multicolumn{2}{|p{6cm}|}{\bf Identificador} & \multicolumn{2}{|p{6cm}|}{R18} \\
\hline
\bf {Tipo de requerimiento:} & No Crítico &  \bf{Tipo de requerimiento:} & Funcional\\     
\hline
\bf Descripción &\multicolumn{3}{|p{10cm}|}
{ El sistema debe proporcionar la manera de poder almacenar en el sistema una lista con las áreas de las ciencias de la computación que estarán disponibles, además, se debe de poder indicar cuales áreas son subáreas de otras áreas y una descripción de cada área} \\
\hline
\bf Datos de salida &\multicolumn{3}{|p{10cm}|}
{ El sistema proporcionará una nueva tabla en la base de datos que permitirá almacenar la información relacionada con las áreas de la ciencia de la computación.} \\
\hline
\bf Resultados esperados &\multicolumn{3}{|p{10cm}|}
{ El sistema se ve estimulado con un cambio  en la base de datos al tener nuevas  tablas para mantener información de las áreas de las ciencias de la computación..} \\
\hline
\bf Origen &\multicolumn{3}{|p{10cm}|}{Documento de descripción del problema.} \\
\hline
\bf Dirigido a  &\multicolumn{3}{|p{10cm}|}
{Catalogador, administrador y usuarios normales.} \\
\hline
\bf Prioridad &\multicolumn{3}{|p{10cm}|}{5} \\
\hline
\bf Requerimientos Asociados &\multicolumn{3}{|p{10cm}|}
{ \begin{itemize}
	\item R08
\end{itemize} } \\
\hline
\multicolumn{4}{|>{\columncolor[rgb]{0.8,0.8,0.8}}c|}{\bf Especificación}\\
\hline


\bf Precondiciones &\multicolumn{3}{|p{10cm}|}
{El sistema debe estar conectado a la base de datos y no tener previamente la área que se trata de crear. El usuario debe de tener como perfil administrador o catalogador.
} \\
\hline
\hline
\bf Poscondiciones &\multicolumn{3}{|p{10cm}|}
{El sistema en su base de datos tiene una descripción mas detallada y amplia de las áreas respecto a ciencias de la computación facilitando la recuperacion de documentos mediante este criterio de búsqueda.} \\
\hline
\bf Criterios de Aceptación &\multicolumn{3}{|p{10cm}|}
{Criterio de aceptación del requerimiento} \\
\hline

\end{longtable}
\end{center}

%\end{document} 