%\documentclass[10pt,a4paper]{article}
%
%
%\usepackage[spanish]{babel}
%\usepackage[utf8]{inputenc}
%\usepackage{geometry}
%\usepackage{colortbl}
%\usepackage{longtable}
%
%\geometry{tmargin=1cm,bmargin=2cm,lmargin=2cm,rmargin=2cm}
%\begin{document}
\begin{center}


\begin{longtable}{|p{3cm}|p{3cm}|p{3cm}|p{3cm}|}

\hline
\multicolumn{2}{|p{6cm}|}{{\bf {Descripción del requerimiento:}}
   Asignar datos a documentos   } & 
     {\bf{ Estado:}} & Análisis \\
\hline
\bf {Creado por:} & 
	Maria Andrea Cruz   & \bf {Actualizado por:} & Felipe Vargas  \\
\hline
\bf {Fecha Creación } & Marzo 31 2011 & \bf {Fecha de  Actualización }& Abril 2 del 2011\\
\hline 
\multicolumn{2}{|p{6cm}|}{\bf Identificador} & \multicolumn{2}{|p{6cm}|}{R08} \\
\hline
\bf {Tipo de requerimiento:} & No Crítico &  \bf{Tipo de requerimiento:} & Funcional\\     
\hline
\bf Descripción &\multicolumn{3}{|p{10cm}|}
{El sistema debe de proporcionar la posibilidad a los usuarios que tengan como perfil catalogador o administrador de asignar datos a los documentos que se encuentran referenciados en la base de datos, estos datos son información sobre el documento como el autor, el titulo principal, el título secundario o traducido, la editorial, la fecha de publicación, la fecha de creación, el idioma en el que esta escrito, los derechos de autor, la descripción o resumen del material, las palabras claves relacionadas con este, el formato del archivo, el tamaño del archivo en bytes, la resolución en píxeles y el software recomendado para abrir el documento. } \\
\hline
\bf Datos de salida &\multicolumn{3}{|p{10cm}|}
{ El sistema actualiza en la base de datos lo referente a los atributos del documento, se informa al usuario que solicito la actualización de la operación exitosa.} \\
\hline
\bf Resultados esperados &\multicolumn{3}{|p{10cm}|}
{ El sistema se ve estimulado con un cambio en la base de datos con nueva información actualizada y correcta del documento..} \\
\hline
\bf Origen &\multicolumn{3}{|p{10cm}|}{Documento de descripción del problema.} \\
\hline
\bf Dirigido a  &\multicolumn{3}{|p{10cm}|}
{Catalogador, administrador.} \\
\hline
\bf Prioridad &\multicolumn{3}{|p{10cm}|}{3} \\
\hline
\bf Requerimientos Asociados &\multicolumn{3}{|p{10cm}|}
{\begin{itemize}
	\item R06
\end{itemize}} \\
\hline
\multicolumn{4}{|>{\columncolor[rgb]{0.8,0.8,0.8}}c|}{\bf Especificación}\\
\hline


\bf Precondiciones &\multicolumn{3}{|p{10cm}|}
{El sistema debe estar conectado a la base de datos y tener logueado a un usuario con los perfiles correspondientes a administrador o catalogador los cuales tienen acceso a una interfaz de catalogación la cual tiene una opción para el registro de documentos.} \\
\hline
\hline
\bf Poscondiciones &\multicolumn{3}{|p{10cm}|}
{El sistema en su base de datos tiene ahora un documento nuevo correspondiente a las áreas de las ciencias de la computación y el cual podrá ser consultado posteriormente y también podrá ser descargado.} \\
\hline
\bf Criterios de Aceptación &\multicolumn{3}{|p{10cm}|}
{Criterio de aceptación del requerimiento} \\
\hline

\end{longtable}
\end{center}

%\end{document} 