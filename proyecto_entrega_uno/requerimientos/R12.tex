%\documentclass[10pt,a4paper]{article}
%
%
%\usepackage[spanish]{babel}
%\usepackage[utf8]{inputenc}
%\usepackage{geometry}
%\usepackage{colortbl}
%\usepackage{longtable}
%
%\geometry{tmargin=1cm,bmargin=2cm,lmargin=2cm,rmargin=2cm}
%\begin{document}
\begin{center}


\begin{longtable}{|p{3cm}|p{3cm}|p{3cm}|p{3cm}|}

\hline
\multicolumn{2}{|p{6cm}|}{{\bf {Descripción del requerimiento:}}
  Clasificar documentos por área de interés. } & 
     {\bf{ Estado:}} & Análisis \\
\hline
\bf {Creado por:} & 
	Maria Andrea Cruz   & \bf {Actualizado por:} & Felipe Vargas  \\
\hline
\bf {Fecha Creación } & Marzo 31 2011 & \bf {Fecha de  Actualización }& Abril 2 del 2011\\
\hline 
\multicolumn{2}{|p{6cm}|}{\bf Identificador} & \multicolumn{2}{|p{6cm}|}{R12} \\
\hline
\bf {Tipo de requerimiento:} & No Crítico &  \bf{Tipo de requerimiento:} & Funcional\\     
\hline
\bf Descripción &\multicolumn{3}{|p{10cm}|}
{El sistema debe de permitir clasificar los documentos registrados en el sistema según su tema, esto se logra relacionando el documento con una o varias área de interés existente en el sistema.} \\
\hline
\bf Datos de salida &\multicolumn{3}{|p{10cm}|}
{El sistema permite catalogar los documentos en un área según el tema que estos tratan ayudando a los usuarios en su búsqueda de documentos.} \\
\hline
\bf resultados esperados &\multicolumn{3}{|p{10cm}|}
{El sistema se ve estimulado con un cambio  en la base de datos modificando los registros de la tabla que mantiene los documentos al actualizar estos.} \\
\hline
\bf Origen &\multicolumn{3}{|p{10cm}|}{Documento de descripción del problema.} \\
\hline
\bf Dirigido a  &\multicolumn{3}{|p{10cm}|}
{Catalogador, administrador y usuarios normales.} \\
\hline
\bf Prioridad &\multicolumn{3}{|p{10cm}|}{5} \\
\hline
\bf Requerimientos Asociados &\multicolumn{3}{|p{10cm}|}
{\begin{itemize}
	\item R07
	\item R08
\end{itemize}
} \\
\hline
\multicolumn{4}{|>{\columncolor[rgb]{0.8,0.8,0.8}}c|}{\bf Especificación}\\
\hline


\bf Precondiciones &\multicolumn{3}{|p{10cm}|}
{El sistema debe de estar conectado a la base de datos permitiendo la consulta referente a las áreas de interés que serán asignadas a los documentos mediante una interfaz a la hora de catalogarlos.} \\
\hline
\hline
\bf Poscondiciones &\multicolumn{3}{|p{10cm}|}
{El sistema tiene en su base de datos información de todos los documetnos de acuerdo a su área de interés lo que facilita la consulta de determinados usuarios interesadosen un área específica} \\
\hline
\bf Criterios de Aceptación &\multicolumn{3}{|p{10cm}|}
{Criterio de aceptación del requerimiento} \\
\hline

\end{longtable}
\end{center}

%\end{document} 