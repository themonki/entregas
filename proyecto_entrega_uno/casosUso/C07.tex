%\documentclass[10pt,a4paper]{article}
%
%
%\usepackage[spanish]{babel}
%\usepackage[utf8]{inputenc}
%\usepackage{geometry}
%\usepackage{colortbl}
%\usepackage{longtable}
%\usepackage{graphicx}
%
%\geometry{tmargin=1cm,bmargin=2cm,lmargin=2cm,rmargin=2cm}
%\begin{document}
\begin{center}


\begin{longtable}{|p{3cm}|p{3cm}|p{3cm}|p{3cm}|}

\hline
\bf {Empresa:} & \multicolumn{2}{|p{6cm}|}  { Escuela de Ingeniería de Sistemas y Computación }  & {\includegraphics[width=80.5pt]{LOGO}} \\
\hline
\bf {Nombre del caso de uso:}&\multicolumn{3}{|p{6cm}|}{Ingresar Palabras Clave } \\
\hline 
\bf Codigo: & CU07  &\bf Fecha: & \\

\hline 
\bf Autor(es ): &Yerminson Gonzalez    & & \\

\hline 
\bf Descripción: &\multicolumn{3}{|p{9cm}|}{ Permite la creación de nuevas palabras claves en caso de que se requiera para registro de un Documento y que den información sobre el tema que trata este Documento.} \\
\hline 
\bf Actores: &\multicolumn{3}{|p{9cm}|}{ Catalogador, Administrador } \\
\hline
\bf Precondiciones: &\multicolumn{3}{|p{9cm}|}{Haber ingresado al Sistema y tener perfil de Catalogador o Administrador.} \\
\hline
\multicolumn{4}{|c|}{\bf {Flujo Normal}}\\
\hline
\multicolumn{2}{|c|} {\bf Actor } & \multicolumn{2}{|c|}{\bf Sistema } \\
\hline
\multicolumn{2}{|p{6cm}|} {
\begin{itemize}
\item[1. ]El caso de uso inicia cuando el Usuario solicita crear una nueva palabra clave sobre un documento relacionado con las ciencias de la computación.
\item[3.] El Usuario ingresa datos en los campos proporcionado por la interfaz del Sistema para creación de nuevas palabras claves.
\item[4. ]El Usuario indica al Sistema que ya a ingresado los datos y que desea crear la nueva área.
\item[7. ]El acepta el mensaje de confirmación generado por el Sistema y el caso de uso finaliza.
\end{itemize}
} 
 & \multicolumn{2}{|p{6cm}|}{
 \begin{itemize}
\item[2.] El Sistema responde mostrando una interfaz con dos campos: nombre y descripción del documento.
\item[5.] El Sistema valida que el nombre de la palabra clave que ha ingresado el Usuario para la nueva palabra clave no exista como nombre de otra palabra clave
\item[6. ]El Sistema crea una nueva palabra clave con respecto a un documento de ciencias de la computación en el sistema y responde con un mensaje al Usuario indicando el éxito de la operación.
\end{itemize}
} \\
\hline
\multicolumn{4}{|c|}{\bf {Flujo Alternativo}}\\
\hline
\multicolumn{2}{|p{6cm}|} {
\begin{itemize}
\item[3.] El Usuario acepta el mensaje de notificación del error generado por el Sistema.
\end{itemize}} &
 \multicolumn{2}{|p{6cm}|}  {
 \begin{itemize}
\item[1.] El Sistema al realizar la validación del nombre y se percata de que el nombre dado a la nueva palabra clave ya existe.
\item[2.] El Sistema genera un mensaje indicando que el nombre dado a la palabra clave no se puede utilizar porque ya existe un palabra clave con ese nombre.
\end{itemize}}\\
\hline
\bf Poscondiciones: &\multicolumn{3}{|p{9cm}|}{El Sistema añade un registro correspondiente a las Palabras Clave que pueden llevar el Documento.} \\
\hline
\bf Excepciones: &\multicolumn{3}{|p{9cm}|}{Fallo de conexión en la base de datos. Falla en el sistema de suministro de energía.} \\
\hline
\bf aprobado por : &   & \bf Fecha &  \\
\hline
\end{longtable}
\end{center}
%\end{document} 
