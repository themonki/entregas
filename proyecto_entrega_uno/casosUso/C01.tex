%\documentclass[10pt,a4paper]{article}
%
%
%\usepackage[spanish]{babel}
%\usepackage[utf8]{inputenc}
%\usepackage{geometry}
%\usepackage{colortbl}
%\usepackage{longtable}
%\usepackage{graphicx}
%
%\geometry{tmargin=1cm,bmargin=2cm,lmargin=2cm,rmargin=2cm}
%\begin{document}
\begin{center}


\begin{longtable}{|p{3cm}|p{3cm}|p{3cm}|p{3cm}|}

\hline
\bf {Empresa:} & \multicolumn{2}{|p{6cm}|}  { Escuela de Ingeniería de Sistemas y Computación }  & {\includegraphics[width=80.5pt]{LOGO}} \\
\hline
\bf {Nombre del caso de uso:}&\multicolumn{3}{|p{6cm}|}{Autenticar} \\
\hline 
\bf Codigo: & CU01  &\bf Fecha: &  \\

\hline 
\bf Autor(es ): & Edgar Andrés Moncada  & & \\

\hline 
\bf Descripcion: &\multicolumn{3}{|p{9cm}|}{ Permite el ingreso y autenticación de los usuarios al sistema} \\
\hline 
\bf Actores: &\multicolumn{3}{|p{9cm}|}{  Administrador, Catalogador, Usuario Normal } \\
\hline
\bf Precondiciones: &\multicolumn{3}{|p{9cm}|}{ Estar registrado en la biblioteca.  } \\
\hline
\multicolumn{4}{|c|}{\bf {Flujo Normal}}\\
\hline
\multicolumn{2}{|c|} {\bf Actor } & \multicolumn{2}{|c|}{\bf Sistema } \\
\hline
\multicolumn{2}{|p{6cm}|} {
\begin{itemize}
\item[1.] El Usuario selecciona la opción ingresar a la biblioteca.
\item[3.] El Usuario rellena los respectivos campos con su login y su contraseña y selecciona la opción ingresar.
\end{itemize}
} 
 & \multicolumn{2}{|p{6cm}|}{
\begin{itemize}
\item[2.] El sistema muestra los campos de login y contraseña.
\item[4.] El sistema verifica que el login exista y que la contraseña corresponde con la que se tiene almacenada.
\item[5.] El sistema actualiza el estado del usuario y el caso de uso termina.
\end{itemize}
} \\
\hline
\multicolumn{4}{|c|}{\bf {Flujo Alternativo}}\\
\hline
\multicolumn{2}{|p{6cm}|} { 
\begin{itemize}
\item[1. A.] El Usuario rellena el campo login con uno que no está registrado en el sistema.
\item[1. B.] El usuario digita mal la contraseña en el campo.
\end{itemize}} &
 \multicolumn{2}{|p{6cm}|}  { 
\begin{itemize}
\item[1. A.] El sistema al verificar se da cuenta que el login no existe
\item[3. A.] El sistema indica que no existe el login y pide los datos de nuevo.
\item[2. A. ]El sistema al verificar se da cuenta que la contraseña es errónea.
\item[3. B.] El sistema envía una notificación indicando que la contraseña es errónea y que repita el proceso.
\end{itemize}}\\
\hline
\bf Poscondiciones: &\multicolumn{3}{|p{9cm}|}{El sistema actualiza el perfil de usuario. El usuario identificado esta activo en el sistema.} \\
\hline
\hline
\bf Excepciones: &\multicolumn{3}{|p{9cm}|}{ El sistema no puede acceder a la base de datos y no puede verificar el login y/o la contraseña del usuario.} \\
\hline
\hline
\bf aprobado por : &   & \bf Fecha &  \\
\hline
\end{longtable}
\end{center}

%\end{document} 
