%\documentclass[10pt,a4paper]{article}
%
%
%\usepackage[spanish]{babel}
%\usepackage[utf8]{inputenc}
%\usepackage{geometry}
%\usepackage{colortbl}
%\usepackage{longtable}
%\usepackage{graphicx}
%
%\geometry{tmargin=1cm,bmargin=2cm,lmargin=2cm,rmargin=2cm}
%\begin{document}
\begin{center}


\begin{longtable}{|p{3cm}|p{3cm}|p{3cm}|p{3cm}|}

\hline
\bf {Empresa:} & \multicolumn{2}{|p{6cm}|}  { Escuela de Ingeniería de Sistemas y Computación }  & {\includegraphics[width=80.5pt]{LOGO}} \\
\hline
\bf {Nombre del caso de uso:}&\multicolumn{3}{|p{6cm}|}{Catalogar } \\
\hline 
\bf Codigo: & CU06  &\bf Fecha: & \\

\hline 
\bf Autor(es ): &Yerminson Gonzalez    & & \\

\hline 
\bf Descripción: &\multicolumn{3}{|p{9cm}|}{ Permite la creación de un nuevo registro para un Documento mediante su adecuada catalogación es decir proporcionando los datos adecuados para su identificación y futura consulta.} \\
\hline 
\bf Actores: &\multicolumn{3}{|p{9cm}|}{ Catalogador } \\
\hline
\bf Precondiciones: &\multicolumn{3}{|p{9cm}|}{Haberse autenticado en el Sistema y tener perfil de Catalogador o Administrador.} \\
\hline
\multicolumn{4}{|c|}{\bf {Flujo Normal}}\\
\hline
\multicolumn{2}{|c|} {\bf Actor } & \multicolumn{2}{|c|}{\bf Sistema } \\
\hline
\multicolumn{2}{|p{6cm}|} {
\begin{itemize}
\item[1.] El caso de uso inicia cuando el catalogador le solicita al sistema que permita catalogar un documento.
\item[3. ]El catalogador llena los campos correspondientes de acuerdo a la información que muestra el libro haciendo uso del listado que presenta algunos campos y otros que si se llenan mediante la especificación escrita del dato.  
\item[4. ]El Catalogador solicita que sean verficados los datos en el Sistema al seleccionar la opción Almacenar Documento.
\item[7. ]El usuario acepa el mensaje de éxito y así termina este caso de uso.
\end{itemize}
} 
 & \multicolumn{2}{|p{6cm}|}{
 \begin{itemize}
\item[ 2.] El sistema le responde enviando una interfaz con los campos correspondientes a la información que se requiere para poder crear un nuevo registro de documento , los campos son: Tipo de material, número de identificación, título principal, título secundario y/o traducido, editorial, fecha publicación, fecha  catalogación, Fecha creación, Idioma, derechos de autor y una breve descripción o resumen del material.
\item[5.] El sistema valida los datos suministrados a través de la interfaz.
\item[6.] El sistema responde mostrando un mensaje de exito que indica que la operación de registro del nuevo documento se llevo con éxito.
\end{itemize}
} \\
\hline
\multicolumn{4}{|c|}{\bf {Flujo Alternativo}}\\
\hline
\multicolumn{2}{|p{6cm}|}{
\begin{itemize}
\item[1.A] 1 El Usuario mientras sumistraba los datos lleno de manera incorrecta información correspondiente al documento y envia la solicitud de guardado al Sistema.
\item[1.B] El Usuario digita en el campo identificación, uno que hace referencia a un  documento ya existente.
\item[2.B] El Usuario solicita al sistema que se validen los datos.
\item[4.B] El Usuario acepta el mensaje e intenta nuevamente.
\item[1.C] El Usuario intenta llenar el campo autor o área a partir de la lista desplegable  y ve que este no se puede listar de las áreas ya existentes o los autores.
\item[2.C] El Usuario debe suspender el registro del documento y realizar el registro tanto del área como el autor del documento que no fueron hallados. 
\item[3.C] El Usuario solicita que se cancele el proceso de registro de un documento. 
\item[5.C] El Usuario confirma la acción diciendo que si desea salir y selecciona la opción para que se de la solicitud.
\end{itemize}
}
 &
 \multicolumn{2}{|p{6cm}|} {
 \begin{itemize}
\item[2.A]El Sistema responde enviando un mensaje de error informando que alguno de los campos no se ha llenado de manera correcta y que debe ser corregido.
\item[3.B]El Sistema le responde con un mensaje que le informa que este documento ya existe y que por favor vuelva a intentar catalogar el documento.
\item[4.C]El Sistema le envía un mensaje con una confirmación que indica si realmente desea salir del proceso de registro.
\item[6.C] El Sistema le informa que se ha cancelado el proceso.
\end{itemize}
}\\
\hline
\bf Poscondiciones: &\multicolumn{3}{|p{9cm}|}{El sistema añade un registro correspondiente a un nuevo documento.} \\
\hline
\bf Excepciones: &\multicolumn{3}{|p{9cm}|}{ Fallo de conexionen la base de datos. Falla en el sistema de suministro de energía.} \\
\hline
\bf aprobado por : &   & \bf Fecha &  \\
\hline
\end{longtable}
\end{center}

%\end{document} 
