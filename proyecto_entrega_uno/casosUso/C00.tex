%\documentclass[10pt,a4paper]{article}
%\usepackage[spanish]{babel}
%\usepackage[utf8]{inputenc}
%\usepackage{geometry}
%\usepackage{colortbl}
%\usepackage{longtable}
%\usepackage{graphicx}
% 
%
%\geometry{tmargin=1cm,bmargin=2cm,lmargin=2cm,rmargin=2cm}
%\begin{document}
\begin{center}


\begin{longtable}{|p{3cm}|p{3cm}|p{3cm}|p{3cm}|}

\hline
\bf {Empresa:} & \multicolumn{2}{|p{6cm}|}  { Escuela de Ingeniería de Sistemas y Computación }  & {\includegraphics[width=80.5pt]{LOGO}} \\

\hline
\bf {Nombre del caso de uso:}&\multicolumn{3}{|p{6cm}|}{Registro de Usuario} \\
\hline 
\bf Codigo: & CU00  &\bf Fecha: & \\

\hline 
\bf Autor(es ): & Edgar Andrés Moncada  & Yerminson Gonzalez &  \\

\hline 
\bf Descripcion: &\multicolumn{3}{|p{9cm}|}{ Permite el registro al sistema de nuevos usuarios.} \\
\hline 
\bf Actores: &\multicolumn{3}{|p{9cm}|}{ Usuario No Registrado } \\
\hline
\bf Precondiciones: &\multicolumn{3}{|p{9cm}|}{ El sistema debe de haberse iniciado } \\
\hline
\multicolumn{4}{|c|}{\bf {Flujo Normal}}\\
\hline
\multicolumn{2}{|c|} {\bf Actor } & \multicolumn{2}{|c|}{\bf Sistema } \\
\hline
\multicolumn{2}{|p{6cm}|} {
\begin{itemize}
\item [1.]El caso de uso inicia cuando el usuario no registrado selecciona la opción de registrarse.
\item [3.] El usuario digíta los campos pedidos.
\item[4.] El usuario le indica al sistema que valide los datos.
\end{itemize}
 } 
 & \multicolumn{2}{|p{6cm}|}{ 
\begin{itemize}
\item[2.] El sistema genera una interfaz que le pide al usuario los siguientes datos obligatorios: login, contraseña, verificación de la contraseña, pregunta y respuesta secreta, nombre. A demás pide los siguientes datos que no son obligatorios: apellidos, género, fecha de nacimiento, email, nivel de escolaridad, vinculo con Univalle (estudiante de pregrado, de posgrado, egresado, profesor activo, jubilado, ninguno). Se pide información sobre las aéreas de interés.
\item[5.] El sistema valida los campos digitados.
\item[6.] El sistema crea una nueva instancia del usuario en la base de datos.
\item[7.] El sistema envía un mensaje de confirmación al usuario.
\end{itemize}} \\
\hline
\multicolumn{4}{|c|}{\bf {Flujo Alternativo}}\\
\hline
\multicolumn{2}{|p{6cm}|} { 
\begin{itemize}
\item[1. A.] El usuario digitó un login existente y selecciona la opción de registrarse.
\item[1. B.] El usuario digitó mal el campo de contraseña y el campo de verificación de contraseña (no son iguales) y selecciona la opción registrarse.
\item[1. C.] El usuario no diíita ninguno de los campos obligatorios y selecciona la opción registrarse.
\end{itemize}
 } &
 \multicolumn{2}{|p{6cm}|}  {  
\begin{itemize}
\item[2.A.] El sistema al validar se da cuenta que el login ya existe.
\item[3.A.] El sistema envía una notificación al usuario indicándole que ingrese un nuevo login.
\item[2.B.] El sistema al validar se da cuenta que el campo de contraseña y el campo de verificación de contraseña no son iguales.
\item[3.B.] El sistema envía una notificación al usuario indicándole que las contraseñas no coinciden y que las vuelva a confirmar.
\item[2.C.] El sistema al validar se da cuenta que los campos obligatorios no tienen datos.
\item[3.C.] El sistema envía una notificación al usuario indicándole que vuelva a confirmar. 
\end{itemize}
}\\
\hline
\bf Poscondiciones: &\multicolumn{3}{|p{9cm}|}{ El sistema almacena los datos del usuario en la base de datos. } \\
\hline
\hline
\bf Excepciones: &\multicolumn{3}{|p{9cm}|}{ El sistema no puede acceder a la base de datos y no puede almacenar los datos. } \\
\hline
\hline
\bf aprobado por : &   & \bf Fecha &  \\
\hline
\end{longtable}
\end{center}

%\end{document} 
