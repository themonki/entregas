%\documentclass[10pt,a4paper]{article}
%
%
%\usepackage[spanish]{babel}
%\usepackage[utf8]{inputenc}
%\usepackage{geometry}
%\usepackage{colortbl}
%\usepackage{longtable}
%\usepackage{graphicx}
%
%\geometry{tmargin=1cm,bmargin=2cm,lmargin=2cm,rmargin=2cm}
%\begin{document}
\begin{center}


\begin{longtable}{|p{3cm}|p{3cm}|p{3cm}|p{3cm}|}

\hline
\bf {Empresa:} & \multicolumn{2}{|p{6cm}|}  { Escuela de Ingeniería de Sistemas y Computación }  & {\includegraphics[width=80.5pt]{LOGO}} \\
\hline
\bf {Nombre del caso de uso:}&\multicolumn{3}{|p{6cm}|}{Modificar usuarios} \\
\hline 
\bf Codigo: & CU02  &\bf Fecha: & \\

\hline 
\bf Autor(es ): & Edgar Andrés Moncada  & Yerminson Gonzalez & \\

\hline 
\bf Descripcion: &\multicolumn{3}{|p{9cm}|}{  Permite la modificación de usuarios , es decir toda la información almacenada en el sistema.} \\
\hline 
\bf Actores: &\multicolumn{3}{|p{9cm}|}{  Administrador } \\
\hline
\bf Precondiciones: &\multicolumn{3}{|p{6cm}|}{ El administrador debe haber consultado sobre algún usuario.} \\
\hline
\multicolumn{4}{|c|}{\bf {Flujo Normal}}\\
\hline
\multicolumn{2}{|c|} {\bf Actor } & \multicolumn{2}{|c|}{\bf Sistema } \\
\hline
\multicolumn{2}{|p{6cm}|} {
\begin{itemize}
\item[1. ]El caso de uso inicia cuando el administrador selecciona al usuario de la lista resultado.
\item[3.] El administrador modifica los datos correspondientes de acuerdo a sus necesidades o ha una actualización de datos con la opción de modificar su perfil en caso de requerirlo.
\end{itemize}
} 
 & \multicolumn{2}{|p{6cm}|}{
 \begin{itemize}
\item[2.] El sistema responde a través de una interfaz que muestra los campos referente a sus atributos, donde se pueden modificar todos los datos y con una opción especial para asignarle un perfil de usuario : Usuario normal, Catalogador o administrador.
\item[4. ]El sistema valida que todos los datos obligatorios se hayan suministrado.
\item[5. ]El sistema accede a la base de datos y realiza una verificación de los datos nuevos y existentes para que no hayan datos iguales donde no pueden serlo.
\item[6.] El sistema realiza la actualización de los datos en la base de datos con los datos suministrados según la preferencia del administrador.
\item[7.] El sistema responde a través de una interfaz de éxito donde muestra que se ha completado la operación.
\end{itemize}
} \\
\hline
\multicolumn{4}{|c|}{\bf {Flujo Alternativo}}\\
\hline
\multicolumn{2}{|p{6cm}|} { 
\begin{itemize}
\item[1.] El administrador modifica los datos correspondientes pero los llena de manera incorrecta, por ejemplo escribiendo un login ya existente.
\end{itemize}} &
 \multicolumn{2}{|p{6cm}|}  { 
 \begin{itemize}
\item[2.] El sistema valida los datos encuentra el problema y responde a través de una interfaz que muestra los campos referentes a los atributos para que sean llenados nuevamente, con información del error cometido.
\end{itemize}
}\\
\hline
\bf Poscondiciones: &\multicolumn{3}{|p{9cm}|}{Atributos de usuario en la base de datos modificados, o su perfil.} \\
\hline
\bf Excepciones: &\multicolumn{3}{|p{9cm}|}{ La base de datos puede desconectarse debido a fallos de energía.} \\
\hline
\bf aprobado por : &   & \bf Fecha &  \\
\hline
\end{longtable}
\end{center}

%\end{document} 
