%\documentclass[10pt,a4paper]{article}
%
%
%\usepackage[spanish]{babel}
%\usepackage[utf8]{inputenc}
%\usepackage{geometry}
%\usepackage{colortbl}
%\usepackage{longtable}
%\usepackage{graphicx}
%
%\geometry{tmargin=1cm,bmargin=2cm,lmargin=2cm,rmargin=2cm}
%\begin{document}
\begin{center}


\begin{longtable}{|p{3cm}|p{3cm}|p{3cm}|p{3cm}|}

\hline
\bf {Empresa:} & \multicolumn{2}{|p{6cm}|}  { Escuela de Ingeniería de Sistemas y Computación }  & {\includegraphics[width=80.5pt]{LOGO}} \\
\hline
\bf {Nombre del caso de uso:}&\multicolumn{3}{|p{6cm}|}{Ingresar Autor} \\
\hline 
\bf Codigo: & CU09  &\bf Fecha: & \\

\hline 
\bf Autor(es ): & Yerminson Gonzalez    & Cristian Leonardo Ríos López & \\

\hline 
\bf Descripción: &\multicolumn{3}{|p{9cm}|}{ Permite la creación de nuevos autores en el Sistema.} \\
\hline 
\bf Actores: &\multicolumn{3}{|p{9cm}|}{ Administrador, Catalogador } \\
\hline
\bf Precondiciones: &\multicolumn{3}{|p{9cm}|}{Tener el perfil de usuario Administrador o Catalogador.} \\
\hline
\multicolumn{4}{|c|}{\bf {Flujo Normal}}\\
\hline
\multicolumn{2}{|c|} {\bf Actor } & \multicolumn{2}{|c|}{\bf Sistema } \\
\hline
\multicolumn{2}{|p{6cm}|} {
\begin{itemize}
\item[1. ]El caso de uso inicia cuando el Usuario solicita crear un nuevo autor.
\item[3.] El Usuario ingresa datos en los campos proporcionado por la interfaz del sistema para creación de nuevos autores.
\item[4. ]El Usuario indica al Sistema que ya a ingresado los datos y que desea crear el nuevo autor.
\item[7.] El Usuario acepta el mensaje de confirmación generado por el sistema y el caso de uso finaliza.
\end{itemize}
} 
 & \multicolumn{2}{|p{6cm}|}{
 \begin{itemize}
\item[2.]El Sistema responde mostrando una interfaz con cuatro campos:  nombre, apellido, correo electrónico y el acrónimo.
\item[5.]El Sistema valida los datos suministrados por el usuario para llevar a acabo el proceso de creación.
\item[6. ]6. El Sistema crea un nuevo autor en el sistema y responde con un mensaje al Usuario indicando el éxito de la operación. 
\end{itemize}
} \\
\hline
\multicolumn{4}{|c|}{\bf {Flujo Alternativo}}\\
\hline
\multicolumn{2}{|p{6cm}|} {
\begin{itemize}
\item[1.] El usuario digita de manera incorrecta los datos y solicita al sistema que se validen.
\item[4.] El usuario acepta el mensaje e intenta nuevamente realizar el registro.
\end{itemize}} &
 \multicolumn{2}{|p{6cm}|}  {
 \begin{itemize}
\item[2.] El sistema verifica los datos y encuentra el error.
\item[3.] El sistema envia un mensaje notificando del error a la hora de llenar los campos.
\end{itemize}}\\
\hline
\bf Poscondiciones: &\multicolumn{3}{|p{9cm}|}{El Sistema crea una nueva tupla de la base de datos correspondiente al nuevo autor.} \\
\hline
\bf Excepciones: &\multicolumn{3}{|p{9cm}|}{El Sistema no puede acceder a la base de datos y no puede crear la nueva tupla o realizar las consultas para las validaciones de los datos.} \\
\hline
\bf aprobado por : &   & \bf Fecha &  \\
\hline
\end{longtable}
\end{center}
%\end{document} 
