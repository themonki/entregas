%\documentclass[10pt,a4paper]{article}
%
%
%\usepackage[spanish]{babel}
%\usepackage[utf8]{inputenc}
%\usepackage{geometry}
%\usepackage{colortbl}
%\usepackage{longtable}
%\usepackage{graphicx}
%\geometry{tmargin=1cm,bmargin=2cm,lmargin=2cm,rmargin=2cm}
%\begin{document}
\begin{center}


\begin{longtable}{|p{3cm}|p{3cm}|p{3cm}|p{3cm}|}

\hline
\bf {Empresa:} & \multicolumn{2}{|p{6cm}|}  { Escuela de Ingeniería de Sistemas y Computación }  & {\includegraphics[width=80.5pt]{LOGO}} \\
\hline
\bf {Nombre del caso de uso:}&\multicolumn{3}{|p{9cm}|}{Consultar Usuarios} \\
\hline 
\bf Codigo: & CU05  &\bf Fecha: & \\

\hline 
\bf Autor(es ): & Edgar Andrés Moncada  & Yerminson Gonzalez & \\

\hline 
\bf Descripcion: &\multicolumn{3}{|p{9cm}|}{ Permite la consulta de Usuarios que están registrados en el Sistema.} \\
\hline 
\bf Actores: &\multicolumn{3}{|p{9cm}|}{  	Administrador } \\
\hline
\bf Precondiciones: &\multicolumn{3}{|p{9cm}|}{El Usuario debe haberse autenticado en el Sistema y tener perfil de Administrador.} \\
\hline
\multicolumn{4}{|c|}{\bf {Flujo Normal}}\\
\hline
\multicolumn{2}{|c|} {\bf Actor } & \multicolumn{2}{|c|}{\bf Sistema } \\
\hline
\multicolumn{2}{|p{6cm}|} {
\begin{itemize}
\item[1. ]El caso de uso inicia cuando el Administrador solicita la consulta de usuarios.
\item[3. ]El Administrador escribe las palabras claves con respecto a la búsqueda y solicita al Sistema que se realiza la búsqueda.
\end{itemize}
} 
 & \multicolumn{2}{|p{6cm}|}{
 \begin{itemize}
 \item[2.] El Sistema responde a través de una interfaz que permite introducir las palabras claves por las cuales se desea buscar algún Usuario.
\item[4.] El Sistema envía la consulta a la base de datos y con la información obtenida genera una interfaz donde muestra de manera organiza los resultados obtenidos.
\item[5.] El Sistema envía esta interfaz al usuario como resultado de su consulta y el caso de uso termina.
\end{itemize}
} \\
\hline
\multicolumn{4}{|c|}{\bf {Flujo Alternativo}}\\
\hline
\multicolumn{2}{|p{6cm}|} { 
\begin{itemize}
\item[1.] El Administrador escribe como palabras clave de la consulta algo que no se encuentra en la base de datos.
\end{itemize}
} &
 \multicolumn{2}{|p{6cm}|}  {
 \begin{itemize}
  \item[2.] El Sistema realiza la consulta a la base de datos, y obtiene como resultado una consulta vacía.
\item[3.] Envía un mensaje informándole al Usuario de que no se encontraron coincidencias para esa búsqueda..
\end{itemize}
}\\
\hline
\bf Poscondiciones: &\multicolumn{3}{|p{9cm}|}{Se crea una vista temporal de los Usuarios que han sido consultados.} \\
\hline
\bf Excepciones: &\multicolumn{3}{|p{9cm}|}{ La base de datos puede desconectarse debido a fallos de energía.} \\
\hline
\bf aprobado por : &   & \bf Fecha &  \\
\hline
\end{longtable}
\end{center}

%\end{document} 
