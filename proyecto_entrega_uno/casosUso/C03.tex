%\documentclass[10pt,a4paper]{article}
%
%
%\usepackage[spanish]{babel}
%\usepackage[utf8]{inputenc}
%\usepackage{geometry}
%\usepackage{colortbl}
%\usepackage{longtable}
%\usepackage{graphicx}
%
%\geometry{tmargin=1cm,bmargin=2cm,lmargin=2cm,rmargin=2cm}
%\begin{document}
\begin{center}


\begin{longtable}{|p{3cm}|p{3cm}|p{3cm}|p{3cm}|}

\hline
\bf {Empresa:} & \multicolumn{2}{|p{6cm}|}  { Escuela de Ingeniería de Sistemas y Computación }  & {\includegraphics[width=80.5pt]{LOGO}} \\
\hline
\bf {Nombre del caso de uso:}&\multicolumn{3}{|p{6cm}|}{ Modificar Datos Usuario} \\
\hline 
\bf Codigo: & CU03  &\bf Fecha: & \\

\hline 
\bf Autor(es ): & Edgar Andrés Moncada  & Yerminson Gonzalez & \\

\hline 
\bf Descripcion: &\multicolumn{3}{|p{9cm}|}{  Permite la modificación de algunos datos del Usuario.} \\
\hline 
\bf Actores: &\multicolumn{3}{|p{9cm}|}{  Usuario, Catalogador	Administrador } \\
\hline
\bf Precondiciones: &\multicolumn{3}{|p{9cm}|}{ El Usuario debe haberse autenticado en el Sistema.} \\
\hline
\multicolumn{4}{|c|}{\bf {Flujo Normal}}\\
\hline
\multicolumn{2}{|c|} {\bf Actor } & \multicolumn{2}{|c|}{\bf Sistema } \\
\hline
\multicolumn{2}{|p{6cm}|} {
\begin{itemize}
\item[1. ]El caso de uso inicia cuando el Usuario solicita editar su perfil.
\item[3. ]El Usuario modifica los datos correspondientes de acuerdo a sus necesidades o ha una actualización de datos.
\end{itemize}
} 
 & \multicolumn{2}{|p{6cm}|}{
 \begin{itemize}
\item[2. ]El Sistema responde a través de una interfaz que muestra los campos referentes a sus atributos, con los siguientes datos editables: contraseña, pregunta y respuesta secreta, nombre, apellidos, género, fecha nacimiento, nivel escolaridad, vinculo con univalle y áreas de interés. Los demás campos no son editables.
\item[4. ]El Sistema valida de que los campos obligatorios contengan datos.
\item[5. ]El Sistema accede a la base de datos y valida que los datos existentes y los nuevos datos no sean iguales cuando no pueden serlo.
\item[6. ]El Sistema accede a la base de datos, y actualiza los valores de acuerdo a los suministrados por el Usuario.
\item[7. ]El Sistema responde a través de una interfaz de éxito donde muestra que se ha completado la operación.
 \end{itemize}
} \\
\hline
\multicolumn{4}{|c|}{\bf {Flujo Alternativo}}\\
\hline
\multicolumn{2}{|p{6cm}|} { \begin{itemize}
\item[1.] El Usuario modifica los datos correspondientes pero los llena de manera incorrecta.
\end{itemize}} &
 \multicolumn{2}{|p{6cm}|}  {
 \begin{itemize}
\item[2.] El Sistema al validar los datos encuentra un error.
\item[3.]El Sistema responde a través de una interfaz que muestra los campos referentes a los atributos para que sean llenados nuevamente, con información del error cometido.
\end{itemize} 
}\\
\hline
\bf Poscondiciones: &\multicolumn{3}{|p{9cm}|}{Los datos modificados del Usuario se almacenan en el Sistema.} \\
\hline
\bf Excepciones: &\multicolumn{3}{|p{9cm}|}{ La base de datos puede desconectarse debido a fallos de energía.} \\
\hline
\bf aprobado por : &   & \bf Fecha &  \\
\hline
\end{longtable}
\end{center}

%\end{document} 
