%\documentclass[]{article}
%\usepackage[spanish]{babel}
%\usepackage[utf8]{inputenc}
%\usepackage{geometry}
%\usepackage{colortbl}
%\usepackage{longtable}
%\geometry{tmargin=3cm,bmargin=3cm,lmargin=3cm,rmargin=2cm}
%\begin{document}
%para incluir comentar hasta acá
\begin{center}
\begin{longtable}{|p{0.225\textwidth}|p{0.225\textwidth}|p{0.225\textwidth}|p{0.225\textwidth}|}
\hline
\multicolumn{2}{|p{0.45\textwidth}|}{{\bf {Descripción del requerimiento:}}
Catalogar documentos. } & {\bf{ Estado}} & Análisis \\
\hline
\multicolumn{2}{|p{0.45\textwidth}}{\bf Identificador} &
\multicolumn{2}{|p{0.45\textwidth}|}{R08} \\
\hline
\multicolumn{2}{|p{0.45\textwidth}}{\bf {Tipo de requerimiento}} &
\multicolumn{2}{|p{0.45\textwidth}|}{Funcional}\\
\hline
\bf {Creado por} & Maria Andrea Cruz & \bf {Fecha } & Mayo 08 2011 \\
\hline
\bf {Actualizado por} & Cristian Ríos & \bf {Fecha }& Mayo 09 2011\\
\hline
\bf Descripción &\multicolumn{3}{p{0.675\textwidth}|}
{El sistema debe de proporcionar la posibilidad a los usuarios que tengan como perfil catalogador o administrador de catalogar documentos, esto es, copiar un nuevo documento al repositorio del sistema y asignarle datos que lo identifiquen en el mismo, estos datos son información sobre el documento como el autor, el titulo principal, el título secundario o traducido, la editorial, la fecha de publicación, la fecha de creación, el idioma en el que esta escrito, los derechos de autor, la descripción o resumen del material, las palabras claves relacionadas con este, el área de las ciencias de la computación a la que pertenece, el formato del archivo, el tamaño del archivo en bytes, la resolución en píxeles y el software recomendado para abrir el documento. Para copiar el documento al repositorio el usuario debe de indicar cual es la ruta actual del documento en el sistema host.} \\
\hline
\bf Datos de entrada &\multicolumn{3}{p{0.675\textwidth}|}{
El usuario que este realizando la operación de catalogación debe de proporciona de manera obligatoria los siguientes datos sobre el documento: el título principal, el idioma en el que esta escrito el documento, la descripción del documento, el formato, el elnace  a donde se encuentra actualmente el documento, el o los autores del documento, las palabras clave relacionadas con el documento, la o las áeras de las ciencias de la computación a la que pertence, los derechos de autor, la editorial, la fecha de publicación y el tipo al que pertenece el documento. Puede proporcinar además el título secundario o traducido, la fecha de creación, el tamaño del archivo en bytes, la resolución en pixeles(si aplica) y el software recomendado para abrir el documento. }\\
\hline
\bf Datos de salida &\multicolumn{3}{p{0.675\textwidth}|}
{ Se informa al usuario que solicitó la catalogación del documento éxito o no de la operación.} \\
\hline
\bf Resultados esperados &\multicolumn{3}{p{0.675\textwidth}|}
{ El sistema deberá realizar actualizaciones en la base de datos creando nuevos registros en las tablas que mantienen la información de los documentos con la información proporcionada por el usuario.} \\
\hline
\bf Origen &\multicolumn{3}{p{0.675\textwidth}|}
{Documento de descripción del problema.} \\
\hline
\bf Dirigido a &\multicolumn{3}{p{0.675\textwidth}|}
{Catalogador, administrador.} \\
\hline
\bf Prioridad &\multicolumn{3}{p{0.675\textwidth}|}{3} \\
\hline
\bf Requerimientos Asociados &\multicolumn{3}{p{0.675\textwidth}|}
{\begin{itemize}
        \item R06
        \item R17
        \item R14
        \item R18
\end{itemize}} \\\hline
\multicolumn{4}{|>{\columncolor[rgb]{0.8,0.8,0.8}}c|}{\bf Especificación}\\
\hline
\bf Precondiciones &\multicolumn{3}{p{0.675\textwidth}|}
{El sistema debe estar conectado a la base de datos y tener logueado a un usuario con los perfiles correspondientes a administrador o catalogador los cuales tienen acceso a una interfaz de catalogación, la cual permite proporcionar información acerca del documento a catalogar. Los autores, palabras clave y áreas de ciencias de la computación asociados con el documento deben de existir previamente en el sistema.} \\
\hline
\bf Poscondicion &\multicolumn{3}{p{0.675\textwidth}|}
{El sistema en su base de datos tiene ahora información de un documento, con esta información el documento podrá ser consultado y descargado posteriormente.} \\
\hline
\bf Criterios de Aceptación &\multicolumn{3}{p{0.675\textwidth}|}
{El requerimiento es aceptado si los usuarios con perfil administrador o catalogador puede adicionar documentos al sistema y proporcionar información sobre ellos, esto es catalogar el documento.} \\
\hline
\end{longtable}
\end{center}
%\end{document} %comentar para inlcuir