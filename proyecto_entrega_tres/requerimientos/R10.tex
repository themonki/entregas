%\documentclass[]{article}
%\usepackage[spanish]{babel}
%\usepackage[utf8]{inputenc}
%\usepackage{geometry}
%\usepackage{colortbl}
%\usepackage{longtable}
%\geometry{tmargin=3cm,bmargin=3cm,lmargin=3cm,rmargin=2cm}
%\begin{document} %comentar hasta esta linea para incluir
\begin{center}
\begin{longtable}{|p{0.225\textwidth}|p{0.225\textwidth}|p{0.225\textwidth}|p{0.225\textwidth}|}
\hline
\multicolumn{2}{|p{0.45\textwidth}|}{{\bf {Función del requerimiento:}}
Llevar un registro de las consultas a documentos realizadas. } & {\bf{ Estado}} & Análisis \\
\hline
\multicolumn{2}{|p{0.45\textwidth}}{\bf Identificador} &
\multicolumn{2}{|p{0.45\textwidth}|}{R10} \\
\hline
\multicolumn{2}{|p{0.45\textwidth}}{\bf {Tipo de requerimiento}} &
\multicolumn{2}{|p{0.45\textwidth}|}{Funcional}\\
\hline
\bf {Creado por} & Cristian Ríos& \bf {Fecha  } & Abril 29 2011\\
\hline
\bf {Actualizado por} & Cristian Ríos & \bf {Fecha  }& Mayo 01 2011\\
\hline
\bf {Actualizado por} & Cristian Ríos & \bf {Fecha  }& Mayo 10 2011\\
\hline
\bf Descripción &\multicolumn{3}{p{0.675\textwidth}|}
{El sistema debe proporcionar la manera de poder llevar un registro de las consultas que se hagan a un documento, en cuanto a consulta se refiera a haber realizado una búsqueda de algún tipo, esto muestra una lista con los posibles resultados, si se selecciona algún ejemplar de esta lista se entiende que el documento a sido consultado. El registro se llevará para cualquier usuario que realice alguna búsqueda, sea registrado o no registrado. Para usuarios no resgistrado existe un usuario por defecto en el sistema.} \\
\hline
\bf Datos de entrada &\multicolumn{3}{p{0.675\textwidth}|}{
El usuario no propocionará de manera explicita datos para este requerimiento, pero para para poder llevar a cabo el registro de las consultas, cada vez que un usuario realice una consulta se proporcionará al sistema el login del usuario que realizó la consulta y el identificador del documento que consultó.}\\
\hline
\bf Datos de salida &\multicolumn{3}{p{0.675\textwidth}|}
{EL sistema no proporcionara datos al usuario que este realizando la búsqueda sobre el registro de las consultas, pero esta información es utilizada por el administrador del sistema para obtener reportes de las transacciones del sistema.} \\
\hline
\bf Resultados esperados &\multicolumn{3}{p{0.675\textwidth}|}
{El sistema deberá realizar actualizaciones en la base de datos agregando un nuevo registro  cada vez que se realice una consutla a la tabla que mantiene la información de las consultas realizadas a documentos.} \\
\hline
\bf Origen &\multicolumn{3}{p{0.675\textwidth}|}
{Documento de descripción del problema.} \\
\hline
\bf Dirigido a &\multicolumn{3}{p{0.675\textwidth}|}
{Usuarios registrado, normal, administrador y catalogador, y usuarios no registrados} \\
\hline
\bf Prioridad &\multicolumn{3}{p{0.675\textwidth}|}{4} \\
\hline
\bf Requerimientos Asociados &\multicolumn{3}{p{0.675\textwidth}|}
{\begin{itemize}
        \item R21
        \item R22
\end{itemize}} \\
\hline
\multicolumn{4}{|>{\columncolor[rgb]{0.8,0.8,0.8}}c|}{\bf Especificación}\\
\hline
\bf Precondiciones &\multicolumn{3}{p{0.675\textwidth}|}
{El sistema debe de estar conectado a la base de datos y algún usuario debe de haber realizado una búsqueda y seleccionado alguna opción arrojada por la búsqueda} \\
\hline
\hline
\bf Poscondicion &\multicolumn{3}{p{0.675\textwidth}|}
{El sistema presenta una actualización en la base de datos con respeto a la tabla que mantiene el registro de los documento  consultados al agregar un nueva nueva tupla. } \\
\hline
\bf Criterios de Aceptación &\multicolumn{3}{p{0.675\textwidth}|}
{El requerimiento es aceptado si cada vez que algún usuario realiza una consulta se mantiene información relacionada a esa consulta, por tanto es posible generar reportes.} \\
\hline
\end{longtable}
\end{center}
%\end{document}