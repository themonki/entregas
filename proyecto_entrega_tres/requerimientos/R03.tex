%\documentclass[]{article}
%\usepackage[spanish]{babel}
%\usepackage[utf8]{inputenc}
%\usepackage{geometry}
%\usepackage{colortbl}
%\usepackage{longtable}
%\geometry{tmargin=3cm,bmargin=3cm,lmargin=3cm,rmargin=2cm}
%\begin{document}
%para incluir comentar hasta acá
\begin{center}
\begin{longtable}{|p{0.225\textwidth}|p{0.225\textwidth}|p{0.225\textwidth}|p{0.225\textwidth}|}
\hline
\multicolumn{2}{|p{0.45\textwidth}|}{{\bf {Función del requerimiento:}}
Modificar datos de usuario. } & {\bf{ Estado}} & Análisis \\
\hline
\bf {Creado por} & Maria Andrea Cruz & \bf {Actualizado por} &  Cristian Ríos\\
\hline
\bf {Fecha Creación } & Marzo 31 2011 & \bf {Fecha de Actualización }& 
Abril 28 2011\\
Mayo 09 2011\\
\hline
\multicolumn{2}{|p{0.45\textwidth}}{\bf Identificador} &
\multicolumn{2}{|p{0.45\textwidth}|}{R03} \\
\hline
\multicolumn{2}{|p{0.45\textwidth}}{\bf {Tipo de requerimiento}} &
\multicolumn{2}{|p{0.45\textwidth}|}{Funcional}\\
\hline
\bf Descripción &\multicolumn{3}{p{0.675\textwidth}|}
{ El sistema debe proporcionar una opción que permita a cada usuario registrado en el sistema modificar los datos de su perfil, a excepción de su login.  Para saber el estado actual de los datos y cuales de estos modificar se realiza una consulta a la base de datos que le informa el contenido actual de cada uno de sus campos.} \\
\hline
\bf Datos de entrada &\multicolumn{3}{p{0.675\textwidth}|}{
Al usuario que intenta modificar sus datos en el sistema se le presentan todos sus datos actuales, que pueden ser cambiados a excepción de su login, nuevamente como al registrarse debe de proporcionar estos datos, el dato que no sea modificado tomará el valor que actualmente tiene.}\\
\hline
\bf Datos de salida &\multicolumn{3}{p{0.675\textwidth}|}
{ Se genera un mensaje de notificación informando al usuario del éxito o no de la operación} \\
\hline
\bf Resultados esperados &\multicolumn{3}{p{0.675\textwidth}|}
{El sistema deberá realizar una actualización en la base de datos con nueva información actualizada y correcta del usuario que solicito esta funcionalidad.} \\
\hline
\bf Origen &\multicolumn{3}{p{0.675\textwidth}|}
{Documento de descripción del problema.} \\
\hline
\bf Dirigido a &\multicolumn{3}{p{0.675\textwidth}|}
{Catalogador, administrador y usuarios normales.} \\
\hline
\bf Prioridad &\multicolumn{3}{p{0.675\textwidth}|}{3} \\
\hline
\bf Requerimientos Asociados &\multicolumn{3}{p{0.675\textwidth}|}
{\begin{itemize}
        \item R06
\end{itemize} } \\
\hline
\multicolumn{4}{|>{\columncolor[rgb]{0.8,0.8,0.8}}c|}{\bf Especificación}\\
\hline
\bf Precondiciones &\multicolumn{3}{p{0.675\textwidth}|}
{El sistema debe estar conectado a la base de datos de la misma manera debe de haber reconocido a un usuario normal es decir este debe de estar logueado y estar observando su determinada interfaz correspondiente a su perfil.} \\
\hline
\bf Poscondicion &\multicolumn{3}{p{0.675\textwidth}|}
{El sistema en la base de datos tiene información actualizada de acuerdo a la los datos que modifico el usuario. Lo que permite una mejor comunicación sistema usuario.} \\
\hline
\bf Criterios de Aceptación &\multicolumn{3}{p{0.675\textwidth}|}
{El requerimiento es aceptado si cada usuario logra modificar sus datos de usuario que mantene el sistema.} \\
\hline
\end{longtable}
\end{center}
%\end{document} %comentar para inlcuir