%\documentclass[]{article}
%\usepackage[spanish]{babel}
%\usepackage[utf8]{inputenc}
%\usepackage{geometry}
%\usepackage{colortbl}
%\usepackage{longtable}
%\geometry{tmargin=3cm,bmargin=3cm,lmargin=3cm,rmargin=2cm}
%\begin{document} %comentar hasta esta linea para incluir
\begin{center}
\begin{longtable}{|p{0.225\textwidth}|p{0.225\textwidth}|p{0.225\textwidth}|p{0.225\textwidth}|}
\hline
\multicolumn{2}{|p{0.45\textwidth}|}{{\bf {Función del requerimiento:}}
Permitir modificar palabras clave. } & {\bf{ Estado}} & Análisis \\
\hline
\bf {Creado por} & Cristian Ríos& \bf {Actualizado por} & Cristian Ríos\\
\hline
\bf {Fecha Creación } & Abril 29 2011 & \bf {Fecha de Actualización }& 
Mayo 01 2011\\
Mayo 10 2011\\
\hline
\multicolumn{2}{|p{0.45\textwidth}}{\bf Identificador} &
\multicolumn{2}{|p{0.45\textwidth}|}{R13} \\
\hline
\multicolumn{2}{|p{0.45\textwidth}}{\bf {Tipo de requerimiento}} &
\multicolumn{2}{|p{0.45\textwidth}|}{Funcional}\\
\hline
\bf Descripción &\multicolumn{3}{p{0.675\textwidth}|}
{El sistema debe proporcionar la manera de poder modificar los datos de las palabras claves existentes por usuarios que tengan como perfil catalogador o administrador. El sistema debe de proporcionar al usuario que este realizando la modificación, los datos actuales que tiene la palabra clave, de las palabras clave lo único que puede modificar es su descripción. } \\
\hline
\bf Datos de entrada &\multicolumn{3}{p{0.675\textwidth}|}{
El usuario que este realizando la operación de modificar la palabra clave debe de proporcionar la descripción de la palabra.}\\
\hline
\bf Datos de salida &\multicolumn{3}{p{0.675\textwidth}|}
{El sistema genera y muestra un mensaje de notificación al usuario que este realizando la actualización indicando el éxito o no de la operación.} \\
\hline
\bf Resultados esperados &\multicolumn{3}{p{0.675\textwidth}|}
{El sistema debera realizar una actualización en la base de datos con nueva información actualizada y correcta de la palabra clave que el usuario desea actualizar.} \\
\hline
\bf Origen &\multicolumn{3}{p{0.675\textwidth}|}
{Documento de descripción del problema.} \\
\hline
\bf Dirigido a &\multicolumn{3}{p{0.675\textwidth}|}
{Admisntrador y catalogador} \\
\hline
\bf Prioridad &\multicolumn{3}{p{0.675\textwidth}|}{3} \\
\hline
\bf Requerimientos Asociados &\multicolumn{3}{p{0.675\textwidth}|}
{\begin{itemize}
        \item R06
        \item R17
\end{itemize}} \\
\hline
\multicolumn{4}{|>{\columncolor[rgb]{0.8,0.8,0.8}}c|}{\bf Especificación}\\
\hline
\bf Precondiciones &\multicolumn{3}{p{0.675\textwidth}|}
{El sistema debe estar conectado a la base de datos y el usuario registrado debe corresponder a un catalogador o un administrador lo que les proporcionará una interfaz que permitirá listar las palabras clave, para seleccionarlas y obtener sus datos y de esta manera modificarlos.} \\
\hline
\hline
\bf Poscondicion &\multicolumn{3}{p{0.675\textwidth}|}
{El sistema presenta una actualización en la base de datos con respeto al registro de palabras clave donde se ha corregido o actualizado información, lo que permite tener una mejor experiencia por parte de los usuarios al consultar. } \\
\hline
\bf Criterios de Aceptación &\multicolumn{3}{p{0.675\textwidth}|}
{El requerimiento es aceptado si un usuario con perfil administrador o catalogador puede modificar y actualizar los datos de las palabras clave.} \\
\hline
\end{longtable}
\end{center}
%\end{document}