%\documentclass[]{article}
%\usepackage[spanish]{babel}
%\usepackage[utf8]{inputenc}
%\usepackage{geometry}
%\usepackage{colortbl}
%\usepackage{longtable}
%\geometry{tmargin=3cm,bmargin=3cm,lmargin=3cm,rmargin=2cm}
%\begin{document}
%para incluir comentar hasta acá
\begin{center}
\begin{longtable}{|p{0.225\textwidth}|p{0.225\textwidth}|p{0.225\textwidth}|p{0.225\textwidth}|}
\hline
\multicolumn{2}{|p{0.45\textwidth}|}{{\bf {Descripción del requerimiento:}}
Verificar el logueo de usuarios. } & {\bf{ Estado}} & Análisis \\
\hline
\multicolumn{2}{|p{0.45\textwidth}}{\bf Identificador} &
\multicolumn{2}{|p{0.45\textwidth}|}{R06} \\
\hline
\multicolumn{2}{|p{0.45\textwidth}}{\bf {Tipo de requerimiento}} &
\multicolumn{2}{|p{0.45\textwidth}|}{Funcional}\\
\hline
\bf {Creado por} & Maria Andrea Cruz & \bf {Fecha } & Marzo 31 2011 \\
\hline
\bf {Actualizado por} & María Andrea & \bf {Fecha }& Abril 28 2011\\
\hline
\bf {Actualizado por} & Cristian Ríos & \bf {Fecha }& Mayo 09 2011\\
\hline
\bf Descripción &\multicolumn{3}{p{0.675\textwidth}|}
{ El sistema debe proporcionar la autenticación de cualquier usuario que se encuentre registrado y desee ingresar al sistema, esto se logra realizando una consulta a la base de datos con el nombre de usuario proporcionado por el usuario, si existe el usuario en la base de datos el resultado de la consulta será la contraseña, la cual se comparará con la proporcionada por el usuario, si ambas coinciden entonces se le otorgará al usuario el acceso al sistema.} \\
\hline
\bf Datos de entrada &\multicolumn{3}{p{0.675\textwidth}|}{
El usuario debe de proporcionar su login y su contraseña para tener acceso al sistema.}\\
\hline
\bf Datos de salida &\multicolumn{3}{p{0.675\textwidth}|}
{ El sistema permite al usuario ingresar al sistema mostrando la interfaz de su respectivo perfil.} \\
\hline
\bf Resultados esperados &\multicolumn{3}{p{0.675\textwidth}|}
{ El deberá cambiar la interfaz del usuario permitiéndole nuevas funcionalidades como por ejemplo opciones de descarga de documentos.} \\
\hline
\bf Origen &\multicolumn{3}{p{0.675\textwidth}|}
{Documento de descripción del problema.} \\
\hline
\bf Dirigido a &\multicolumn{3}{p{0.675\textwidth}|}
{Catalogador, administrador y usuarios normales.} \\
\hline
\bf Prioridad &\multicolumn{3}{p{0.675\textwidth}|}{5} \\
\hline
\bf Requerimientos Asociados &\multicolumn{3}{p{0.675\textwidth}|}
{\begin{itemize}
        \item R01
\end{itemize}} \\\hline
\multicolumn{4}{|>{\columncolor[rgb]{0.8,0.8,0.8}}c|}{\bf Especificación}\\
\hline
\bf Precondiciones &\multicolumn{3}{p{0.675\textwidth}|}
{El sistema debe estar conectado a la base de datos ya que debe reconocer que el usuario esta registrado realizando una consulta a esta a partir de los datos suministrados en la interfaz de login.} \\
\hline
\hline
\bf Poscondicion &\multicolumn{3}{p{0.675\textwidth}|}
{El sistema debe de mostrar una nueva interfaz para el usuario lo que indica que este proceso se realizó satisfactoriamente, en caso de no lograrlo se le seguirá solicitando que realice el proceso de loguin si quiere acceder a nuevas funcionalidades.} \\
\hline
\bf Criterios de Aceptación &\multicolumn{3}{p{0.675\textwidth}|}
{El requerimiento se acepta cuando un usuario registrado ingresando los datos correctos de login y password pueda ingresar al sistema; y cuando se presente impedimento para ingresar al sistema cuando los datos sean incorrectos} \\
\hline
\end{longtable}
\end{center}
%\end{document} %comentar para inlcuir