%\documentclass[]{article}
%\usepackage[spanish]{babel}
%\usepackage[utf8]{inputenc}
%\usepackage{geometry}
%\usepackage{colortbl}
%\usepackage{longtable}
%\geometry{tmargin=3cm,bmargin=3cm,lmargin=3cm,rmargin=2cm}
%\begin{document}
%para incluir comentar hasta acá
\begin{center}
\begin{longtable}{|p{0.225\textwidth}|p{0.225\textwidth}|p{0.225\textwidth}|p{0.225\textwidth}|}
\hline
\multicolumn{2}{|p{0.45\textwidth}|}{{\bf {Función del requerimiento:}}
Permitir el registro de nuevos usuarios. } & {\bf{ Estado}} & Análisis \\
\hline
\bf {Creado por} & Maria Andrea Cruz & \bf {Actualizado por} & Felipe Vargas \\
\hline
\bf {Fecha Creación } & Marzo 31 2011 & \bf {Fecha de Actualización }& 
Abril 02 2011\\
Mayo 09 2011\\
\hline
\multicolumn{2}{|p{0.45\textwidth}}{\bf Identificador} &
\multicolumn{2}{|p{0.45\textwidth}|}{R01} \\
\hline
\multicolumn{2}{|p{0.45\textwidth}}{\bf {Tipo de requerimiento}} &
\multicolumn{2}{|p{0.45\textwidth}|}{Funcional}\\
\hline
\bf Descripción &\multicolumn{3}{p{0.675\textwidth}|}
{El sistema después de que el usuario proporcione los datos correspondiente (login, contraseña, pregunta secreta, respuesta secreta, nombres, apellidos, género, fecha de nacimiento, correo electrónico, nivel de escolaridad, vínculo con Univalle, áreas de interés.) los almacenará en la base de datos. El usuario pasará a ser un usuario registrado, con acceso al sistema mediante logueo.} \\
\hline
\bf Datos de entrada &\multicolumn{3}{p{0.675\textwidth}|}{
Para que un usuario se registre debe de proporcionar obligatoriamente un login, una contraseña, una pregunta secreta, una respuesta secreta a la pregunta, su primer nombre, su primer apellido, su género y correo electrónico. Opcionalmente puede proporcionar el segundo nombre, el segundo apellido, la fecha de nacimiento, el nivel de escolaridad, el vínculo con univalle y sus área de conocimiento de interés.}\\
\hline
\bf Datos de salida &\multicolumn{3}{p{0.675\textwidth}|}
{Un mensaje que informa al usuario que el registro se ha dado de manera exitosa y que puede loguearse de ahora en adelante formando parte de los usuarios registrados.} \\
\hline
\bf Resultados esperados &\multicolumn{3}{p{0.675\textwidth}|}
{El sistema deberá realizar una modificacion en la base de datos de usuarios lo que permite que en un futuro sea reconocido.} \\
\hline
\bf Origen &\multicolumn{3}{p{0.675\textwidth}|}
{Documento de descripción del problema.} \\
\hline
\bf Dirigido a &\multicolumn{3}{p{0.675\textwidth}|}
{Estudiantes , profesores , funcionarios de Univalle y administrador} \\
\hline
\bf Prioridad &\multicolumn{3}{p{0.675\textwidth}|}{5} \\
\hline
\bf Requerimientos Asociados &\multicolumn{3}{p{0.675\textwidth}|}
{} \\
\hline
\multicolumn{4}{|>{\columncolor[rgb]{0.8,0.8,0.8}}c|}{\bf Especificación}\\
\hline
\bf Precondiciones &\multicolumn{3}{p{0.675\textwidth}|}
{El sistema debe estar conectado a la base de datos, la interfaz debe corresponder a la de usuario no registrado, es decir, la inicial.} \\
\hline
\hline
\bf Poscondicion &\multicolumn{3}{p{0.675\textwidth}|}
{El sistema tiene un nuevo usuario que se ve especificado como un nuevo registro en la base de datos en lo que corresponde a usuarios. } \\
\hline
\bf Criterios de Aceptación &\multicolumn{3}{p{0.675\textwidth}|}
{El requerimiento es aceptado si es posible registrarse como usuario en el sistema Biblioteca Digital.} \\
\hline
\end{longtable}
\end{center}
%\end{document} %comentar para inlcuir