%\documentclass[]{article}
%\usepackage[spanish]{babel}
%\usepackage[utf8]{inputenc}
%\usepackage{geometry}
%\usepackage{colortbl}
%\usepackage{longtable}
%\geometry{tmargin=3cm,bmargin=3cm,lmargin=3cm,rmargin=2cm}
%\begin{document}
%para incluir comentar hasta acá
\begin{center}

\begin{longtable}{|p{0.225\textwidth}|p{0.225\textwidth}|p{0.225\textwidth}|p{0.225\textwidth}|}
\hline
\multicolumn{2}{|p{0.45\textwidth}|}{{\bf {Función del requerimiento:}}
Eliminar usuarios. } & {\bf{ Estado}} & Análisis \\
\hline
\multicolumn{2}{|p{0.45\textwidth}}{\bf Identificador} &
\multicolumn{2}{|p{0.45\textwidth}|}{R04} \\
\hline
\multicolumn{2}{|p{0.45\textwidth}}{\bf {Tipo de requerimiento}} &
\multicolumn{2}{|p{0.45\textwidth}|}{Funcional}\\
\hline
\bf {Creado por} & Maria Andrea Cruz & \bf {Fecha } & Marzo 31 2011 \\
\hline
\bf {Actualizado por} & Cristian Ríos  & \bf {Fecha }& Abril 28 2011\\
\hline
\bf {Actualizado por} & Cristian Ríos  & \bf {Fecha }& Mayo 09 2011\\
\hline
\bf Descripción &\multicolumn{3}{p{0.675\textwidth}|}
{ El sistema debe proporcionar a los usuarios administradores una
opción que permita eliminar lógicamente usuarios registrados, esto es, se
le cambiará su estado a 'deshabilitado' lo que impedirá que ingrese al
sistema, restringiendo así sus funcionalidades dentro del mismo. Los
usuarios que se desea eliminar se seleccionan a partir de una lista
creada por una consulta a la base de datos que muestre el o los posibles
usuarios a eliminar.} \\
\hline
\bf Datos de entrada &\multicolumn{3}{p{0.675\textwidth}|}{
Para poder cambiar de estado a algún usuario se presentan al administrador todos los datos del usuario a modificar de manera informativa para verificar la identidad del usuario. El administrador debe de proporcionar el dato estado, con el que se le realizará la actualización al usuario.}\\
\hline
\bf Datos de salida &\multicolumn{3}{p{0.675\textwidth}|}
{ El sistema genera y muestra un mensaje de confirmación preguntando al administrador si está totalmente seguro de realizar dicha operación, de ser así, genera y muestra un mensaje informando del éxito o no de la operación solicitada.} \\
\hline
\bf Resultados esperados &\multicolumn{3}{p{0.675\textwidth}|}
{ El sistema deberá realizar una actualización del estado del usuario o los usuarios eliminados en la base de datos.} \\
\hline
\bf Origen &\multicolumn{3}{p{0.675\textwidth}|}
{Documento de descripción del problema.} \\
\hline
\bf Dirigido a &\multicolumn{3}{p{0.675\textwidth}|}
{Administrador.} \\
\hline
\bf Prioridad &\multicolumn{3}{p{0.675\textwidth}|}{3} \\
\hline
\bf Requerimientos Asociados &\multicolumn{3}{p{0.675\textwidth}|}
{\begin{itemize}
        \item R06
\end{itemize}} \\
\hline
\multicolumn{4}{|>{\columncolor[rgb]{0.8,0.8,0.8}}c|}{\bf Especificación}\\
\hline
\bf Precondiciones &\multicolumn{3}{p{0.675\textwidth}|}
{El sistema debe estar conectado a la base de datos , un administrador debe estar logueado para poder tener acceso a la interfaz de eliminación de usuarios proporcionada por el sistema.} \\
\hline
\bf Poscondicion &\multicolumn{3}{p{0.675\textwidth}|}
{El sistema en caso de tener éxito en la acción eliminar, actualiza en su base de datos los registros de los usuarios indicados para ser eliminados , por lo cual, el usuario ya no podrá ingresar al sistema.} \\
\hline
\bf Criterios de Aceptación &\multicolumn{3}{p{0.675\textwidth}|}
{El requerimiento es aceptado si un usuario administrador puede buscar posibles usuarios a eliminar y de estos seleccionar los que desee para posteriormente eliminarlos, además, el usuario eliminado le será imposible ingresar al sistema.} \\
\hline
\end{longtable}
\end{center}
%\end{document} %comentar para inlcuir