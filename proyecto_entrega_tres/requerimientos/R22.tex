%\documentclass[]{article}
%\usepackage[spanish]{babel}
%\usepackage[utf8]{inputenc}
%\usepackage{geometry}
%\usepackage{colortbl}
%\usepackage{longtable}
%\geometry{tmargin=3cm,bmargin=3cm,lmargin=3cm,rmargin=2cm}
%\begin{document}
%para incluir comentar hasta acá
\begin{center}
\begin{longtable}{|p{0.225\textwidth}|p{0.225\textwidth}|p{0.225\textwidth}|p{0.225\textwidth}|}
\hline
\multicolumn{2}{|p{0.45\textwidth}|}{{\bf {Función del requerimiento:}}
Permitir la consulta avanzada de documentos. } & {\bf{ Estado}} & Análisis \\
\hline
\multicolumn{2}{|p{0.45\textwidth}}{\bf Identificador} &
\multicolumn{2}{|p{0.45\textwidth}|}{R22} \\
\hline
\multicolumn{2}{|p{0.45\textwidth}}{\bf {Tipo de requerimiento}} &
\multicolumn{2}{|p{0.45\textwidth}|}{Funcional}\\
\hline
\bf {Creado por} & Maria Andrea Cruz & \bf {Fecha  } & Marzo 31 2011\\
\hline
\bf {Actualizado por} & Maria Andrea Cruz  & \bf {Fecha  }& Abril 28 2011\\
\hline
\bf {Actualizado por} & Cristian Ríos  & \bf {Fecha  }& Mayo 10 2011\\

\hline
\bf Descripción &\multicolumn{3}{p{0.675\textwidth}|}
{El sistema debe permitir la consulta avanzada, donde el usuario registrado o no registrado ingresa alguno de los siguientes datos: Titulo, autor, palabra clave, tipo de documento y/o área, y como datos adicionales idioma del documento, formato del archivo y fecha de publicación.} \\
\hline
\bf Datos de entrada &\multicolumn{3}{p{0.675\textwidth}|}{
El usuario que desee realizar la búsqueda deberá proporcioanr alguno o todos de los siguientes datos: título del documento, autor, palabra clave, área a la que pertence el documento, tipo de documento, idioma del documento, formato del archivo del documento y/o fecha de publicación. El dato que no sea proprcionado no será tenido en cuenta para realizar la búsuqeda.}\\
\hline
\bf Datos de salida &\multicolumn{3}{p{0.675\textwidth}|}
{El sistema debe desplegar el resultado de la búsqueda en la pantalla.} \\
\hline
\bf Resultados esperados &\multicolumn{3}{p{0.675\textwidth}|}
{El sistema debe realizar una consulta a la base de datos, si la consulta no es vacía despleglar el resultado como una lista en la pantalla, de lo contrario mostrar una notificación de documentos no encontrado.} \\
\hline
\bf Origen &\multicolumn{3}{p{0.675\textwidth}|}
{Documento de descripción del problema.} \\
\hline
\bf Dirigido a &\multicolumn{3}{p{0.675\textwidth}|}
{Usuarios registrados(normal, administrador y catalogador) y no registrados.} \\
\hline
\bf Prioridad &\multicolumn{3}{p{0.675\textwidth}|}{5} \\
\hline
\bf Requerimientos Asociados &\multicolumn{3}{p{0.675\textwidth}|}
{\begin{itemize}
         \item R24
\end{itemize}} \\
\hline
\multicolumn{4}{|>{\columncolor[rgb]{0.8,0.8,0.8}}c|}{\bf Especificación}\\
\hline
\bf Precondiciones &\multicolumn{3}{p{0.675\textwidth}|}
{Haber iniciado el sistema.} \\
\hline
\hline
\bf Poscondicion &\multicolumn{3}{p{0.675\textwidth}|}
{Una lista de documentos en la pantalla o una notificación de no encontrado el documento. } \\
\hline
\bf Criterios de Aceptación &\multicolumn{3}{p{0.675\textwidth}|}
{El requerimiento es aceptado si cualquier usuario logra realizar consultas de documentos en la biblioteca digital especificando parámetros bien definidos.} \\
\hline
\end{longtable}
\end{center}
%\end{document} %comentar para inlcuir