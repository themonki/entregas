%\documentclass[]{article}
%\usepackage[spanish]{babel}
%\usepackage[utf8]{inputenc}
%\usepackage{geometry}
%\usepackage{colortbl}
%\usepackage{longtable}
%\geometry{tmargin=3cm,bmargin=3cm,lmargin=3cm,rmargin=2cm}
%\begin{document}
%para incluir comentar hasta acá
\begin{center}
\begin{longtable}{|p{0.225\textwidth}|p{0.225\textwidth}|p{0.225\textwidth}|p{0.225\textwidth}|}
\hline
\multicolumn{2}{|p{0.45\textwidth}|}{{\bf {Función del requerimiento:}}
Generar reportes en formato PDF del total de usuarios registrados. } & {\bf{ Estado}} & Análisis \\
\hline
\bf {Creado por} & Maria Andrea Cruz & \bf {Actualizado por} & María Andrea Cruz\\
\hline
\bf {Fecha Creación } & Marzo 31 2011 & \bf {Fecha de Actualización }& 
Abril 28 2011\\
Mayo 10 2011\\
\hline
\multicolumn{2}{|p{0.45\textwidth}}{\bf Identificador} &
\multicolumn{2}{|p{0.45\textwidth}|}{R30} \\
\hline
\multicolumn{2}{|p{0.45\textwidth}}{\bf {Tipo de requerimiento}} &
\multicolumn{2}{|p{0.45\textwidth}|}{Funcional}\\
\hline
\bf Descripción &\multicolumn{3}{p{0.675\textwidth}|}
{El sistema debe proveer una interfaz para que el usuario con perfil administrador pueda generar reportes en formato PDF de todos los usuarios registrados en el sistema.} \\
\hline
\bf Datos de entrada &\multicolumn{3}{p{0.675\textwidth}|}{
El usuario administrador que esté solicitando el reporte debe de proporcionar el nombre del reporte al momento en que el sistema pide los parámetros para generar el reporte, en este caso el usuario deberá elegir 'total de usuarios registrados'. Además, una vez se haya generado el reporte debera de proporcionar la dirección donde se guardará el reporte.}\\
\hline
\bf Datos de salida &\multicolumn{3}{p{0.675\textwidth}|}
{El sistema generará y mostrará un cuadro de dialogo notificando la generación y entrega del reporte.} \\
\hline
\bf Resultados esperados &\multicolumn{3}{p{0.675\textwidth}|}
{El sistema deberá consulta a la base de datos todos los usuarios resgistrados, y organizando los resultados por nombre en una plantilla generar el reporte en formato PDF.} \\
\hline
\bf Origen &\multicolumn{3}{p{0.675\textwidth}|}
{Documento de descripción del problema.} \\
\hline
\bf Dirigido a &\multicolumn{3}{p{0.675\textwidth}|}
{Administrador} \\
\hline
\bf Prioridad &\multicolumn{3}{p{0.675\textwidth}|}{3} \\
\hline
\bf Requerimientos Asociados &\multicolumn{3}{p{0.675\textwidth}|}
{\begin{itemize}
\item R01
\item R02
\end{itemize}} \\
\hline
\multicolumn{4}{|>{\columncolor[rgb]{0.8,0.8,0.8}}c|}{\bf Especificación}\\
\hline
\bf Precondiciones &\multicolumn{3}{p{0.675\textwidth}|}
{Haberse logueado como administrador.} \\
\hline
\bf Poscondicion &\multicolumn{3}{p{0.675\textwidth}|}
{Un archivo PDF que corresponde al informe elaborado por el sistema, onde organizado en tablas se muestran la informacion de todos los usuarios registrados, esto es: login, nombre1, apellido1, email, vinculo con Univalle, género, fecha de registro, fecha de nacimiento, fecha de registro, tipo de usuario y estado.} \\
\hline
\bf Criterios de Aceptación &\multicolumn{3}{p{0.675\textwidth}|}
{El requerimiento es aceptado si se genera el reporte en formato PDF y  con la información pedida.} \\
\hline
\end{longtable}
\end{center}
%\end{document} %comentar para inlcuir