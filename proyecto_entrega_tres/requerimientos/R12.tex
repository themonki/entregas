%\documentclass[]{article}
%\usepackage[spanish]{babel}
%\usepackage[utf8]{inputenc}
%\usepackage{geometry}
%\usepackage{colortbl}
%\usepackage{longtable}
%\geometry{tmargin=3cm,bmargin=3cm,lmargin=3cm,rmargin=2cm}
%\begin{document} %comentar hasta esta linea para incluir
\begin{center}
\begin{longtable}{|p{0.225\textwidth}|p{0.225\textwidth}|p{0.225\textwidth}|p{0.225\textwidth}|}
\hline
\multicolumn{2}{|p{0.45\textwidth}|}{{\bf {Función del requerimiento:}}
Reactivar usuario Eliminado. } & {\bf{ Estado}} & Análisis \\
\hline
\bf {Creado por} & Cristian Ríos& \bf {Actualizado por} & Cristian Ríos\\
\hline
\bf {Fecha Creación } & Abril 28 2011 & \bf {Fecha de Actualización }& 
Abril 30 2011\\
Mayo 10 2011\\
\hline
\multicolumn{2}{|p{0.45\textwidth}}{\bf Identificador} &
\multicolumn{2}{|p{0.45\textwidth}|}{R12} \\
\hline
\multicolumn{2}{|p{0.45\textwidth}}{\bf {Tipo de requerimiento}} &
\multicolumn{2}{|p{0.45\textwidth}|}{Funcional}\\
\hline
\bf Descripción &\multicolumn{3}{p{0.675\textwidth}|}
{El sistema debe permitir a los usuarios administradores una opción que permita reactivar a un usuario que a sido eliminado con anterioridad, esto es, cambiar el estado del usuario a ‘habilitado’ lo que le permitirá ingresar de nuevo al sistema y obtener nuevamente sus funcionalidades dentro del mismo. Los usuarios que se desea reactivar se seleccionan a partir de una lista creada por una consulta a la base de datos que muestre el o los posibles
usuarios a inactivos, candidatos a ser reactivados.} \\
\hline
\bf Datos de entrada &\multicolumn{3}{p{0.675\textwidth}|}{
Para poder cambiar de estado a un usuario se presentan al administrador todos los datos del usuario a modificar de manera informativa para verificar la identidad del usuario. El administrador debe de proporcionar el dato estado con el que se le realizará la actualización al usuario.}\\
\hline
\bf Datos de salida &\multicolumn{3}{p{0.675\textwidth}|}
{El sistema genera y muestra un mensaje de confirmación preguntando al administrador si está totalmente seguro de realizar dicha operación, de ser así, genera y muestra un mensaje informando del éxito o no de la operación solicitada.} \\
\hline
\bf Resultados esperados &\multicolumn{3}{p{0.675\textwidth}|}
{El sistema deberá realizar una actualización del estado del usuario o los usuarios reactivados en la base de datos.} \\
\hline
\bf Origen &\multicolumn{3}{p{0.675\textwidth}|}
{Documento de descripción del problema.} \\
\hline
\bf Dirigido a &\multicolumn{3}{p{0.675\textwidth}|}
{Administrador} \\
\hline
\bf Prioridad &\multicolumn{3}{p{0.675\textwidth}|}{3} \\
\hline
\bf Requerimientos Asociados &\multicolumn{3}{p{0.675\textwidth}|}
{\begin{itemize}
        \item R06
        \item R01
        \item R04
\end{itemize}} \\
\hline
\multicolumn{4}{|>{\columncolor[rgb]{0.8,0.8,0.8}}c|}{\bf Especificación}\\
\hline
\bf Precondiciones &\multicolumn{3}{p{0.675\textwidth}|}
{El sistema debe estar conectado a la base de datos , un administrador debe estar logueado para poder tener acceso a la interfaz de reactivación de usuarios proporcionada por el sistema.} \\
\hline
\bf Poscondicion &\multicolumn{3}{p{0.675\textwidth}|}
{El sistema en caso de tener éxito en la acción ractivar, actualiza en su base de datos los registros de los usuarios indicados para serreactivados , por lo cual, el usuario podrá ingresar nuevamente al sistema. } \\
\hline
\bf Criterios de Aceptación &\multicolumn{3}{p{0.675\textwidth}|}
{El requerimiento es aceptado si un usuario administrador puede buscar posibles usuarios a reactivar y de estos seleccionar los que desee para posteriormente reactivarlos en el sistema, además, el usuario reactivado ahora le será posible ingresar nuevamente al sistema. } \\
\hline
\end{longtable}
\end{center}
%\end{document}