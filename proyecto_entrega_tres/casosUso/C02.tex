%\documentclass[]{article}
%\usepackage[spanish]{babel}
%\usepackage[utf8]{inputenc}
%\usepackage{geometry}
%\usepackage{colortbl}
%\usepackage{longtable}
%\usepackage{graphicx}
%\geometry{tmargin=3cm,bmargin=3cm,lmargin=3cm,rmargin=2cm}
%
%\begin{document}
\begin{center}
\begin{longtable}{|p{0.225\textwidth}|p{0.225\textwidth}|p{0.225\textwidth}|p{0.225\textwidth}|}
\hline
{\bf {Empresa:}} &
\multicolumn{2}{p{0.45\textwidth}|} { Escuela de Ingeniería de Sistemas y Computación } &
{\includegraphics[width=80.5pt]{LOGO}} \\
\hline
\bf {Nombre del caso de uso:}&\multicolumn{3}{l|}{
Modificar usuarios.
} \\
\hline
\bf Codigo: & 
CU02 &\bf Fecha: & 
Abril 2 2011 \\
\hline
\bf Autor(es ): & 
Yerminson Gonzalez & 
Edgar Andrés Moncada & 
 \\
\hline
\bf Descripcion: &\multicolumn{3}{p{0.675\textwidth}|}
{
Permite la modificación de usuarios , es decir alguna de la información almacenada en el sistema.
} \\
\hline
\bf Actores: &\multicolumn{3}{p{0.675\textwidth}|}{
Administrador. 
} \\
\hline
\bf Precondiciones: &\multicolumn{3}{p{0.675\textwidth}|}
{
El administrador debe haber consultado sobre algún usuario.
} \\
\hline
\multicolumn{4}{|c|}{\bf {Flujo Normal}}\\
\hline
\multicolumn{2}{|c}{\bf Actor} & \multicolumn{2}{|c|}{\bf Sistema } \\
\hline
\multicolumn{2}{|p{0.45\textwidth}}
{
\begin{itemize}
\item[1. ]El caso de uso inicia cuando el administrador selecciona al usuario de la lista resultado.
\item[3.] El administrador modifica los datos de acuerdo a sus necesidades o ha una actualización de datos.
\end{itemize}
} &
\multicolumn{2}{|p{0.45\textwidth}|}
{
\begin{itemize}
\item[2.] El sistema responde a través de una interfaz que muestra los campos referente a sus atributos, donde se pueden modificar el estado o el perfil de algún usuario seleccionado el dato perfil de usuario puede ser: Usuario normal, Catalogador o administrador, el dato estado puede ser: Habilitado o deshabilitado.
\item[4. ]El sistema valida que todos los datos obligatorios se hayan suministrado.
\item[5. ]El sistema accede a la base de datos y realiza una verificación de los datos nuevos y existentes para que no hayan datos iguales donde no pueden serlo.
\item[6.] El sistema realiza la actualización de los datos en la base de datos con los datos suministrados según la preferencia del administrador.
\item[7.] El sistema responde a través de una interfaz de éxito donde muestra que se ha completado la operación y el caso de uso termina.
\end{itemize}
} \\
\hline
\multicolumn{4}{|c|}{\bf {Flujo Alternativo}}\\
\hline
\multicolumn{2}{|p{0.45\textwidth}}
{} &
\multicolumn{2}{|p{0.45\textwidth}|}
{
\begin{itemize}
\item[4.1.] El Sistema valida los datos y encuentra que hay errores.
\item[4.2.] El Sistema responde a través de una interfaz que muestra los campos referentes a los atributos para que sean llenados nuevamente, con información del error cometido.
\end{itemize}
} \\
\hline
\bf Poscondiciones: &\multicolumn{3}{p{0.675\textwidth}|}
{
Atributos de usuario en la base de datos modificados.
} \\
\hline
\bf Excepciones: &\multicolumn{3}{p{0.675\textwidth}|}
{
La base de datos puede desconectarse debido a fallos de energía.
} \\
\hline
\bf Aprobado por : & 
 & \bf Fecha & 
 \\
\hline
\end{longtable}
\end{center}
%\end{document}