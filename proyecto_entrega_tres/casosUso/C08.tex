%\documentclass[]{article}
%\usepackage[spanish]{babel}
%\usepackage[utf8]{inputenc}
%\usepackage{geometry}
%\usepackage{colortbl}
%\usepackage{longtable}
%\usepackage{graphicx}
%\geometry{tmargin=3cm,bmargin=3cm,lmargin=3cm,rmargin=2cm}
%
%\begin{document}
\begin{center}
\begin{longtable}{|p{0.225\textwidth}|p{0.225\textwidth}|p{0.225\textwidth}|p{0.225\textwidth}|}
\hline
{\bf {Empresa:}} &
\multicolumn{2}{p{0.45\textwidth}|} { Escuela de Ingeniería de Sistemas y Computación } &
{\includegraphics[width=80.5pt]{LOGO}} \\
\hline
\bf {Nombre del caso de uso:}&\multicolumn{3}{l|}{
Ingresar Área.
} \\
\hline
\bf Codigo: & 
CU08 &\bf Fecha: & 
Abril 02 2011 \\
\hline
\bf Autor(es ): & 
Yerminson Gonzalez & 
Cristian Ríos & 
 \\
\hline
\bf Descripcion: &\multicolumn{3}{p{0.675\textwidth}|}
{
Permite la creación de nuevas áreas de ciencias de la computación en el sistema.
} \\
\hline
\bf Actores: &\multicolumn{3}{p{0.675\textwidth}|}{
Administrador, Catalogador. 
} \\
\hline
\bf Precondiciones: &\multicolumn{3}{p{0.675\textwidth}|}
{
Tener el perfil de usuario Administrador o catalogador.
} \\
\hline
\multicolumn{4}{|c|}{\bf {Flujo Normal}}\\
\hline
\multicolumn{2}{|c}{\bf Actor} & \multicolumn{2}{|c|}{\bf Sistema } \\
\hline
\multicolumn{2}{|p{0.45\textwidth}}
{
\begin{itemize}
\item[1. ]El caso de uso inicia cuando el Usuario solicita crear una nueva área de ciencias de la computación.
\item[3.] El Usuario ingresa datos en los campos proporcionado por la interfaz del sistema para creación de nuevas áreas.
\item[4. ]El Usuario indica al Sistema que ya a ingresado los datos y que desea crear la nueva área.
\item[8.] El Usuario acepta el mensaje de confirmación generado por el Sistema y el caso de uso finaliza.
\end{itemize}
} &
\multicolumn{2}{|p{0.45\textwidth}|}
{
\begin{itemize}
\item[2.] El Sistema responde mostrando una interfaz con tres campos: un campo para indicar el nombre, un campo para indicar una descripción de la nueva área y un tercer campo donde se especifica la área a la que pertenece si la área que se esta ingresando es una subárea.
\item[5.]El Sistema valida que el nombre de la área que a ingresado el Usuario para la nueva área no exista como nombre de otra área.
\item[6. ]6. El Sistema valida que si lo que se esta creando es una subárea, el nombre que se haya indicado como área exista previamente en el sistema.
\item[7.] El Sistema crea una nueva área de ciencias de la computación en el sistema y responde con un mensaje al Usuario indicando el éxito de la operación. 
\end{itemize}
} \\
\hline
\multicolumn{4}{|c|}{\bf {Flujo Alternativo}}\\
\hline
\multicolumn{2}{|p{0.45\textwidth}}
{
\begin{itemize}
\item[7.1] El Usuario acepta el mensaje de notificación del error generado por el Sistema.
\end{itemize}
} &
\multicolumn{2}{|p{0.45\textwidth}|}
{
\begin{itemize}
\item[5.1.] El Sistema al realizar la validación del nombre y se percata de que el nombre dado a la nueva área ya existe.
\item[6.1.] El Sistema genera un mensaje indicando que el nombre dado al área no se puede utilizar porque ya existe un área con ese nombre.
\item[8.1.] El Sistema regresa a la interfaz que permite crear una nueva área de ciencias de la computación para que el Usuario indique otro nombre para continuar con la operación normalmente.
\item[6.1.] El Sistema al realizar la validación del nombre de la área cuando se esta creando una sebárea se percata de que el nombre dado no existe en el sistema como un área de ciencias de la computación.
\item[6.2.  ]El Sistema genera un mensaje indicando que el nombre dado al área a la que pertenece la subárea no existe.
\item[8.2.] El Sistema regresa a la interfaz que permite crear una nueva área de ciencias de la computación para que el usuario indique una nombre de área válido a la cual pertenece la subárea que se esta creando.
\end{itemize}
} \\
\hline
\bf Poscondiciones: &\multicolumn{3}{p{0.675\textwidth}|}
{
El Sistema crea una nueva tupla de la base de datos correspondiente a la nueva área de ciencias de la computación.
} \\
\hline
\bf Excepciones: &\multicolumn{3}{p{0.675\textwidth}|}
{
El Sistema no puede acceder a la base de datos y no puede crear la nueva tupla o realizar las consultas para las validaciones de los datos.
} \\
\hline
\bf Aprobado por : & 
 & \bf Fecha & 
 \\
\hline
\end{longtable}
\end{center}
%\end{document}