\subsection{Planteamiento del problema y pertinencia del mismo}
        Cada día dentro de la escuela de ingeniería de sistemas se generan producciones académicas 			por parte de estudiantes y profesores, producciones que de alguna manera no se almacenan
        correctamente en algunos casos, sobre todo, las producciones digitales que a pesar de que
        fueron presentadas en ciertas circunstancias importantes no son del conocimiento de todos o
        no están al alcance de todos porque no se pensó que el tema podría llegar a ser de interés
        general. Por esta razón se cree que mantener la información en un solo lugar para que esta
        pueda ser usada en algún momento como herramienta en la formación académica de los 
        estudiantes y profesores se convirtió en un problema que buscamos afrontar y solucionar de
        la mejor manera, intentando beneficiar a todas las personas involucradas e interesadas en
        el.
        El problema nos muestra como se maneja ineficientemente los documentos en la escuela de
        ingeniería de sistemas y computación, es por esta razón que de alguna manera la solución de
        este nos permitirá tener un mejor acceso a la información según nuestro intereses y de esta
        forma ayudar a que el conocimiento dentro de la escuela de ingeniería de sistemas pueda ser
        accedido, compartido, analizado y estudiado por muchas mas personas permitiendo ampliar de
        manera cómoda las formas de pensar y el conocimiento sin sufrir por la búsqueda de 
        información.
        
        \subsection{Objetivos}
        	\subsection{Objetivo General}
        	\begin{itemize}
        		\item Desarrollar una aplicación para una biblioteca digital que permita reunir de
        		manera organizada un conjunto de documentos que han sido producidos en el área de
        		ciencias de la computación a lo largo del tiempo y que son de utilidad para la 
        		escuela de ingeniería de sistemas y computación - EISC de la Universidad del Valle.
        	\end{itemize}
        	
        	\subsection{Objetivos Específicos}
        	\begin{itemize}
        		\item Manejar la información generada de manera centralizada y ordenada para 
        		permitir un excelente control sobre ella y su rápida ubicación.
        		\item Permitir a los usuarios realizar consultas sobre documentos almacenados y si
        		se registran tener la posibilidad de descargarlo y usarlo como complemento en su
        		aprendizaje.
        		\item Generar estadísticas que muestren el trabajo realizado por el sistema para
        		realizar un seguimiento de este y observar el comportamiento de los usuarios en el
        		uso del mismo.
        		\item Mejorar la manera en que las personas obtiene información de buena calidad
        		relacionada con las ciencias de la computación que permita facilitar su aprendizaje
        		y comprensión de los temas de interés.
        	\end{itemize}      
        
        \subsection{Justificación}
        La escuela de ingeniería de sistemas y computación se ve con la necesidad de organizar los
        documentos que produce, con ciertas características facilitando la búsqueda, también
        necesita de un sistema que integre a los usuarios interesados para que además de poder
        consultar documentos puedan descargarlos, logrando una retroalimentación mucho mas marcada
        que con la organización actual de documentos donde no están disponibles para todos los
        interesados. por otro lado se debe seguir estadística sobre esta información para realizar
        análisis sobre los contenidos.
        Es por todo lo anterior que el planteamiento del sistema biblioteca digital, ofrece una
        solución que satisface todas esas necesidades, y que se presenta como un sistema práctico y
        accesible a todos los interesados por los documentos de las ciencias de la computación.
        
        \subsection{Área de aplicación}
        El área institucional sobre todo para instituciones educativas en este caso la universidad
        del valle la cual tiene personas dentro de su campus que están interesadas en temas 
        relacionados con ciencias de la computación y que serian beneficiados de gran manera por
        ese sistema.
        
        %*********************************************************************
        \subsection{Cronograma de actividades}
			%\documentclass[6pt]{article}
%\usepackage[utf8]{inputenc}
%\usepackage[spanish]{babel}
%\usepackage[]{graphicx}
%\usepackage{colortbl} %para las tablas
%\usepackage{longtable} %para las tablas
%\usepackage{geometry}
%\geometry{tmargin=3cm,bmargin=3cm,lmargin=3cm,rmargin=2cm}
%
%\begin{document}
\begin{center}
% \tiny
 \scriptsize
% \footnotesize
% \small
% \normalsize
% \large
% \Large
% \LARGE
\begin{longtable}{|p{0.08\textwidth}|p{0.13\textwidth}|p{0.08\textwidth}|p{0.08\textwidth}|p{0.08\textwidth}|p{0.13\textwidth}|p{0.13\textwidth}|p{0.08\textwidth}|}
\hline
{\begin{center}\bf Número Entregable\end{center}} & 
{\begin{center}\bf Producto\end{center}} & 
{\begin{center}\bf Fecha Inicio\end{center}} & 
{\begin{center}\bf Fecha Finalización\end{center}} & 
{\begin{center}\bf Fecha Revisión\end{center}} & 
{\begin{center}\bf Responsables\end{center}} & 
{\begin{center}\bf Observaciones\end{center}} & 
{\begin{center}\bf Fecha Entrega\end{center}}\\
\hline
		
{\begin{center} 1 \end{center}} & 
{\begin{center}Levantamiento de Requerimientos\end{center}} & 
{\begin{center}3/30/2011\end{center}} & 
{\begin{center}4/2/2011\end{center}} & 
{\begin{center}4/2/2011\end{center}} & 
{\begin{center}Cristian Leonardo Ríos, Yerminson Gonzalez, Luis Felipe Vargas, María Andrea Cruz\end{center}} & 
{} & 
{\begin{center}4/6/2011\end{center}}\\
\hline

{\begin{center}1\end{center}} & 
{\begin{center}Creación de diagramas de Casos de Uso\end{center}} & 
{\begin{center}4/2/2011\end{center}} & 
{\begin{center}4/4/2011\end{center}} & 
{\begin{center}4/4/2011\end{center}} & 
{\begin{center}Edgar Andrés Moncada, Luis Felipe Vargas\end{center}} & 
{} & 
{\begin{center}4/6/2011\end{center}}\\
\hline

{\begin{center}1\end{center}} & 
{\begin{center}Redacción de los Casos de Uso Extendido\end{center}} & 
{\begin{center} 4/3/2011 \end{center}} & 
{\begin{center} 4/5/2011 \end{center}} & 
{\begin{center} 4/5/2011 \end{center}} & 
{\begin{center} Cristian Leonardo Ríos, Yerminson Gonzalez, Edgar Andrés Moncada \end{center}} & 
{} & 
{\begin{center} 4/6/2011 \end{center}}\\
\hline
		
{\begin{center} 1 \end{center}} & 
{\begin{center} Redacción del plan de Proyecto \end{center}} & 
{\begin{center} 4/2/2011 \end{center}} & 
{\begin{center} 4/5/2011 \end{center}} & 
{\begin{center} 4/5/2011 \end{center}} & 
{\begin{center} Todos los desarrolladores \end{center}} & 
{\begin{center} Problemas con algunas secciones del Plan de Desarrollo \end{center}} & 
{\begin{center} 4/6/2011 \end{center}}\\
\hline

{\begin{center} 2 \end{center}} & 
{\begin{center} Corrección Redacción Plan de Proyecto \end{center}} & 
{\begin{center} 4/20/2011 \end{center}} & 
{\begin{center} 4/29/2011 \end{center}} & 
{\begin{center} 4/29/2011 \end{center}} & 
{\begin{center} Cristian Leonardo Ríos, Yerminson Gonzalez, María Andrea Cruz \end{center}} & 
{\begin{center}  \end{center}} & 
{\begin{center} 4/29/2011 \end{center}}\\
\hline

{\begin{center} 2 \end{center}} & 
{\begin{center} Corrección Casos de Uso Extendido \end{center}} & 
{\begin{center} 4/20/2011 \end{center}} & 
{\begin{center} 4/29/2011 \end{center}} & 
{\begin{center} 4/29/2011 \end{center}} & 
{\begin{center} Edgar Andrés Moncada, Yerminson Gonzalez \end{center}} & 
{\begin{center}  \end{center}} & 
{\begin{center} 4/29/11 \end{center}}\\
\hline

{\begin{center} 2 \end{center}} & 
{\begin{center} Diagramas de Clase \end{center}} & 
{\begin{center} 4/25/2011 \end{center}} & 
{\begin{center} 4/29/2011 \end{center}} & 
{\begin{center} 4/29/2011 \end{center}} & 
{\begin{center} Edgar Andrés Moncada, Yerminson Gonzalez, Luis Felipe Vargas, María Andrea Cruz \end{center}} & 
{\begin{center}  \end{center}} & 
{\begin{center} 4/29/2011 \end{center}}\\
\hline

{\begin{center} 2 \end{center}} & 
{\begin{center} Diagrama Modelo de datos en UML \end{center}} & 
{\begin{center} 4/25/2011 \end{center}} & 
{\begin{center} 4/29/2011 \end{center}} & 
{\begin{center} 4/29/2011 \end{center}} & 
{\begin{center} Yerminson Gonzalez \end{center}} & 
{\begin{center}  \end{center}} & 
{\begin{center} 4/29/2011 \end{center}}\\
\hline	

{\begin{center} 2 \end{center}} & 
{\begin{center} Requerimientos de Usuario norma IEEE830 \end{center}} & 
{\begin{center} 4/23/2011 \end{center}} & 
{\begin{center} 4/29/2011 \end{center}} & 
{\begin{center} 4/29/2011 \end{center}} & 
{\begin{center} María Andrea Cruz \end{center}} & 
{\begin{center}  \end{center}} & 
{\begin{center} 4/29/2011 \end{center}}\\
\hline	

%------------------------------------------------------------------

{\begin{center} 3 \end{center}} & 
{\begin{center} Corrección redacción Plan de Proyecto \end{center}} & 
{\begin{center} 5/14/2011 \end{center}} & 
{\begin{center} 5/14/2011 \end{center}} & 
{\begin{center} 5/14/2011 \end{center}} & 
{\begin{center} Cristian Leonardo Ríos \end{center}} & 
{\begin{center}  \end{center}} & 
{\begin{center} 5/18/2011 \end{center}}\\
\hline		

{\begin{center} 3 \end{center}} & 
{\begin{center} Corrección Diagramas de Clase \end{center}} & 
{\begin{center} 5/12/2011 \end{center}} & 
{\begin{center} 5/13/2011 \end{center}} & 
{\begin{center} 5/14/2011 \end{center}} & 
{\begin{center} María Andrea Cruz, Edgar Andrés Moncada, Yerminson Gonzalez \end{center}} & 
{\begin{center}  \end{center}} & 
{\begin{center} 5/18/2011 \end{center}}\\
\hline

{\begin{center} 3 \end{center}} & 
{\begin{center} Corrección Modelo de Datos \end{center}} & 
{\begin{center} 5/14/2011 \end{center}} & 
{\begin{center} 5/15/2011 \end{center}} & 
{\begin{center} 5/15/2011 \end{center}} & 
{\begin{center} Yerminson Gonzalez \end{center}} & 
{\begin{center}  \end{center}} & 
{\begin{center} 5/18/2011 \end{center}}\\
\hline

{\begin{center} 3 \end{center}} & 
{\begin{center} Creación Diagramas de Paquetes \end{center}} & 
{\begin{center} 5/12/2011 \end{center}} & 
{\begin{center} 5/13/2011 \end{center}} & 
{\begin{center} 5/13/2011 \end{center}} & 
{\begin{center} Edgar Andrés Moncada \end{center}} & 
{\begin{center}  \end{center}} & 
{\begin{center} 5/18/2011 \end{center}}\\
\hline

{\begin{center} 3 \end{center}} & 
{\begin{center} Creación Diagramas de Secuencia\end{center}} & 
{\begin{center} 5/11/2011 \end{center}} & 
{\begin{center} 5/16/2011 \end{center}} & 
{\begin{center} 5/17/2011 \end{center}} & 
{\begin{center} María Andréa Cruz, Luis Felipe Vargas, Cristian Leonardo Ríos \end{center}} & 
{\begin{center}  \end{center}} & 
{\begin{center} 5/18/2011 \end{center}}\\
\hline

{\begin{center} 3 \end{center}} & 
{\begin{center} Prototipos de Interfaz de Usuario \end{center}} & 
{\begin{center} 5/13/2011 \end{center}} & 
{\begin{center} 5/14/2011 \end{center}} & 
{\begin{center} 5/14/2011 \end{center}} & 
{\begin{center} Luis Felipe Vargas \end{center}} & 
{\begin{center}  \end{center}} & 
{\begin{center} 5/18/2011 \end{center}}\\
\hline

{\begin{center} 3 \end{center}} & 
{\begin{center} Diagrama de despliegue \end{center}} & 
{\begin{center} 5/15/2011 \end{center}} & 
{\begin{center} 5/15/2011 \end{center}} & 
{\begin{center} 5/15/2011 \end{center}} & 
{\begin{center} Cristian Leonardo Ríos \end{center}} & 
{\begin{center}  \end{center}} & 
{\begin{center} 5/18/2011 \end{center}}\\
\hline

{\begin{center} 3 \end{center}} & 
{\begin{center} Modelo de Implementación Inicial \end{center}} & 
{\begin{center} 5/2/2011 \end{center}} & 
{\begin{center} 5/16/2011 \end{center}} & 
{\begin{center} 5/16/2011 \end{center}} & 
{\begin{center} Todos los desarrolladores \end{center}} & 
{\begin{center}  \end{center}} & 
{\begin{center} 5/18/2011 \end{center}}\\
\hline

%------------------------------------------------------------------

{\begin{center} 4 \end{center}} & 
{\begin{center} Corrección redacción Plan de Proyecto \end{center}} & 
{\begin{center} 5/23/2011 \end{center}} & 
{\begin{center} 6/2/2011 \end{center}} & 
{\begin{center} 6/2/2011 \end{center}} & 
{\begin{center} Cristian Leonardo Ríos, Edgar Andrés Moncada \end{center}} & 
{\begin{center}  \end{center}} & 
{\begin{center} 6/8/2011 \end{center}}\\
\hline

{\begin{center} 4 \end{center}} & 
{\begin{center} Corrección Diagramas de Clase \end{center}} & 
{\begin{center} 5/23/2011 \end{center}} & 
{\begin{center} 6/7/2011 \end{center}} & 
{\begin{center} 6/7/2011 \end{center}} & 
{\begin{center} María Andrea Cruz, Yerminson Gonzalez \end{center}} & 
{\begin{center}  \end{center}} & 
{\begin{center} 6/8/2011 \end{center}}\\
\hline

{\begin{center} 4 \end{center}} & 
{\begin{center} Corrección Modelo de Datos \end{center}} & 
{\begin{center} 5/27/2011 \end{center}} & 
{\begin{center} 5/29/2011 \end{center}} & 
{\begin{center} 5/29/2011 \end{center}} & 
{\begin{center} Yerminson Gonzalez \end{center}} & 
{\begin{center}  \end{center}} & 
{\begin{center} 6/8/2011 \end{center}}\\
\hline
		
{\begin{center} 4 \end{center}} & 
{\begin{center} Corrección Diagramas de Paquetes \end{center}} & 
{\begin{center} 6/1/2011 \end{center}} & 
{\begin{center} 6/7/2011 \end{center}} & 
{\begin{center} 6/7/2011 \end{center}} & 
{\begin{center} Edgar Andrés Moncada \end{center}} & 
{\begin{center}  \end{center}} & 
{\begin{center} 6/8/2011 \end{center}}\\
\hline
	
{\begin{center} 4 \end{center}} & 
{\begin{center} Corrección Diagramas de Secuencia \end{center}} & 
{\begin{center} 5/23/2011 \end{center}} & 
{\begin{center} 6/7/2011 \end{center}} & 
{\begin{center} 6/7/2011 \end{center}} & 
{\begin{center}  María Andrea Cruz \end{center}} & 
{\begin{center}  \end{center}} & 
{\begin{center} 6/8/2011 \end{center}}\\
\hline

{\begin{center} 4  \end{center}} & 
{\begin{center} Actualización Prototipos de Interfaz de Usuario \end{center}} & 
{\begin{center} 6/7/2011 \end{center}} & 
{\begin{center} 6/9/2011  \end{center}} & 
{\begin{center} 6/9/2011 \end{center}} & 
{\begin{center} Edgar Andrés Moncada \end{center}} & 
{\begin{center}  \end{center}} & 
{\begin{center} 6/8/2011  \end{center}}\\
\hline

{\begin{center} 4 \end{center}} & 
{\begin{center} Corrección Diagrama de Despliegue \end{center}} & 
{\begin{center} 5/27/2011 \end{center}} & 
{\begin{center} 6/7/2011 \end{center}} & 
{\begin{center} 6/7/2011 \end{center}} & 
{\begin{center} Crstian Leonardo Ríos \end{center}} & 
{\begin{center}  \end{center}} & 
{\begin{center} 6/8/2011 \end{center}}\\
\hline

{\begin{center} 4 \end{center}} & 
{\begin{center} Material de Apoyo al Usuario Final \end{center}} & 
{\begin{center} 5/23/2011 \end{center}} & 
{\begin{center} 6/7/2011 \end{center}} & 
{\begin{center} 6/7/2011 \end{center}} & 
{\begin{center} María Andrea Cruz, Luis Felipe Vargas \end{center}} & 
{\begin{center}  \end{center}} & 
{\begin{center} 6/8/2011 \end{center}}\\
\hline
	
{\begin{center} 4 \end{center}} & 
{\begin{center} Implementacion Final \end{center}} & 
{\begin{center} 5/18/2011 \end{center}} & 
{\begin{center} 6/7/2011 \end{center}} & 
{\begin{center} 6/7/2011 \end{center}} & 
{\begin{center} Todos los desarrolladores. \end{center}} & 
{\begin{center}  \end{center}} & 
{\begin{center} 6/8/2011 \end{center}}\\
\hline

%{\begin{center}  \end{center}} & 
%{\begin{center}  \end{center}} & 
%{\begin{center}  \end{center}} & 
%{\begin{center}  \end{center}} & 
%{\begin{center}  \end{center}} & 
%{\begin{center}  \end{center}} & 
%{\begin{center}  \end{center}} & 
%{\begin{center}  \end{center}}\\
%\hline
		
		\end{longtable} 
	\end{center}
%\end{document}
        %**********************************************************************