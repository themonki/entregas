%\documentclass[11pt]{article}
%\usepackage[utf8]{inputenc}
%\usepackage[spanish]{babel}
%\usepackage[]{graphicx}
%\usepackage{colortbl} %para las tablas
%\usepackage{longtable} %para las tablas
%\usepackage{geometry}
%\geometry{tmargin=3cm,bmargin=3cm,lmargin=3cm,rmargin=2cm}
%
%\begin{document}

\begin{center}
        \textbf{Resumen}
\end{center}

En el documento se describirá detalladamente El sistema Biblioteca digital, un proyecto con el que
se construirá una aplicación software que permitirá gestionar todos los documentos académicos
generados en la Escuela de Ingenieria de Sistemas y Computación (EISC) de la Universidad del Valle,
para ello el documento se estructurará en  siete grandes secciones. \\

La primera es el \textit{Propósito}, que describe, como su nombre lo indica, el propósito del
Sistema Biblioteca Digital.\\

La segunda es el \textit{Alcance} mostrando los límites del sistema.

Como tercera Sección tenemos el \textit{Contexto}, sección que nos muestra el \textit{Planteamiento
del problema y pertinencia del mismo} donde se responde a preguntas como ¿Por qué se plantea o como
surge el problema y que tan pertienente es?, además de ¿Por qué se decide llevar a cabo esta
solución en particular?, continuando tenemos los \textit{Objetivos} tanto \textit{Objetivos
Generales} como \textit{Objetivos Específicos} mostrando las metas propuestas para lograr la
construcción del sistema y por ende la solución al problema. La \textit{Justificación} describe el
por qué el Sistema Biblioteca Digital es una buena solución para el problema que se viene
presentando en la EISC, seguido del \textit{Área de aplicación} del producto resultado del
proyecto, esto es, la aplicación del Sistema Biblioteca Digital en el caso particular de la
Universidad del Valle y por último \textit{Cronograma de actividades} que plantea una agenda para
seguirse permitiendo realizar de manera ordenada todas las actividades con respecto al desarrollo
del proyecto.\\
 
En la cuarta sección encontramos los  \textit{Requerimientos} que tiene como contenido los
requisitos del sistema que se han planteado por parte de Martha Millán y Mauricio Gaona, y algunos
propuestos por los desarrolladores del proyecto por creerlos pertinentes. Esta sección nos muestra
la \textit{Descripción del sistema}, que explica a grandes rasgos lo que se trata de hacer
manteniendo un enfoque en los requisitos del sistema. La  \textit{Visión y alcance} explica de
manera mucho más profunda el Sistema Biblioteca Digital y al igual que descripción del sistema,
mantiene un enfoque de requisitos. También encontramos los \textit{Involucrados} que son las
personas que supervisan el proyecto y los desarrolladores del mismo y los \textit{Usuarios}
mostrando y definiendo las personas o usuarios finales del sistema, esto es de importancia puesto
que en adelante se referirán a algunos actores con nombres de usuarios descritos aquí. Continuando
en la estructura del documento en cuanto a requerimientos, encontramos la \textit{Matriz de
requerimientos}, una matriz que contiene todos los requerimientos que se han identificado y
\textit{Descripción detallada} donde cada requerimiento de la matriz de requerimientos es pasado a
una plantilla de documento que describe con muchas más profundidad el requerimiento, esto con el
fin de eliminar ambigüedades.\\

La quinta sección corresponde a \textit{Análisis} donde se muestra la \textit{Descripción del
subsistema} planteando el procedimiento a seguir con las iteraciones realizadas en la construcción
del proyecto y el de \textit{Diagrama de casos de uso}, que contiene, precisamente, toda la
información y gráficos sobre los diagramas de casos de uso, además encontramos
\textit{Especificación de casos de uso} donde se toma cada uno de los casos de uso graficados y se
llevan a una plantilla de documento donde se explicará el flujo normal y alternativo, si lo tiene,
que debe de seguir con sus pre y post condiciones.\\

En la sexta sección llamada \textit{Diseño} encontramos los diagramas de clase y el modelo de datos de la aplicación.\\

Finalmente tenemos la sección de \textit{Referencias} donde se listan todos los documentos que se
necesitaron tener en cuenta o se investigaron para la realización de este plan de proyecto.\\

En cuanto a documentación se refiere, se ha completado en un 95\% después de tres revisiones y la
construcción y agregación de los artefactos correspondientes. Para la implementación de la
aplicación ya se tiene en funcionamiento los módulos de gestión de usuarios y catalogación de
documentos, queda faltando las  búsquedas generales, las búsquedas avanzadas y la generación de
reportes, por lo que se podria decir que se a avanzado en un 40\% aproximadamente.

%\end{document}