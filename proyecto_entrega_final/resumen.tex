%\documentclass[11pt]{article}
%\usepackage[utf8]{inputenc}
%\usepackage[spanish]{babel}
%\usepackage[]{graphicx}
%\usepackage{colortbl} %para las tablas
%\usepackage{longtable} %para las tablas
%\usepackage{geometry}
%\geometry{tmargin=3cm,bmargin=3cm,lmargin=3cm,rmargin=2cm}
%
%\begin{document}

\begin{center}
        \textbf{Resumen}
\end{center}

En el presente documento se describirá detalladamente el proyecto \textit{Sistema Biblioteca Digital}, un proyecto con el que se construirá una aplicación software que permitirá gestionar todos los documentos académicos generados en la Escuela de Ingeniería de Sistemas y Computación (EISC) de la Universidad del Valle. Para esta descripción se usarán diferentes artefactos que ayudaran a mostrar el  diseño y funcionamiento del sistema ayudando a la comprensión de este.\\

El Sistema Biblioteca Digital esta dividido en módulos, cada módulo se encuentra especializado en diferentes funciones, los módulos que forman el sistema son: \textit{Consultas}, donde encontramos lo relacionado con consulta básica y consulta avanzada de documentos. \textit{Documento}, donde se maneja la catalogación y la modificación de documentos. \textit{Gestión Documentos}, donde se encuentra la gestión de información relacionada con los documentos, tal como autores, palabras clave, tipo de material, entre otros. \textit{Reportes}, donde se maneja todo lo relacionado con reportes sobre el estado del sistema. \textit{Usuarios}, encargado de realizar toda la gestión de usuarios, tal como registro de nuevos usuarios o modificación de datos de usuario. \textit{Utilidades}, donde se encuentran funcionalidades no propias de la gestión en la biblioteca digital pero usadas por los demás módulos para su correcto funcionamiento y por último el módulo \textit{Principal}, que es el módulo que integra los demás módulos y presenta al usuario final una agradable interfaz gráfica que permite interactuar con el sistema.

%\end{document}