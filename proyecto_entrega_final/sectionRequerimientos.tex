%\documentclass[11pt]{article}
%\usepackage[utf8]{inputenc}
%\usepackage[spanish]{babel}
%\usepackage[]{graphicx}
%\usepackage{colortbl} %para las tablas
%\usepackage{longtable} %para las tablas
%\usepackage{geometry}
%\geometry{tmargin=3cm,bmargin=3cm,lmargin=3cm,rmargin=2cm}
%
%\begin{document}

\subsection{Descripción del sistema}
        El sistema busca tener agrupados y organizados los documentos creados y que posee la
        escuela de ingeniería de sistemas y computación (eisc), teniendo en cuenta, los metadatos
        que el catalogador le otorgue a cada documento y las áreas de interés. Con el fin de tener
        a disposición de la comunidad de la eisc y de interesados en los documentos de las ciencias
        de la computación.
        
        El sistema correrá en un equipo como aplicación de escritorio, que este conectado a 
        Internet para acceder a la base de datos, donde se encontrarán alojados toda la información
        de documentos y usuarios.
        
        \subsection{Visión y alcance}
        El proyecto tiene como fin la elaboración del sistema biblioteca digital solicitado por los
        profesores y representantes de la dirección de la EISC Marta Millan y Mauricio Gaona para
        la poner a la disposición de los interesados en las ciencias de computación los diferentes
        tipos de documentos que se producen en la Escuela por Estudiantes y Docentes.
        
        Los módulos principales del sistema son:
        
        \textbf{La gestión de usuarios:} incluye los tipo de usuario registrado y no registrado.
        Para los registrados se administraran con 3 perfiles de usuarios a los cuales se les otorga
        ciertos permisos. El administrador que gestiona las operaciones y la información 
        relacionada con los demás usuarios, asignar los perfiles y los demás permisos. El 
        catalogador que se le otorga los permisos de catalogación de los documentos digitales. Y
        por ultimo los usuarios normales y los anteriores que podrán consultar y descargar
        documentos. Se almacenaran para los usuarios registrados datos personales, sobre áreas de 
        interés y demográficos.
        
        \textbf{La catalogación de los documentos:} los diferentes documentos se almacenaran con 
        los diferentes metadatos para su consulta. Se debe guardar datos como autor, área a la que 
        pertenece, datos básicos y palabras claves.
        
        \textbf{La generación de Reportes:} Es necesario hacer un seguimiento, a las consultas,
        catalogaciones y descargas de documentos por lo que el sistema debe generar reportes en 
        formato PDF que contengan esta información.
        
                \subsubsection{Entregables}
                \textbf{Primer Entregable:} En el primer entregable se inicion las fases Inicio y
                Elaboración de la metodología RUP. En este entregable se incluyen:
                
                \begin{itemize}
                \item\textbf{Plan del proyecto:} es este documento, que consigna el plan global a
                seguir para el desarrollo del sistema biblioteca digital.
                \item\textbf{Matriz de Requerimientos y descripción detallada:} Comprende los
                requisitos del sistema especificados por los stakeholders y el documento del
                planteamiento del problema.
                Se presenta una matriz que contiene identificación de los requerimientos,
                descripción, fuente, prioridad, estado e involucrados. A cada requerimiento se le
                realiza una descripción detallada usando una plantilla de documento en donde además
                de los datos anteriores de la matriz incluye requerimientos asociados, datos salida
                y resultados esperados.
                \item\textbf{Diagrama de Casos de uso:} Corresponde a la representación gráfica de
                los actores y sus relaciones con las funciones del sistema.
                \item\textbf{Casos de uso expandido:} Son las especificaciones de los casos de uso,
                se realiza por cada caso de uso usando la plantilla de documento que incluye los
                campos precondiciones, flujo normal, flujo alternativo, postcondiciones y 
                excepciones.
                \end{itemize}
                
                \textbf{Segundo Entregable:} Aquí se se refinan fases inicio y elaboración y se
                inicia fase de construcción.
                
                \begin{itemize}
                \item\textbf{Plan de proyecto:} Refinamiento del primer entregable, y adición de
                algunos apartes.
                \item\textbf{Diagrama de casos de uso:} Refinamiento del primer entregable.
                \item\textbf{Casos de uso expandido:} Refinamiento del primer entregable.
                \item\textbf{Requerimientos de usuario norma IEEE830:} Documento que especifica la
                buena obtención de requisitos de sotfware de manera que el cliente conozca bien que
                es lo que desea del sistema, y que los desarrrolladores sepan que es lo que deben
                realizar, tratando de prever malas estimaciones de tiempo y recursos.
                \end{itemize}
                
                \textbf{Tercer Entregable:} Se refinan fases inicio, elaboración y construcción y
                se adicionan artefactos a la fase de construcción.
                
                \begin{itemize}
                \item\textbf{Plan de proyecto:} Refinamiento del segundo entregable.
                \item\textbf{Diagrama de casos de uso:} Refinamiento del segundo entregable.
                \item\textbf{Casos de uso expandido:} Refinamiento del segundo entregable.
                \item\textbf{Requerimientos de usuario norma IEEE830:} Refinamiento del segundo
                entregable.
                \item\textbf{Diagrama del modelo de datos en UML:} Debido a que el sistema
                biblioteca digital hace uso de una base de datos, este diagrama tiene el modelo
                Entidad Relación y Relacional haciendo uso de un profile UML para modelado de
                datos.
                \item\textbf{Diagrama de clases:} Modelo estático que describirá la estructura del
                sistema biblioteca digital mostrando sus clases, atributos y relaciones entre
                ellas, la primera versión de este diagrama de clases esta sujeta a modificaciones
                conforme el proyecto avance y se realicen iteraciones en la fase de construcción.
                Estarán contenidos en el aparte Análisis del plan de proyecto.
                \item\textbf{Diagrama de paquetes:} muestra cómo el sistema biblioteca digital está
                dividido en agrupaciones lógicas mostrando las dependencias entre esas
                agrupaciones. se presenta de modo modular y denotan una jerarquía. Este diagrama
                permite un mejor acoplamiento a nivel externo.
                \item\textbf{Diagramas de secuencia:} En estos diagramas se mostrara la interacción
                de los diferentes objetos del sistema a través del tiempo. Se modelan para cada
                caso de uso, por lo que para este artefacto los casos de uso deben estar lo
                suficientemente desarrrollados y refinados.
                \item\textbf{Prototipos interfaz de usuario:} Serán unos prototipos creados con
                herramientas para gráficos y otros interactivos con el fin de obtener
                retroalimentación de los stakeholders con respecto a los requerimientos, así ellos
                tendrán un adelanto de lo que será la interfaz del sistema.
                \item\textbf{Modelo de implementación inicial:} Se establece estructura para la
                implementación inicial, concretando manejo de ejecutables, ajustes de subsistemas
                de implementación. Se establecen responsables y módulos para la integración.
                \end{itemize}
                
                \textbf{Cuarto Entregable:} Aquí se realizan los últimos refinamientos de las fases
                inicio, elaboración y construcción y se inicia fase de transacción.
                
                \begin{itemize}
                \item\textbf{Plan de proyecto:} Refinamiento del tercer entregable.
                \item\textbf{Diagrama de casos de uso:} Refinamiento del tercer entregable.
                \item\textbf{Casos de uso expandido:} Refinamiento del tercer entregable.
                \item\textbf{Requerimientos de usuario norma IEEE830:} Refinamiento del tercer
                entregable.
                \item\textbf{Diagrama del modelo de datos en UML:} Refinamiento del tercer
                entregable.                
                \item\textbf{Diagrama de clases:} Refinamiento del tercer entregable.
                \item\textbf{Diagrama de paquetes:} Refinamiento del tercer entregable.
                \item\textbf{Diagramas de secuencia:} Refinamiento del tercer entregable.
                \item\textbf{Prototipos interfaz de usuario:} Refinamiento del tercer entregable.
                \item\textbf{Casos de pruebas:} Usando plantilla de documento, se establecen los
                casos de pruebas, describiendo entradas de la prueba, y salidas esperadas, también
                procedimiento para realizar las pruebas. Dependiendo del tipo de prueba podrán ser
                o no automatizables. 
                \item\textbf{Diagramas de despliegue:} Se utiliza para modelar el hardware 
                utilizado en las implementaciones de sistemas y las relaciones entre sus
                componentes.                
                \item\textbf{Material de apoyo al usuario final:} Documentos para el usuario final
                que corresponden a las especificaciones para hacer uso del sistema biblioteca
                digital y ayuda para el usuario.
                \item\textbf{Modelo de implementacion final:} Producto.
                \end{itemize}
                
                \subsubsection{Involucrados}
                Las personas involucradas en el desarrollo del sistema biblioteca digital son los
                representantes de la eisc Martha Millán y Mauricio Gaona, y los desarrolladores del
                proyecto Yerminson Doney Gonzalez Muñóz, Cristian Leonardo Ríos López, Edgar Andrés
                Moncada Taborda, Luis Felipe Vargas Rojas y María Andrea Cruz Blandón.
                
        \subsection{Usuarios}
        \textbf{Usuario normal:} el usuario normal de el sistema tendrá la posibilidad de consultar
        documentos , descargar documentos , solicitar notificaciones de novedades de ciertas áreas
        además de modificar algunos de sus datos.\\
        
        \textbf{Catalogador:} que a su ves es un usuario normal tendrá la capacidad de catalogar
        documentos lo que incluye subir documentos , además de añadir palabras claves y nuevas
        áreas de ciencias de la computación al sistema.\\
        
        \textbf{Administrador:} podrá hacer cambios de perfiles a los usuario el sistema, también
        podrá borrar usuarios, y cambiar datos de los mismos.


   %**********************************************************************************        
		\subsection{Matriz de requerimientos}
               %\documentclass[]{article}
%\usepackage[spanish]{babel}
%\usepackage[utf8]{inputenc}
%\usepackage{geometry}
%\usepackage{colortbl}
%\usepackage{longtable}
%\geometry{tmargin=3cm,bmargin=3cm,lmargin=3cm,rmargin=2cm}
%\begin{document}
%para incluir comentar hasta acá
\begin{center}
\begin{longtable}{|p{0.04\textwidth}|p{0.256\textwidth}|p{0.16\textwidth}|p{0.064\textwidth}|p{0.12\textwidth}|p{0.16\textwidth}|}
\hline
\multicolumn{6}{|>{\columncolor[rgb]{0.8,0.8,0.8}}c|}{ MATRIZ DE REQUERIMIENTOS }\\
\hline
\bf {id} &\bf { Descripción} & \bf {Fuente} & \bf {Prio.} &\bf { Estado} & \bf {Usuarios involucrados}\\
\hline
R1
&        
Permitir el registro de nuevos usuarios
&        
Marta Millán y Mauricio Gaona.
&        
alta
&        
Revisión        
&
Estudiantes, profesores, funcionarios de Univalle y Administrador.\\
\hline
R2
&        
Proporcionar un perfil a cada usuario.
&        
Marta Millán y Mauricio Gaona.
&        
media
&        
Revisión
        
&
Catalogador, Administrador y Usuario normal.\\
\hline
R3
&        
Modificar datos de usuario.
&        
Marta Millán y Mauricio Gaona.
&        
baja
&        
Revisión
&        
Catalogador, usuario normal y administrador.\\
\hline
R4
&        
Eliminar usuario.
&        
Marta Millán.
&        
baja
&        
Revisión
&        
Administrador\\
\hline
R5
&        
Enviar notificación a los usuarios de nuevos documentos registrados en un área de su interés.
&        
Marta Millán y Mauricio Gaona.
&        
media
&        
Revisión
&        
Catalogador, usuario normal, administrador.\\
\hline
R6
&        
Verificar el logueo de usuarios.
&        
Marta Millán y Mauricio Gaona.
&        
media
&        
Revisión
&        
Catalogador, usuario normal, administrador.\\
\hline
%R7
%&        
%Almacenar Documentos Digitales.
%&        
%Marta Millán y Mauricio Gaona.
%&        
%alta
%&        
%Revisión
%&        
%Catalogador, Administrador\\
%\hline
R8
&        
Catalogar documentos.
&        
Marta Millán y Mauricio Gaona.
&        
alta
&        
Revisión
&        
Catalogador y Administrador\\
\hline
R9
&        
Modificar datos de documentos.
&        
Marta Millán y Mauricio Gaona.
&        
media
&        
Revisión
&        
Catalogador y administrador\\
\hline
R10
&        
Llevar un registro de las consultas a documentos realizadas.
&        
Marta Millán y Mauricio Gaona.
&        
media
&        
Revisión
&        
Usuarios registrados, normales, catalogador, administrador y usuarios no registrados. \\
\hline
R11
&
Llevar un registro de las descargas a documentos realizadas
&        
Marta Millán y Mauricio Gaona.
&        
media
&        
Revisión
&        
Usuarios registrados normal, administrador y catalogador\\
\hline
R12
&
Reactivar usuarios eliminados.
&        
Marta Millán y Mauricio Gaona.
&        
media
&        
Revisión
&        
Catalogador\\
\hline
R13
&        
Permitir modificar palabras clave.
&        
Marta Millán y Mauricio Gaona.
&        
media
&        
Revisión
&        
Catalogador, Administrador\\
\hline
R14        
&
Almacenar nuevos autores de los documentos presentes en el sistema.
&        
Marta Millán y Mauricio Gaona.
&        
media
&        
Revisión
&        
Catalogador, Administrador\\
\hline
R15
&        
Modificar información sobre autores existentes.
&        
Marta Millán y Mauricio Gaona.
&        
media
&        
Revisión
&        
Catalogador, Administrador\\
\hline
%R16
%&        
%Añadir palabras claves a documentos almacenados en la BD.
%&        
%Marta Millán y Mauricio Gaona.
%&        
%media
%&        
%Revisión
%&        
%Catalogador, Administrador\\
%\hline
R17
&        
Crear nuevas palabras claves.
&        
Marta Millán y Mauricio Gaona.
&        
media
&        
Revisión
&        
Catalogador, Administrador\\
\hline
R18
&        
Almacenar nuevas áreas de interés con sus respectivas subáreas.
&        
Marta Millán y Mauricio Gaona.
&        
media
&        
Revisión
&        
Catalogador y Administrador.\\
\hline
R19
&        
Permitir la modificación de áreas de interés existentes.
&        
Marta Millán y Mauricio Gaona.
&        
media
&        
Revisión
&        
Catalogador, y administrador.\\
\hline
%R20
%        &
%Permitir el registro de nuevas áreas
%        &        
%Marta Millán y Mauricio Gaona.
%&
%media
%        &
%Revisión
%        &
%Catalogador y Administrador\\
%\hline
R21
&
Permitir la consulta básica de documentos.
&        
Marta Millán y Mauricio Gaona.
&        
alta
&        
Revisión
&        
Usuarios registrados, normales, catalogador, administrador y usuarios no registrados. \\
\hline
R22
&        
Permitir la consulta avanzada de documentos.
&        
Marta Millán y Mauricio Gaona.
&        
alta
&        
Revisión
&        
Usuarios registrados, normales, catalogador, administrador y usuarios no registrados. \\
\hline
R23
&
Permitir la descarga de documentos.
&
Marta Millán y Mauricio Gaona.
&
alta
&
Revisión
&
Usuarios Registrados, catalogadores y administrador.\\
\hline
R24
&
Listar nombre de documento y su correspondiente autor como resultado de una consulta
&
Desarrolladores.
&
media
&
Revisión
&
Usuarios registrados o no registrados, catalogador y administrador.\\
\hline
R25
&        
Mostrar ficha técnica de un documento.
&        
Desarrolladores.
&        
media
&        
Revisión
&        
Usuarios registrados normales, catalogador y administrador, usuarios no registrados.\\
\hline
R26
&        
Generar reportes en formato PDF de los documentos descargados por fecha.
&        
Marta Millán y Mauricio Gaona.
&        
media
&        
Revisión
&        
Administrador.\\
\hline
R27
&
Generar reportes en formato PDF de los documentos descargados por área.
&
Marta Millán y Mauricio Gaona.
&
media
&
Revisión
&
Administrador.\\
\hline
R28
&        
Generar reportes en formato PDF de los documentos existentes por área
&        
Marta Millán y Mauricio Gaona.
&        
media
&        
Revisión
&        
Administrador.\\
\hline
R29
&        
Generar reportes en formato PDF de los documentos existentes por autor.
&        
Marta Millán y Mauricio Gaona.
&        
media
&        
Revisión
&        
Administrador.\\
\hline
R30
&        
Generar reportes en formato PDF del total de usuarios registrados.
&        
Marta Millán y Mauricio Gaona.
&        
media
&        
Revisión
&        
Administrador.\\
\hline
R31
&        
Generar reportes en formato PDF del total de usuarios registrados por fecha.
&        
Marta Millán y Mauricio Gaona.
&        
media
&        
Revisión
&        
Administrador.\\
\hline
R32
&
Generar reportes en formato PDF de las consultas por fecha.
&
Marta Millán y Mauricio Gaona.
&
media
&
Revisión
&
Administrador.\\
\hline
R33
&
Generar reportes en formato PDF de las consultas por área.
&
Marta Millán y Mauricio Gaona.
&
media
&
Revisión
&
Administrador.\\
\hline
R34
&
Generar reportes en formato PDF de los documentos catalogados por fecha.
&
Marta Millán y Mauricio Gaona.
&
media
&
Revisión
&
Administrador.\\
\hline
R35
&
Generar reportes en formato PDF de los documentos descargados por usuario.
&
Marta Millán y Mauricio Gaona.
&
media
&
Revisión
&
Administrador.\\
\hline
R36
&
Generar reportes en formato PDF de los documentos existentes por tipo de documento.
&        
Marta Millán y Mauricio Gaona
&        
media
&        
Revisión
&        
Administrador.\\
\hline
R37
&
Generar reportes en formato PDF de los documentos existentes por formato.
&
Marta Millán y Mauricio Gaona
&
media
&
Revisión
&
Administrador.\\
\hline
R38
&
Permitir el logout del sistema.
&
Desarrolladores
&
media
&
Revisión
&
Todos los usuarios registrados.\\
\hline
\end{longtable}
\end{center}
%\end{document} %para incluir comentar esta linea
                %\pagebreak        
       
        \subsection{Descripción detallada}
                %\documentclass[]{article}
%\usepackage[spanish]{babel}
%\usepackage[utf8]{inputenc}
%\usepackage{geometry}
%\usepackage{colortbl}
%\usepackage{longtable}
%\geometry{tmargin=3cm,bmargin=3cm,lmargin=3cm,rmargin=2cm}
%\begin{document}
%para incluir comentar hasta acá
\begin{center}
\begin{longtable}{|p{0.225\textwidth}|p{0.225\textwidth}|p{0.225\textwidth}|p{0.225\textwidth}|}
\hline
\multicolumn{2}{|p{0.45\textwidth}|}{{\bf {Función del requerimiento:}}
Permitir el registro de nuevos usuarios. } & {\bf{ Estado}} & Análisis \\
\hline
\multicolumn{2}{|p{0.45\textwidth}}{\bf Identificador} &
\multicolumn{2}{|p{0.45\textwidth}|}{R01} \\
\hline
\multicolumn{2}{|p{0.45\textwidth}}{\bf {Tipo de requerimiento}} &
\multicolumn{2}{|p{0.45\textwidth}|}{Funcional}\\
\hline
\bf {Creado por} & Maria Andrea Cruz & \bf {Fecha } & Marzo 31 2011 \\
\hline
\bf {Actualizado por} & Felipe Vargas & \bf {Fecha }& Abril 02 2011\\
\hline
\bf {Actualizado por} & Cristian Ríos & \bf {Fecha }& Mayo 09 2011\\
\hline
\bf Descripción &\multicolumn{3}{p{0.675\textwidth}|}
{El sistema después de que el usuario proporcione los datos correspondiente (login, contraseña, pregunta secreta, respuesta secreta, nombres, apellidos, género, fecha de nacimiento, correo electrónico, nivel de escolaridad, vínculo con Univalle, áreas de interés.) los almacenará en la base de datos. El usuario pasará a ser un usuario registrado, con acceso al sistema mediante logueo.} \\
\hline
\bf Datos de entrada &\multicolumn{3}{p{0.675\textwidth}|}{
Para que un usuario se registre debe de proporcionar obligatoriamente un login, una contraseña, una pregunta secreta, una respuesta secreta a la pregunta, su primer nombre, su primer apellido, su género y correo electrónico. Opcionalmente puede proporcionar el segundo nombre, el segundo apellido, la fecha de nacimiento, el nivel de escolaridad, el vínculo con univalle y sus área de conocimiento de interés.}\\
\hline
\bf Datos de salida &\multicolumn{3}{p{0.675\textwidth}|}
{Un mensaje que informa al usuario que el registro se ha dado de manera exitosa y que puede loguearse de ahora en adelante formando parte de los usuarios registrados.} \\
\hline
\bf Resultados esperados &\multicolumn{3}{p{0.675\textwidth}|}
{El sistema deberá realizar una modificacion en la base de datos de usuarios lo que permite que en un futuro sea reconocido.} \\
\hline
\bf Origen &\multicolumn{3}{p{0.675\textwidth}|}
{Documento de descripción del problema.} \\
\hline
\bf Dirigido a &\multicolumn{3}{p{0.675\textwidth}|}
{Estudiantes , profesores , funcionarios de Univalle y administrador} \\
\hline
\bf Prioridad &\multicolumn{3}{p{0.675\textwidth}|}{5} \\
\hline
\bf Requerimientos Asociados &\multicolumn{3}{p{0.675\textwidth}|}
{} \\
\hline
\multicolumn{4}{|>{\columncolor[rgb]{0.8,0.8,0.8}}c|}{\bf Especificación}\\
\hline
\bf Precondiciones &\multicolumn{3}{p{0.675\textwidth}|}
{El sistema debe estar conectado a la base de datos, la interfaz debe corresponder a la de usuario no registrado, es decir, la inicial.} \\
\hline
\hline
\bf Poscondicion &\multicolumn{3}{p{0.675\textwidth}|}
{El sistema tiene un nuevo usuario que se ve especificado como un nuevo registro en la base de datos en lo que corresponde a usuarios. } \\
\hline
\bf Criterios de Aceptación &\multicolumn{3}{p{0.675\textwidth}|}
{El requerimiento es aceptado si es posible registrarse como usuario en el sistema Biblioteca Digital.} \\
\hline
\end{longtable}
\end{center}
%\end{document} %comentar para inlcuir
                %%\pagebreak
                
                %\documentclass[]{article}
%\usepackage[spanish]{babel}
%\usepackage[utf8]{inputenc}
%\usepackage{geometry}
%\usepackage{colortbl}
%\usepackage{longtable}
%\geometry{tmargin=3cm,bmargin=3cm,lmargin=3cm,rmargin=2cm}
%\begin{document}
%para incluir comentar hasta acá
\begin{center}
\begin{longtable}{|p{0.225\textwidth}|p{0.225\textwidth}|p{0.225\textwidth}|p{0.225\textwidth}|}
\hline
\multicolumn{2}{|p{0.45\textwidth}|}{{\bf {Función del requerimiento:}}
Proporcionar un perfil a cada usuario. } & {\bf{ Estado}} & Análisis \\
\hline
\multicolumn{2}{|p{0.45\textwidth}}{\bf Identificador} &
\multicolumn{2}{|p{0.45\textwidth}|}{R02} \\
\hline
\multicolumn{2}{|p{0.45\textwidth}}{\bf {Tipo de requerimiento}} &
\multicolumn{2}{|p{0.45\textwidth}|}{Funcional}\\
\hline
\bf {Creado por} & Maria Andrea Cruz & \bf {Fecha } & Marzo 31 2011 \\
\hline
\bf {Actualizado por} & Cristian Ríos & \bf {Fecha }& Abril 28 2011\\
\hline
\bf {Actualizado por} & Cristian Ríos & \bf {Fecha } & Mayo 09 2011\\


\hline
\bf Descripción &\multicolumn{3}{p{0.675\textwidth}|}
{El sistema debe proporcionar un perfil a cada usuario, esto aplica a los usuarios que ya están registrados, es decir, se encuentran en la base de datos, esta acción es gestionada por el administrador. Por defecto todo usuario registrado tiene como perfil ‘Usuario Normal’} \\
\hline
\bf Datos de entrada &\multicolumn{3}{p{0.675\textwidth}|}{
Para poder asignar un perfil a cada usuario se presentan al administrador todos los datos del usuario a modificar de manera informativa para verificar la identidad del usuario. El administrador debe de proporcionar el dato perfil con el que se le realizará la actualización al usuario.}\\
\hline
\bf Datos de salida &\multicolumn{3}{p{0.675\textwidth}|}
{El administrador recibe una mensaje notificando el éxito o no de la operación y el usuario implicado recibe una notificación de su cambio de estado en el sistema.} \\
\hline
\bf Resultados esperados &\multicolumn{3}{p{0.675\textwidth}|}
{El sistema deberá actualizar la base de datos permitiendo al usuario que se le a actualizado su perfil tener nuevas funcionalidades en el sistema.} \\
\hline
\bf Origen &\multicolumn{3}{p{0.675\textwidth}|}
{Documento de descripción del problema.} \\
\hline
\bf Dirigido a &\multicolumn{3}{p{0.675\textwidth}|}
{Catalogador, administrador y usuarios normales.} \\
\hline
\bf Prioridad &\multicolumn{3}{p{0.675\textwidth}|}{3} \\
\hline
\bf Requerimientos Asociados &\multicolumn{3}{p{0.675\textwidth}|}
{\begin{itemize}
        \item R06
\end{itemize} } \\
\hline
\multicolumn{4}{|>{\columncolor[rgb]{0.8,0.8,0.8}}c|}{\bf Especificación}\\
\hline
\bf Precondiciones &\multicolumn{3}{p{0.675\textwidth}|}
{El sistema debe estar conectado a una base de datos, un administrador debe haberse logueado en el sistema lo que permite privilegios en cuanto al cambio de perfil de otros usuarios.} \\
\hline
\hline
\bf Poscondicion &\multicolumn{3}{p{0.675\textwidth}|}
{El sistema de ahora en adelante reconoce al usuario modificado en su perfil de manera diferente, otorgándole determinados privilegios así como su interfaz de usuario correspondiente.} \\
\hline
\bf Criterios de Aceptación &\multicolumn{3}{p{0.675\textwidth}|}
{El requerimiento es aceptado si cada usuario que se registre tiene como perfil 'Usuario Normal' y si posteriormente, un usuario administrador puede actualizar el perfil del usuario a 'Catalogador' o 'Administrador'.}
\\
\hline
\end{longtable}
\end{center}
%\end{document} %comentar para inlcuir
                %%\pagebreak
                
                %\documentclass[10pt,a4paper]{article}
%
%
%\usepackage[spanish]{babel}
%\usepackage[utf8]{inputenc}
%\usepackage{geometry}
%\usepackage{colortbl}
%\usepackage{longtable}
%
%\geometry{tmargin=1cm,bmargin=2cm,lmargin=2cm,rmargin=2cm}
%\begin{document}
\begin{center}


\begin{longtable}{|p{3cm}|p{3cm}|p{3cm}|p{3cm}|}

\hline
\multicolumn{2}{|p{6cm}|}{{\bf {Descripción del requerimiento:}}
    Modificar datos de usuario.  } & 
     {\bf{ Estado:}} & Análisis \\
\hline
\bf {Creado por:} & 
	Maria Andrea Cruz   & \bf {Actualizado por:} & Felipe Vargas  \\
\hline
\bf {Fecha Creación } & Marzo 31 2011 & \bf {Fecha de  Actualización }& Abril 2 del 2011\\
\hline 
\multicolumn{2}{|p{6cm}|}{\bf Identificador} & \multicolumn{2}{|p{6cm}|}{R03} \\
\hline
\bf {Tipo de requerimiento:} & No Crítico &  \bf{Tipo de requerimiento:} & Funcional\\     
\hline
\bf Descripción &\multicolumn{3}{|p{10cm}|}
{ El sistema debe proporcionar una opción que permita modificar algunos de  los datos que el usuario registró, para saber que datos modificar lo hace a partir de una consulta a la base de datos que le informe el contenido actual de cada uno de los campos.} \\
\hline
\bf Datos de salida &\multicolumn{3}{|p{10cm}|}
{ El sistema actualiza en la base de datos lo referente a los atributos de usuario y se informa al usuario de la operación exitosa.} \\
\hline
\bf resultados esperados &\multicolumn{3}{|p{10cm}|}
{El sistema se ve estimulado con un cambio en la base de datos con nueva información actualizada y correcta del usuario que solicito esta funcionalidad. } \\
\hline
\bf Origen &\multicolumn{3}{|p{10cm}|}{Documento de descripción del problema.} \\
\hline
\bf Dirigido a  &\multicolumn{3}{|p{10cm}|}
{Catalogador, administrador y usuarios normales.} \\
\hline
\bf Prioridad &\multicolumn{3}{|p{10cm}|}{3} \\
\hline
\bf Requerimientos Asociados &\multicolumn{3}{|p{10cm}|}
{\begin{itemize}
	\item R06
\end{itemize} } \\
\hline
\multicolumn{4}{|>{\columncolor[rgb]{0.8,0.8,0.8}}c|}{\bf Especificación}\\
\hline


\bf Precondiciones &\multicolumn{3}{|p{10cm}|}
{El sistema debe estar conectado a la base de datos de la misma manera debe de haber reconocido a un usuario normal es decir este debe de estar logueado y observando su determinada interfaz correspondiente a su perfil.} \\
\hline
\hline
\bf Poscondiciones &\multicolumn{3}{|p{10cm}|}
{El sistema en la base de datos tiene información actualizada de acuerdo a la los datos que modifico el usuario. Lo que permite una mejor comunicación sistema usuario.} \\
\hline
\bf Criterios de Aceptación &\multicolumn{3}{|p{10cm}|}
{Criterio de aceptación del requerimiento} \\
\hline

\end{longtable}
\end{center}

%\end{document} 
                %%\pagebreak
                
                %\documentclass[10pt,a4paper]{article}
%
%
%\usepackage[spanish]{babel}
%\usepackage[utf8]{inputenc}
%\usepackage{geometry}
%\usepackage{colortbl}
%\usepackage{longtable}
%
%\geometry{tmargin=1cm,bmargin=2cm,lmargin=2cm,rmargin=2cm}
%\begin{document}
\begin{center}


\begin{longtable}{|p{3cm}|p{3cm}|p{3cm}|p{3cm}|}

\hline
\multicolumn{2}{|p{6cm}|}{{\bf {Descripción del requerimiento:}}
    Eliminar usuarios.  } & 
     {\bf{ Estado:}} & Análisis \\
\hline
\bf {Creado por:} & 
	Maria Andrea Cruz   & \bf {Actualizado por:} & Felipe Vargas  \\
\hline
\bf {Fecha Creación } & Marzo 31 2011 & \bf {Fecha de  Actualización }& Abril 2 del 2011\\
\hline 
\multicolumn{2}{|p{6cm}|}{\bf Identificador} & \multicolumn{2}{|p{6cm}|}{R04} \\
\hline
\bf {Tipo de requerimiento:} & No Crítico &  \bf{Tipo de requerimiento:} & Funcional\\     
\hline
\bf Descripción &\multicolumn{3}{|p{10cm}|}
{ El sistema debe proporcionar a los usuarios administradores una opción que permita eliminar usuarios registrados, los usuarios que se desea eliminar se seleccionan a partir de una consulta a la base de datos que muestre el o los usuarios a eliminar.} \\
\hline
\bf Datos de salida &\multicolumn{3}{|p{10cm}|}
{ El sistema actualiza en la base de datos lo referente a la tabla de usuarios se informa al usuario de la operación exitosa.} \\
\hline
\bf resultados esperados &\multicolumn{3}{|p{10cm}|}
{ El sistema se ve estimulado con un cambio en la base de datos con nueva información en la cual se tiene un usuario menos en lo registros.} \\
\hline
\bf Origen &\multicolumn{3}{|p{10cm}|}{Documento de descripción del problema.} \\
\hline
\bf Dirigido a  &\multicolumn{3}{|p{10cm}|}
{Administrador} \\
\hline
\bf Prioridad &\multicolumn{3}{|p{10cm}|}{3} \\
\hline
\bf Requerimientos Asociados &\multicolumn{3}{|p{10cm}|}
{\begin{itemize}
	\item R06
\end{itemize}} \\
\hline
\multicolumn{4}{|>{\columncolor[rgb]{0.8,0.8,0.8}}c|}{\bf Especificación}\\
\hline


\bf Precondiciones &\multicolumn{3}{|p{10cm}|}
{El sistema debe estar conectado a la base de datos , un administrador debe estar logueado para poder tener acceso a la interfaz de eliminación de usuarios proporcionada por el sistema.} \\
\hline
\hline
\bf Poscondiciones &\multicolumn{3}{|p{10cm}|}
{El sistema en caso de tener éxito en la acción elimina de su base de datos el registro del usuario indicado para ser eliminado , por lo cual ya  no podrá ingresar al sistema. En caso de que el usuario no exista ya no es necesario borrarlo  pues no existe.} \\
\hline
\bf Criterios de Aceptación &\multicolumn{3}{|p{10cm}|}
{Criterio de aceptación del requerimiento} \\
\hline

\end{longtable}
\end{center}

%\end{document} 
                %\pagebreak
                
                %\documentclass[10pt,a4paper]{article}
%
%
%\usepackage[spanish]{babel}
%\usepackage[utf8]{inputenc}
%\usepackage{geometry}
%\usepackage{colortbl}
%\usepackage{longtable}
%
%\geometry{tmargin=1cm,bmargin=2cm,lmargin=2cm,rmargin=2cm}
%\begin{document}
\begin{center}


\begin{longtable}{|p{3cm}|p{3cm}|p{3cm}|p{3cm}|}

\hline
\multicolumn{2}{|p{6cm}|}{{\bf {Descripción del requerimiento:}}
     Enviar notificación a los usuarios de nuevos documentos registrados en un área de su interés. } & 
     {\bf{ Estado:}} & Análisis \\
\hline
\bf {Creado por:} & 
	Maria Andrea Cruz   & \bf {Actualizado por:} & Felipe Vargas  \\
\hline
\bf {Fecha Creación } & Marzo 31 2011 & \bf {Fecha de  Actualización }& Abril 2 del 2011\\
\hline 
\multicolumn{2}{|p{6cm}|}{\bf Identificador} & \multicolumn{2}{|p{6cm}|}{R05} \\
\hline
\bf {Tipo de requerimiento:} & No Crítico &  \bf{Tipo de requerimiento:} & Funcional\\     
\hline
\bf Descripción &\multicolumn{3}{|p{10cm}|}
{El sistema debe proporcionar a cada usuario una notificación cuando este ingrese la siguiente vez al sistema de los nuevos documentos que hayan sido catalogados en un área en la que el usuario haya mostrado interés. } \\
\hline
\bf Datos de salida &\multicolumn{3}{|p{10cm}|}
{El sistema informará al usuario de los nuevos documentos que ahora se encuentran en su área de interés. } \\
\hline
\bf resultados esperados &\multicolumn{3}{|p{10cm}|}
{El sistema se ve estimulado cuando los usuarios a partir de las notificaciones empiezan a consultar y descargar los nuevos documentos. } \\
\hline
\bf Origen &\multicolumn{3}{|p{10cm}|}{Documento de descripción del problema.} \\
\hline
\bf Dirigido a  &\multicolumn{3}{|p{10cm}|}
{Catalogador, administrador y usuarios normales.} \\
\hline
\bf Prioridad &\multicolumn{3}{|p{10cm}|}{4} \\
\hline
\bf Requerimientos Asociados &\multicolumn{3}{|p{10cm}|}
{\begin{itemize}
	\item R07
	\item R08
\end{itemize}
} \\
\hline
\multicolumn{4}{|>{\columncolor[rgb]{0.8,0.8,0.8}}c|}{\bf Especificación}\\
\hline


\bf Precondiciones &\multicolumn{3}{|p{10cm}|}
{El sistema debe tener acceso a la base de datos para establecer que usuarios tiene como áreas de interés las que corresponden a los nuevos documentos catalogados.} \\
\hline
\hline
\bf Poscondiciones &\multicolumn{3}{|p{10cm}|}
{El sistema nota que los usuarios han sido notificados cuando ellos empiezan a consultar estos nuevos documentos , todo gracias a la información suministrada por el sistema.} \\
\hline
\bf Criterios de Aceptación &\multicolumn{3}{|p{10cm}|}
{Criterio de aceptación del requerimiento} \\
\hline

\end{longtable}
\end{center}

%\end{document} 
                \pagebreak
                
                %\documentclass[]{article}
%\usepackage[spanish]{babel}
%\usepackage[utf8]{inputenc}
%\usepackage{geometry}
%\usepackage{colortbl}
%\usepackage{longtable}
%\geometry{tmargin=3cm,bmargin=3cm,lmargin=3cm,rmargin=2cm}
%\begin{document}
%para incluir comentar hasta acá
\begin{center}
\begin{longtable}{|p{0.225\textwidth}|p{0.225\textwidth}|p{0.225\textwidth}|p{0.225\textwidth}|}
\hline
\multicolumn{2}{|p{0.45\textwidth}|}{{\bf {Descripción del requerimiento:}}
Verificar el logueo de usuarios. } & {\bf{ Estado}} & Análisis \\
\hline
\multicolumn{2}{|p{0.45\textwidth}}{\bf Identificador} &
\multicolumn{2}{|p{0.45\textwidth}|}{R06} \\
\hline
\multicolumn{2}{|p{0.45\textwidth}}{\bf {Tipo de requerimiento}} &
\multicolumn{2}{|p{0.45\textwidth}|}{Funcional}\\
\hline
\bf {Creado por} & Maria Andrea Cruz & \bf {Fecha } & Marzo 31 2011 \\
\hline
\bf {Actualizado por} & María Andrea & \bf {Fecha }& Abril 28 2011\\
\hline
\bf {Actualizado por} & Cristian Ríos & \bf {Fecha }& Mayo 09 2011\\
\hline
\bf Descripción &\multicolumn{3}{p{0.675\textwidth}|}
{ El sistema debe proporcionar la autenticación de cualquier usuario que se encuentre registrado y desee ingresar al sistema, esto se logra realizando una consulta a la base de datos con el nombre de usuario proporcionado por el usuario, si existe el usuario en la base de datos el resultado de la consulta será la contraseña, la cual se comparará con la proporcionada por el usuario, si ambas coinciden entonces se le otorgará al usuario el acceso al sistema.} \\
\hline
\bf Datos de entrada &\multicolumn{3}{p{0.675\textwidth}|}{
El usuario debe de proporcionar su login y su contraseña para tener acceso al sistema.}\\
\hline
\bf Datos de salida &\multicolumn{3}{p{0.675\textwidth}|}
{ El sistema permite al usuario ingresar al sistema mostrando la interfaz de su respectivo perfil.} \\
\hline
\bf Resultados esperados &\multicolumn{3}{p{0.675\textwidth}|}
{ El deberá cambiar la interfaz del usuario permitiéndole nuevas funcionalidades como por ejemplo opciones de descarga de documentos.} \\
\hline
\bf Origen &\multicolumn{3}{p{0.675\textwidth}|}
{Documento de descripción del problema.} \\
\hline
\bf Dirigido a &\multicolumn{3}{p{0.675\textwidth}|}
{Catalogador, administrador y usuarios normales.} \\
\hline
\bf Prioridad &\multicolumn{3}{p{0.675\textwidth}|}{5} \\
\hline
\bf Requerimientos Asociados &\multicolumn{3}{p{0.675\textwidth}|}
{\begin{itemize}
        \item R01
\end{itemize}} \\\hline
\multicolumn{4}{|>{\columncolor[rgb]{0.8,0.8,0.8}}c|}{\bf Especificación}\\
\hline
\bf Precondiciones &\multicolumn{3}{p{0.675\textwidth}|}
{El sistema debe estar conectado a la base de datos ya que debe reconocer que el usuario esta registrado realizando una consulta a esta a partir de los datos suministrados en la interfaz de login.} \\
\hline
\hline
\bf Poscondicion &\multicolumn{3}{p{0.675\textwidth}|}
{El sistema debe de mostrar una nueva interfaz para el usuario lo que indica que este proceso se realizó satisfactoriamente, en caso de no lograrlo se le seguirá solicitando que realice el proceso de loguin si quiere acceder a nuevas funcionalidades.} \\
\hline
\bf Criterios de Aceptación &\multicolumn{3}{p{0.675\textwidth}|}
{El requerimiento se acepta cuando un usuario registrado ingresando los datos correctos de login y password pueda ingresar al sistema; y cuando se presente impedimento para ingresar al sistema cuando los datos sean incorrectos} \\
\hline
\end{longtable}
\end{center}
%\end{document} %comentar para inlcuir
                %%\pagebreak
                
                %%\documentclass[10pt,a4paper]{article}
%
%
%\usepackage[spanish]{babel}
%\usepackage[utf8]{inputenc}
%\usepackage{geometry}
%\usepackage{colortbl}
%\usepackage{longtable}
%
%\geometry{tmargin=1cm,bmargin=2cm,lmargin=2cm,rmargin=2cm}
%\begin{document}
\begin{center}


\begin{longtable}{|p{3cm}|p{3cm}|p{3cm}|p{3cm}|}

\hline
\multicolumn{2}{|p{6cm}|}{{\bf {Descripción del requerimiento:}}
   Almacenar documentos para su descarga   } & 
     {\bf{ Estado:}} & Análisis \\
\hline
\bf {Creado por:} & 
	Maria Andrea Cruz   & \bf {Actualizado por:} & Felipe Vargas  \\
\hline
\bf {Fecha Creación } & Marzo 31 2011 & \bf {Fecha de  Actualización }& Abril 2 del 2011\\
\hline 
\multicolumn{2}{|p{6cm}|}{\bf Identificador} & \multicolumn{2}{|p{6cm}|}{R07} \\
\hline
\bf {Tipo de requerimiento:} & Crítico &  \bf{Tipo de requerimiento:} & Funcional\\     
\hline
\bf Descripción &\multicolumn{3}{|p{10cm}|}
{ El sistema debe de proporcionar la manera de acceder a los archivos que se encuentran almacenados en el disco duro y de los cuales en la base de datos se almacena su ruta en el sistema host para poder realizar la modificación de ellos si es necesario y el envío de una copia de este al usuario que lo este solicitando , además, el sistema debe permitir que se agreguen más documentos al sistema por parte de usuarios que tengan como perfil administrador o catalogador..} \\
\hline
\bf Datos de salida &\multicolumn{3}{|p{10cm}|}
{ El sistema proporciona una copia del archivo al cual se hace referencia y es enviada al usuario solicitante.} \\
\hline
\bf Resultados esperados &\multicolumn{3}{|p{10cm}|}
{ El sistema se ve estimulado ya que si el documento enviado al usuario le fue de utilidad este continuará con las búsquedas en la base de datos para solicitar más archivos de su interés.} \\
\hline
\bf Origen &\multicolumn{3}{|p{10cm}|}{Documento de descripción del problema.} \\
\hline
\bf Dirigido a  &\multicolumn{3}{|p{10cm}|}
{Catalogador, administrador.} \\
\hline
\bf Prioridad &\multicolumn{3}{|p{10cm}|}{5} \\
\hline
\bf Requerimientos Asociados &\multicolumn{3}{|p{10cm}|}
{\begin{itemize}
	\item R08
\end{itemize}} \\
\hline
\multicolumn{4}{|>{\columncolor[rgb]{0.8,0.8,0.8}}c|}{\bf Especificación}\\
\hline


\bf Precondiciones &\multicolumn{3}{|p{10cm}|}
{El sistema debe tener acceso a la base de datos la cual proporciona información útil acerca de la dirección del archivo que solicita el usuario así como una interfaz sencilla con las opciones para la descarga. } \\
\hline
\hline
\bf Poscondiciones &\multicolumn{3}{|p{10cm}|}
{El sistema muestra una interfaz referente a la culminación de la descarga y guarda los registros de dicha descarga ,el sistema reconoce que se ha descargado un nuevo archivo de la base de datos lo que será de utilidad para un usuario que estaba interesado en dicha área de interés. } \\
\hline
\bf Criterios de Aceptación &\multicolumn{3}{|p{10cm}|}
{Criterio de aceptación del requerimiento} \\
\hline

\end{longtable}
\end{center}

%\end{document} 
                %%\pagebreak
                
                %\documentclass[]{article}
%\usepackage[spanish]{babel}
%\usepackage[utf8]{inputenc}
%\usepackage{geometry}
%\usepackage{colortbl}
%\usepackage{longtable}
%\geometry{tmargin=3cm,bmargin=3cm,lmargin=3cm,rmargin=2cm}
%\begin{document}
%para incluir comentar hasta acá
\begin{center}
\begin{longtable}{|p{0.225\textwidth}|p{0.225\textwidth}|p{0.225\textwidth}|p{0.225\textwidth}|}
\hline
\multicolumn{2}{|p{0.45\textwidth}|}{{\bf {Descripción del requerimiento:}}
Asignar datos a documentos. } & {\bf{ Estado}} & Análisis \\
\hline
\bf {Creado por} & Maria Andrea Cruz & \bf {Actualizado por} & Maria Andrea Cruz \\
\hline
\bf {Fecha Creación } & Marzo 31 2011 & \bf {Fecha de Actualización }& Abril 28 2011\\
\hline
\multicolumn{2}{|p{0.45\textwidth}}{\bf Identificador} &
\multicolumn{2}{|p{0.45\textwidth}|}{R08} \\
\hline
\multicolumn{2}{|p{0.45\textwidth}}{\bf {Tipo de requerimiento}} &
\multicolumn{2}{|p{0.45\textwidth}|}{Funcional}\\
\hline
\bf Descripción &\multicolumn{3}{p{0.675\textwidth}|}
{El sistema debe de proporcionar la posibilidad a los usuarios que tengan como perfil catalogador o administrador de asignar datos a los documentos que se encuentran referenciados en la base de datos, estos datos son información sobre el documento como el autor, el titulo principal, el título secundario o traducido, la editorial, la fecha de publicación, la fecha de creación, el idioma en el que esta escrito, los derechos de autor, la descripción o resumen del material, las palabras claves relacionadas con este, el área de las ciencias de la computación a la que pertenece, el formato del archivo, el tamaño del archivo en bytes, la resolución en píxeles y el software recomendado para abrir el documento. Para saber que documentos existen y a saber a cuales de estos debe adicionar datos se debe realizar una consulta a la base de datos.} \\
\hline
\bf Datos de salida &\multicolumn{3}{p{0.675\textwidth}|}
{ Se informa al usuario que solicito la adición de los datos del documento si de la operación fue exitosa.} \\
\hline
\bf Resultados esperados &\multicolumn{3}{p{0.675\textwidth}|}
{ El sistema deberá realizar actualizaciones en la base de datos con nueva información actualizada y correcta del documento..} \\
\hline
\bf Origen &\multicolumn{3}{p{0.675\textwidth}|}
{Documento de descripción del problema.} \\
\hline
\bf Dirigido a &\multicolumn{3}{p{0.675\textwidth}|}
{Catalogador, administrador.} \\
\hline
\bf Prioridad &\multicolumn{3}{p{0.675\textwidth}|}{3} \\
\hline
\bf Requerimientos Asociados &\multicolumn{3}{p{0.675\textwidth}|}
{\begin{itemize}
        \item R06
        \item R07
        \item R17
        \item R14
        \item R18
\end{itemize}} \\\hline
\multicolumn{4}{|>{\columncolor[rgb]{0.8,0.8,0.8}}c|}{\bf Especificación}\\
\hline
\bf Precondiciones &\multicolumn{3}{p{0.675\textwidth}|}
{El sistema debe estar conectado a la base de datos y tener logueado a un usuario con los perfiles correspondientes a administrador o catalogador los cuales tienen acceso a una interfaz de catalogación, la cual muestra información correspondiente a los documentos y permite introducir una nueva para estos. Los autores, palabras clave y áreas de ciencias de la computación asociados con el documento deben de existir previamente en el sistema.} \\
\hline
\bf Poscondicion &\multicolumn{3}{p{0.675\textwidth}|}
{El sistema en su base de datos tiene ahora información de un documento agregado anteriormente, con esta información el documento podrá ser consultado y descargado posteriormente.} \\
\hline
\bf Criterios de Aceptación &\multicolumn{3}{p{0.675\textwidth}|}
{El requerimiento es aceptado si los usuarios con perfil administrador o catalogador puede adicionar datos a los documentos, esto es catalogar el documento.} \\
\hline
\end{longtable}
\end{center}
%\end{document} %comentar para inlcuir
                %%\pagebreak
                
                %\documentclass[10pt,a4paper]{article}
%
%
%\usepackage[spanish]{babel}
%\usepackage[utf8]{inputenc}
%\usepackage{geometry}
%\usepackage{colortbl}
%\usepackage{longtable}
%
%\geometry{tmargin=1cm,bmargin=2cm,lmargin=2cm,rmargin=2cm}
%\begin{document}
\begin{center}


\begin{longtable}{|p{3cm}|p{3cm}|p{3cm}|p{3cm}|}

\hline
\multicolumn{2}{|p{6cm}|}{{\bf {Descripción del requerimiento:}}
   Modificar datos de documentos.   } & 
     {\bf{ Estado:}} & Análisis \\
\hline
\bf {Creado por:} & 
	Maria Andrea Cruz   & \bf {Actualizado por:} & Felipe Vargas  \\
\hline
\bf {Fecha Creación } & Marzo 31 2011 & \bf {Fecha de  Actualización }& Abril 2 del 2011\\
\hline 
\multicolumn{2}{|p{6cm}|}{\bf Identificador} & \multicolumn{2}{|p{6cm}|}{R09} \\
\hline
\bf {Tipo de requerimiento:} & No Crítico &  \bf{Tipo de requerimiento:} & Funcional\\     
\hline
\bf Descripción &\multicolumn{3}{|p{10cm}|}
{ El sistema debe proporcionar la manera de poder modificar los datos de los documentos que se encuentran referenciados en la base de datos por usuarios que tengan como perfil catalogador o administrador. El sistema debe de proporcionar al usuario que este realizando la modificación los datos actuales que tiene el documento para que pueda decidir cuales de estos campos será modificados.} \\
\hline
\bf Datos de salida &\multicolumn{3}{|p{10cm}|}
{ El sistema actualiza en la base de datos lo referente a los atributos de documentos y se informa al usuario que este realizando la actualización de la operación exitosa.} \\
\hline
\bf Resultados esperados &\multicolumn{3}{|p{10cm}|}
{ El sistema se ve estimulado con un cambio en la base de datos con nueva información actualizada y correcta del documento que el usuario desea actualizar.} \\
\hline
\bf Origen &\multicolumn{3}{|p{10cm}|}{Documento de descripción del problema.} \\
\hline
\bf Dirigido a  &\multicolumn{3}{|p{10cm}|}
{Catalogador, administrador.} \\
\hline
\bf Prioridad &\multicolumn{3}{|p{10cm}|}{3} \\
\hline
\bf Requerimientos Asociados &\multicolumn{3}{|p{10cm}|}
{\begin{itemize}
	\item R06
	\item R07
	\item R08
\end{itemize}} \\
\hline
\multicolumn{4}{|>{\columncolor[rgb]{0.8,0.8,0.8}}c|}{\bf Especificación}\\
\hline


\bf Precondiciones &\multicolumn{3}{|p{10cm}|}
{El sistema debe estar conectado a la base de datos y el usuario registrado debe corresponder a un catalogador o un administrador lo que les proporcionará una interfaz que permitirá listar documentos y de esta manera modificarlos.} \\
\hline
\hline
\bf Poscondiciones &\multicolumn{3}{|p{10cm}|}
{El sistema presenta una actualización en la base de datos con respeto al registro de documentos donde se ha corregido o actualizado información lo que permite tener una mejor experiencia por parte  de los usuarios al consultar.} \\
\hline
\bf Criterios de Aceptación &\multicolumn{3}{|p{10cm}|}
{Criterio de aceptación del requerimiento} \\
\hline

\end{longtable}
\end{center}

%\end{document} 
                \pagebreak
                
                %\documentclass[10pt,a4paper]{article}
%
%
%\usepackage[spanish]{babel}
%\usepackage[utf8]{inputenc}
%\usepackage{geometry}
%\usepackage{colortbl}
%\usepackage{longtable}
%
%\geometry{tmargin=1cm,bmargin=2cm,lmargin=2cm,rmargin=2cm}
%\begin{document}
\begin{center}


\begin{longtable}{|p{3cm}|p{3cm}|p{3cm}|p{3cm}|}

\hline
\multicolumn{2}{|p{6cm}|}{{\bf {Descripción del requerimiento:}}
     Eliminar documentos y su información. } & 
     {\bf{ Estado:}} & Análisis \\
\hline
\bf {Creado por:} & 
	Maria Andrea Cruz   & \bf {Actualizado por:} & Felipe Vargas  \\
\hline
\bf {Fecha Creación } & Marzo 31 2011 & \bf {Fecha de  Actualización }& Abril 2 del 2011\\
\hline 
\multicolumn{2}{|p{6cm}|}{\bf Identificador} & \multicolumn{2}{|p{6cm}|}{R10} \\
\hline
\bf {Tipo de requerimiento:} & No Crítico &  \bf{Tipo de requerimiento:} & Funcional\\     
\hline
\bf Descripcion &\multicolumn{3}{|p{10cm}|}
{El sistema debe proporcionar la posibilidad de que los usuarios que tengan como perfil administrador o catalogador puedan eliminar cualquier documento que se encuentre referenciado en la base de datos y toda la información relacionada con el documento. El sistema debe de proveer por medio de una búsqueda en la base de datos el o los documentos que se desean eliminar.} \\
\hline
\bf Datos de salida &\multicolumn{3}{|p{10cm}|}
{El sistema actualiza en la base de datos lo referente a la tabla que mantiene la información de los documentos eliminando de esta la tupla del documento en cuestión y elimina del sistema host el documento, notificando al usuario que este realizando la eliminación de la operación exitosa.} \\
\hline
\bf resultados esperados &\multicolumn{3}{|p{10cm}|}
{El sistema se ve estimulado con un cambio en la base de datos con nueva información en la cual se tiene un documento menos en lo registros.} \\
\hline
\bf Origen &\multicolumn{3}{|p{10cm}|}{Documento de descripción del problema.} \\
\hline
\bf Dirigido a  &\multicolumn{3}{|p{10cm}|}
{Catalogador y administrador} \\
\hline
\bf Prioridad &\multicolumn{3}{|p{10cm}|}{3} \\
\hline
\bf Requerimientos Asociados &\multicolumn{3}{|p{10cm}|}
{\begin{itemize}
	\item R06
	\item R07
	\item R08
\end{itemize}} \\
\hline
\multicolumn{4}{|>{\columncolor[rgb]{0.8,0.8,0.8}}c|}{\bf Especificación}\\
\hline


\bf Precondiciones &\multicolumn{3}{|p{10cm}|}
{El sistema debe estar conectado a la base de datos, también debe estar logueado un usuario bajo el perfil de administrador o de catalogador lo que le proporcionará una interfaz para esta operación de eliminación de documentos.} \\
\hline
\hline
\bf Poscondiciones &\multicolumn{3}{|p{10cm}|}
{El sistema cambia en su base de datos el registro correspondiente  a documentos eliminando el documento solicitado lo que en un futuro no permitirá que usuarios en su resultado de consulta tengan en la respuesta alguna coincidencia con este documento} \\
\hline
\bf Criterios de Aceptación &\multicolumn{3}{|p{10cm}|}
{Criterio de aceptación del requerimiento} \\
\hline

\end{longtable}
\end{center}

%\end{document} 
                %%\pagebreak
                
                %\documentclass[10pt,a4paper]{article}
%
%
%\usepackage[spanish]{babel}
%\usepackage[utf8]{inputenc}
%\usepackage{geometry}
%\usepackage{colortbl}
%\usepackage{longtable}
%
%\geometry{tmargin=1cm,bmargin=2cm,lmargin=2cm,rmargin=2cm}
%\begin{document}
\begin{center}


\begin{longtable}{|p{3cm}|p{3cm}|p{3cm}|p{3cm}|}

\hline
\multicolumn{2}{|p{6cm}|}{{\bf {Descripción del requerimiento:}}
     Permitir el almacenamiento, y modificación de información de  documentos solo a administradores y catalogadores. } & 
     {\bf{ Estado:}} & Análisis \\
\hline
\bf {Creado por:} & 
	Maria Andrea Cruz   & \bf {Actualizado por:} & Felipe Vargas  \\
\hline
\bf {Fecha Creación } & Marzo 31 2011 & \bf {Fecha de  Actualización }& Abril 2 del 2011\\
\hline 
\multicolumn{2}{|p{6cm}|}{\bf Identificador} & \multicolumn{2}{|p{6cm}|}{R11} \\
\hline
\bf {Tipo de requerimiento:} & Crítico &  \bf{Tipo de requerimiento:} & Funcional\\     
\hline
\bf Descripcion &\multicolumn{3}{|p{10cm}|}
{El sistema debe de permitir operaciones realizada sobre los documentos a usuarios con perfil administrador o catalogador, estas operaciones son, la creación o carga de nuevos documentos en el sistema, la asignación o modificación de cualquier dato relacionado con el documento y la eliminación tanto del documento como de la información asociada a este.} \\
\hline
\bf Datos de salida &\multicolumn{3}{|p{10cm}|}
{El sistema actualiza en la base de datos lo referente a los atributos de documentos si lo realizado fue una modificación o actualiza en la base de datos las tablas relacionadas eliminando de estas las tuplas a las que hace referencia el documento y se informa al usuario que este realizando la operación el éxito de la misma.} \\
\hline
\bf resultados esperados &\multicolumn{3}{|p{10cm}|}
{El sistema se ve estimulado con un cambio en la base de datos con nueva información actualizada y correcta del documento o eliminación de esta según las especificaciones del usuario.} \\
\hline
\bf Origen &\multicolumn{3}{|p{10cm}|}{Documento de descripción del problema.} \\
\hline
\bf Dirigido a  &\multicolumn{3}{|p{10cm}|}
{Catalogador y Administrador} \\
\hline
\bf Prioridad &\multicolumn{3}{|p{10cm}|}{3} \\
\hline
\bf Requerimientos Asociados &\multicolumn{3}{|p{10cm}|}
{\begin{itemize}
	\item R02
	\item R08
\end{itemize}} \\
\hline
\multicolumn{4}{|>{\columncolor[rgb]{0.8,0.8,0.8}}c|}{\bf Especificación}\\
\hline


\bf Precondiciones &\multicolumn{3}{|p{10cm}|}
{El sistema debe estar conectado a la base de datos y los usuarios que ingresen al sistema deben de estar bajo e perfil de administrador o catalogador para que se otorguen privilegios en cuanto a gestión de documentos.} \\
\hline
\hline
\bf Poscondiciones &\multicolumn{3}{|p{10cm}|}
{El sistema muestra a los usuarios la interfaz correspondiente a su perfil con sus determinados privilegios, las operaciones realizadas afectan de manera directa la estructura de documentos en la base de datos} \\
\hline
\bf Criterios de Aceptación &\multicolumn{3}{|p{10cm}|}
{Criterio de aceptación del requerimiento} \\
\hline

\end{longtable}
\end{center}

%\end{document} 
                %%\pagebreak
                
                %\documentclass[10pt,a4paper]{article}
%
%
%\usepackage[spanish]{babel}
%\usepackage[utf8]{inputenc}
%\usepackage{geometry}
%\usepackage{colortbl}
%\usepackage{longtable}
%
%\geometry{tmargin=1cm,bmargin=2cm,lmargin=2cm,rmargin=2cm}
%\begin{document}
\begin{center}


\begin{longtable}{|p{3cm}|p{3cm}|p{3cm}|p{3cm}|}

\hline
\multicolumn{2}{|p{6cm}|}{{\bf {Descripción del requerimiento:}}
  Clasificar documentos por área de interés. } & 
     {\bf{ Estado:}} & Análisis \\
\hline
\bf {Creado por:} & 
	Maria Andrea Cruz   & \bf {Actualizado por:} & Felipe Vargas  \\
\hline
\bf {Fecha Creación } & Marzo 31 2011 & \bf {Fecha de  Actualización }& Abril 2 del 2011\\
\hline 
\multicolumn{2}{|p{6cm}|}{\bf Identificador} & \multicolumn{2}{|p{6cm}|}{R12} \\
\hline
\bf {Tipo de requerimiento:} & No Crítico &  \bf{Tipo de requerimiento:} & Funcional\\     
\hline
\bf Descripción &\multicolumn{3}{|p{10cm}|}
{El sistema debe de permitir clasificar los documentos registrados en el sistema según su tema, esto se logra relacionando el documento con una o varias área de interés existente en el sistema.} \\
\hline
\bf Datos de salida &\multicolumn{3}{|p{10cm}|}
{El sistema permite catalogar los documentos en un área según el tema que estos tratan ayudando a los usuarios en su búsqueda de documentos.} \\
\hline
\bf resultados esperados &\multicolumn{3}{|p{10cm}|}
{El sistema se ve estimulado con un cambio  en la base de datos modificando los registros de la tabla que mantiene los documentos al actualizar estos.} \\
\hline
\bf Origen &\multicolumn{3}{|p{10cm}|}{Documento de descripción del problema.} \\
\hline
\bf Dirigido a  &\multicolumn{3}{|p{10cm}|}
{Catalogador, administrador y usuarios normales.} \\
\hline
\bf Prioridad &\multicolumn{3}{|p{10cm}|}{5} \\
\hline
\bf Requerimientos Asociados &\multicolumn{3}{|p{10cm}|}
{\begin{itemize}
	\item R07
	\item R08
\end{itemize}
} \\
\hline
\multicolumn{4}{|>{\columncolor[rgb]{0.8,0.8,0.8}}c|}{\bf Especificación}\\
\hline


\bf Precondiciones &\multicolumn{3}{|p{10cm}|}
{El sistema debe de estar conectado a la base de datos permitiendo la consulta referente a las áreas de interés que serán asignadas a los documentos mediante una interfaz a la hora de catalogarlos.} \\
\hline
\hline
\bf Poscondiciones &\multicolumn{3}{|p{10cm}|}
{El sistema tiene en su base de datos información de todos los documetnos de acuerdo a su área de interés lo que facilita la consulta de determinados usuarios interesadosen un área específica} \\
\hline
\bf Criterios de Aceptación &\multicolumn{3}{|p{10cm}|}
{Criterio de aceptación del requerimiento} \\
\hline

\end{longtable}
\end{center}

%\end{document} 
                %%\pagebreak
                
                %\documentclass[10pt,a4paper]{article}
%
%
%\usepackage[spanish]{babel}
%\usepackage[utf8]{inputenc}
%\usepackage{geometry}
%\usepackage{colortbl}
%\usepackage{longtable}
%
%\geometry{tmargin=1cm,bmargin=2cm,lmargin=2cm,rmargin=2cm}
%\begin{document}
\begin{center}


\begin{longtable}{|p{3cm}|p{3cm}|p{3cm}|p{3cm}|}

\hline
\multicolumn{2}{|p{6cm}|}{{\bf {Descripción del requerimiento:}}
 Restricción de la clasificación de documentos por área de interés. } & 
     {\bf{ Estado:}} & Análisis \\
\hline
\bf {Creado por:} & 
	Maria Andrea Cruz   & \bf {Actualizado por:} & Felipe Vargas  \\
\hline
\bf {Fecha Creación } & Marzo 31 2011 & \bf {Fecha de  Actualización }& Abril 2 del 2011\\
\hline 
\multicolumn{2}{|p{6cm}|}{\bf Identificador} & \multicolumn{2}{|p{6cm}|}{R13} \\
\hline
\bf {Tipo de requerimiento:} & No Crítico &  \bf{Tipo de requerimiento:} & Funcional\\     
\hline
\bf Descripcion &\multicolumn{3}{|p{10cm}|}
{El sistema deberá permitir clasificar los documentos registrados en el sistema según su tema solo por usuarios que tengan como perfil catalogador.} \\
\hline
\bf Datos de salida &\multicolumn{3}{|p{10cm}|}
{El sistema permite catalogar los documentos en un área según el tema que estos tratan ayudando a los usuarios en su búsqueda de documentos.} \\
\hline
\bf resultados esperados &\multicolumn{3}{|p{10cm}|}
{El sistema se ve estimulado con un cambio  en la base de datos modificando los registros de la tabla que mantiene los documentos al actualizar estos.} \\
\hline
\bf Origen &\multicolumn{3}{|p{10cm}|}{Documento de descripción del problema.} \\
\hline
\bf Dirigido a  &\multicolumn{3}{|p{10cm}|}
{Catalogador, administrador y usuarios normales.} \\
\hline
\bf Prioridad &\multicolumn{3}{|p{10cm}|}{5} \\
\hline
\bf Requerimientos Asociados &\multicolumn{3}{|p{10cm}|}
{\begin{itemize}
	\item R02
\end{itemize}} \\
\hline
\multicolumn{4}{|>{\columncolor[rgb]{0.8,0.8,0.8}}c|}{\bf Especificación}\\
\hline


\bf Precondiciones &\multicolumn{3}{|p{10cm}|}
{El sistema debe estar conectado a la base de datos lo que permite consultar y validar que el tipo de usuario es un catalogador o administrador que es el único con permisos para registrar documentos.} \\
\hline
\hline
\bf Poscondiciones &\multicolumn{3}{|p{10cm}|}
{El sistema en su base de datos tiene una mayor cantidad de documentos lo mejorara los resultados de las consultas por parte de usuarios interesados en esta área de interés.} \\
\hline
\bf Criterios de Aceptación &\multicolumn{3}{|p{10cm}|}
{Criterio de aceptación del requerimiento} \\
\hline

\end{longtable}
\end{center}

%\end{document} 
                %%\pagebreak
                
                %\documentclass[10pt,a4paper]{article}
%
%
%\usepackage[spanish]{babel}
%\usepackage[utf8]{inputenc}
%\usepackage{geometry}
%\usepackage{colortbl}
%\usepackage{longtable}
%
%\geometry{tmargin=1cm,bmargin=2cm,lmargin=2cm,rmargin=2cm}
%\begin{document}
\begin{center}


\begin{longtable}{|p{3cm}|p{3cm}|p{3cm}|p{3cm}|}

\hline
\multicolumn{2}{|p{6cm}|}{{\bf {Descripción del requerimiento:}}
Almacenar información de los autores. } & 
     {\bf{ Estado:}} &  \\
\hline
\bf {Creado por:} & 
	Maria Andrea Cruz   & \bf {Actualizado por:} & Felipe Vargas  \\
\hline
\bf {Fecha Creación } & Marzo 31 2011 & \bf {Fecha de  Actualización }& Abril 2 del 2011\\
\hline 
\multicolumn{2}{|p{6cm}|}{\bf Identificador} & \multicolumn{2}{|p{6cm}|}{R14} \\
\hline
\bf {Tipo de requerimiento:} & No Crítico &  \bf{Tipo de requerimiento:} & Funcional\\     
\hline
\bf Descripcion &\multicolumn{3}{|p{10cm}|}
{El sistema debe de proporcionar la manera de poder almacenar en el sistema datos de los autores de los documentos que se encuentran registrados en el sistema, los datos que se mantiene de un autor son: el nombre, apellido, correo electrónico y acrónimo, donde todos son cadenas de caracteres.} \\
\hline
\bf Datos de salida &\multicolumn{3}{|p{10cm}|}
{El sistema permite almacenar al usuario información de los autores de los documentos mostrando al usuario que este realizando la operación un mensaje de notificación cuando la operación tiene éxito.} \\
\hline
\bf resultados esperados &\multicolumn{3}{|p{10cm}|}
{El sistema se ve estimulado con un cambio  en la base de datos al tener nuevas  tablas para mantener información de los autores.} \\
\hline
\bf Origen &\multicolumn{3}{|p{10cm}|}{Documento de descripción del problema.} \\
\hline
\bf Dirigido a  &\multicolumn{3}{|p{10cm}|}
{Catalogador, administrador.} \\
\hline
\bf Prioridad &\multicolumn{3}{|p{10cm}|}{5} \\
\hline
\bf Requerimientos Asociados &\multicolumn{3}{|p{10cm}|}
{\begin{itemize}
	\item R06
	\item R08
\end{itemize}} \\
\hline
\multicolumn{4}{|>{\columncolor[rgb]{0.8,0.8,0.8}}c|}{\bf Especificación}\\
\hline


\bf Precondiciones &\multicolumn{3}{|p{10cm}|}
{El sistema debe estar conectado a la base de datos permitiendo añadir registros respecto a los autores, que son manejados por los documentos a la hora de su catalogación dentro de la  biblioteca digital, el perfil debe de ser administrador o catalogador.} \\
\hline
\hline
\bf Poscondiciones &\multicolumn{3}{|p{10cm}|}
{El sistema en su base de datos tiene un nuevo documento. Lo que permite que el catalogo de documentos se realice de manera eficiente ya que esto permite realizar de mejor manera las consultas.
} \\
\hline
\bf Criterios de Aceptación &\multicolumn{3}{|p{10cm}|}
{Criterio de aceptación del requerimiento} \\
\hline

\end{longtable}
\end{center}

%\end{document} 
                %%\pagebreak
                
                %\documentclass[]{article}
%\usepackage[spanish]{babel}
%\usepackage[utf8]{inputenc}
%\usepackage{geometry}
%\usepackage{colortbl}
%\usepackage{longtable}
%\geometry{tmargin=3cm,bmargin=3cm,lmargin=3cm,rmargin=2cm}
%\begin{document}
%para incluir comentar hasta acá
\begin{center}
\begin{longtable}{|p{0.225\textwidth}|p{0.225\textwidth}|p{0.225\textwidth}|p{0.225\textwidth}|}
\hline
\multicolumn{2}{|p{0.45\textwidth}|}{{\bf {Función del requerimiento:}}
Permitir modificar los datos de los autores existentes en el sistema. } & {\bf{ Estado}} & Análisis \\
\hline
\bf {Creado por} & Maria Andrea Cruz & \bf {Actualizado por} & Cristian Ríos\\
\hline
\bf {Fecha Creación } & Marzo 31 2011 & \bf {Fecha de Actualización }& 
Abril 28 2011\\
Mayo 09 2011\\
\hline
\multicolumn{2}{|p{0.45\textwidth}}{\bf Identificador} &
\multicolumn{2}{|p{0.45\textwidth}|}{R15} \\
\hline
\multicolumn{2}{|p{0.45\textwidth}}{\bf {Tipo de requerimiento}} &
\multicolumn{2}{|p{0.45\textwidth}|}{Funcional}\\
\hline
\bf Descripción &\multicolumn{3}{p{0.675\textwidth}|}
{ El sistema debe de permitir a los usuarios con perfil administrador o catalogador modificar los datos de los autores que en el momento se encuentren registrados. El sistema proporcionará al usuario que este tratando de realizar la operación un lista de los posibles autores a modificar, de alli podrá seleccionar al que desee y una vez seleccionado, el sistema debe mostrar la información actual del autor y permitir asignar nueva información} \\
\hline
\bf Datos de entrada &\multicolumn{3}{p{0.675\textwidth}|}{
El usuario que dese modificar un autor existente debe de proporcionar su nombre, apellido, dirección de correo electrónico y acrónimo, si alguno de estos datos no se es proporcionado, se toma el dato que existe actualmente.}\\
\hline
\bf Datos de salida &\multicolumn{3}{p{0.675\textwidth}|}
{El sistema genera un mensaje y notifica al usuario que este realizando la operación del éxito o no de la misma.} \\
\hline
\bf Resultados esperados &\multicolumn{3}{p{0.675\textwidth}|}
{El sistema deberá actualizar en la tabla de la base de datos que mantiene la información de los autores el registro relacionado con el autor al que se le han modificado los datos. } \\
\hline
\bf Origen &\multicolumn{3}{p{0.675\textwidth}|}
{Documento de descripción del problema.} \\
\hline
\bf Dirigido a &\multicolumn{3}{p{0.675\textwidth}|}
{Catalogador y administrador} \\
\hline
\bf Prioridad &\multicolumn{3}{p{0.675\textwidth}|}{3} \\
\hline
\bf Requerimientos Asociados &\multicolumn{3}{p{0.675\textwidth}|}
{\begin{itemize}
        \item R14
\end{itemize}} \\
\hline
\multicolumn{4}{|>{\columncolor[rgb]{0.8,0.8,0.8}}c|}{\bf Especificación}\\
\hline
\bf Precondiciones &\multicolumn{3}{p{0.675\textwidth}|}
{El sistema debe estar conectado a la base de datos permitiendo modificar registros respecto a los autores registrados en la biblioteca digital y realizar consultas del estado actual de los autores, además el usuario que desee realizar la operación debe de haber ingresado al sistema y tener como perfil administrador o catalogador.} \\
\hline
\bf Poscondicion &\multicolumn{3}{p{0.675\textwidth}|}
{El sistema realiza una actualizacion de información referente al autor modificado en la base de datos} \\
\hline
\bf Criterios de Aceptación &\multicolumn{3}{p{0.675\textwidth}|}
{El requerimiento es aceptado si al ingresar un usuario adminsitrador o catalogador al sistema este tiene la opción de poder modificar los datos de los autores existentes en el sistema.} \\
\hline
\end{longtable}
\end{center}
%\end{document} %comentar para inlcuir
                %%\pagebreak
                
                %%\documentclass[10pt,a4paper]{article}
%
%
%\usepackage[spanish]{babel}
%\usepackage[utf8]{inputenc}
%\usepackage{geometry}
%\usepackage{colortbl}
%\usepackage{longtable}
%
%\geometry{tmargin=1cm,bmargin=2cm,lmargin=2cm,rmargin=2cm}
%\begin{document}
\begin{center}


\begin{longtable}{|p{3cm}|p{3cm}|p{3cm}|p{3cm}|}

\hline
\multicolumn{2}{|p{6cm}|}{{\bf {Descripción del requerimiento:}}
  Asociar documentos del sistema con palabras clave.    } & 
     {\bf{ Estado:}} & Análisis \\
\hline
\bf {Creado por:} & 
	Maria Andrea Cruz   & \bf {Actualizado por:} & Felipe Vargas  \\
\hline
\bf {Fecha Creación } & Marzo 31 2011 & \bf {Fecha de  Actualización }& Abril 2 del 2011\\
\hline 
\multicolumn{2}{|p{6cm}|}{\bf Identificador} & \multicolumn{2}{|p{6cm}|}{R16} \\
\hline
\bf {Tipo de requerimiento:} & No Crítico &  \bf{Tipo de requerimiento:} & Funcional\\     
\hline
\bf Descripcion &\multicolumn{3}{|p{10cm}|}
{El sistema debe de permitir a los usuarios con perfil administrador o catalogador asociar palabras clave a los documentos existentes en el sistema, las palabras clave deben también de existir en el sistema antes de poder usarse. Estas palabras clave son útiles al momento de realizar la búsqueda de algún documento en el sistema. } \\
\hline
\bf Datos de salida &\multicolumn{3}{|p{10cm}|}
{El sistema actualiza en la tabla de la base de datos que mantiene la información de los documentos el registro relacionado con el documento al que se le han agregado nuevas palabras clave y proporciona un mensaje al usuario notificando el éxito de la operación. } \\
\hline
\bf resultados esperados &\multicolumn{3}{|p{10cm}|}
{ El sistema se ve estimulado con un cambio  en la base de datos ya que se actualizará en la tabla que mantiene la información de los documentos el registro del documento al que se le a agregado nuevas palabras clave.} \\
\hline
\bf Origen &\multicolumn{3}{|p{10cm}|}{Documento de descripción del problema.} \\
\hline
\bf Dirigido a  &\multicolumn{3}{|p{10cm}|}
{Catalogador, administrador y usuarios normales.} \\
\hline
\bf Prioridad &\multicolumn{3}{|p{10cm}|}{5} \\
\hline
\bf Requerimientos Asociados &\multicolumn{3}{|p{10cm}|}
{\begin{itemize}
	\item R08
\end{itemize} } \\
\hline
\multicolumn{4}{|>{\columncolor[rgb]{0.8,0.8,0.8}}c|}{\bf Especificación}\\
\hline


\bf Precondiciones &\multicolumn{3}{|p{10cm}|}
{El sistema debe estar conectado a la base de datos, contar con un libro que se desee catalogar y unas palabras clave, además el usuario debe de tener como perfil administrador o catalogador.} \\
\hline
\hline
\bf Poscondiciones &\multicolumn{3}{|p{10cm}|}
{El sistema almacena en sus registros el nuevo documento junto con la palabra clave asociada lo que permite que las consultas bien realizadas tengan éxito seguro debido al uso de palabras que perimiten identificar temas.} \\
\hline
\bf Criterios de Aceptación &\multicolumn{3}{|p{10cm}|}
{Criterio de aceptación del requerimiento} \\
\hline

\end{longtable}
\end{center}

%\end{document} 
                %%\pagebreak
                
                %\documentclass[10pt,a4paper]{article}
%
%
%\usepackage[spanish]{babel}
%\usepackage[utf8]{inputenc}
%\usepackage{geometry}
%\usepackage{colortbl}
%\usepackage{longtable}
%
%\geometry{tmargin=1cm,bmargin=2cm,lmargin=2cm,rmargin=2cm}
%\begin{document}
\begin{center}


\begin{longtable}{|p{3cm}|p{3cm}|p{3cm}|p{3cm}|}

\hline
\multicolumn{2}{|p{6cm}|}{{\bf {Descripción del requerimiento:}}
    Creación de nuevas palabras clave.  } & 
     \bf{ Estado:} & Análisis \\
\hline
\bf {Creado por:} & 
	Maria Andrea Cruz   & \bf {Actualizado por:} & Felipe Vargas  \\
\hline
\bf {Fecha Creación } & Marzo 31 2011 & \bf {Fecha de  Actualización }& Abril 2 del 2011\\
\hline 
\multicolumn{2}{|p{6cm}|}{\bf Identificador} & \multicolumn{2}{|p{6cm}|}{R17} \\
\hline
\bf {Tipo de requerimiento:} & No Crítico &  \bf{Tipo de requerimiento:} & Funcional\\     
\hline
\bf Descripcion &\multicolumn{3}{|p{10cm}|}
{El sistema debe de permitir a los usuarios con perfil administrador o catalogador la creación de nuevas palabras clave en el sistema, para poder crear una nueva palabra clave es necesario darle un nombre y una descripción. } \\
\hline
\bf Datos de salida &\multicolumn{3}{|p{10cm}|}
{El sistema agregará una nueva palabra clave a la lista de palabras clave que mantiene el sistema y proporcionara un mensaje al usuario indicando que se realizó la operación. } \\
\hline
\bf resultados esperados &\multicolumn{3}{|p{10cm}|}
{ El sistema se ve estimulado con un cambio  en la base de datos ya que se creará una nueva tupla en la taba que mantiene las palabras clave.} \\
\hline
\bf Origen &\multicolumn{3}{|p{10cm}|}{Documento de descripción del problema.} \\
\hline
\bf Dirigido a  &\multicolumn{3}{|p{10cm}|}
{Catalogador, administrador y usuarios normales.} \\
\hline
\bf Prioridad &\multicolumn{3}{|p{10cm}|}{5} \\
\hline
\bf Requerimientos Asociados &\multicolumn{3}{|p{10cm}|}
{\begin{itemize}
	\item R06
	\item R08
\end{itemize} } \\
\hline
\multicolumn{4}{|>{\columncolor[rgb]{0.8,0.8,0.8}}c|}{\bf Especificación}\\
\hline


\bf Precondiciones &\multicolumn{3}{|p{10cm}|}
{El sistema debe estar conectado a la base de datos además de contar con un libro que se desee catalogar del cual  no tenemos palabras claves que lo identifiquen, el usuario debe tener como perfil administrador o catalogador} \\
\hline
\hline
\bf Poscondiciones &\multicolumn{3}{|p{10cm}|}
{El sistema en su base de datos tiene unos registros en cuanto a palabras claves lo que hace la búsqueda de documentos más sactisfactoriamente.} \\
\hline
\bf Criterios de Aceptación &\multicolumn{3}{|p{10cm}|}
{Criterio de aceptación del requerimiento} \\
\hline

\end{longtable}
\end{center}

%\end{document} 
                %\pagebreak
                
                %\documentclass[10pt,a4paper]{article}
%
%
%\usepackage[spanish]{babel}
%\usepackage[utf8]{inputenc}
%\usepackage{geometry}
%\usepackage{colortbl}
%\usepackage{longtable}
%
%\geometry{tmargin=1cm,bmargin=2cm,lmargin=2cm,rmargin=2cm}
%\begin{document}
\begin{center}


\begin{longtable}{|p{3cm}|p{3cm}|p{3cm}|p{3cm}|}

\hline
\multicolumn{2}{|p{6cm}|}{{\bf {Descripción del requerimiento:}}
  Almacenar áreas de interés con sus respectivas subáreas.    } & 
     {\bf{ Estado:}} & Análisis \\
\hline
\bf {Creado por:} & 
	Maria Andrea Cruz   & \bf {Actualizado por:} & Felipe Vargas  \\
\hline
\bf {Fecha Creación } & Marzo 31 2011 & \bf {Fecha de  Actualización }& Abril 2 del 2011\\
\hline 
\multicolumn{2}{|p{6cm}|}{\bf Identificador} & \multicolumn{2}{|p{6cm}|}{R18} \\
\hline
\bf {Tipo de requerimiento:} & No Crítico &  \bf{Tipo de requerimiento:} & Funcional\\     
\hline
\bf Descripción &\multicolumn{3}{|p{10cm}|}
{ El sistema debe proporcionar la manera de poder almacenar en el sistema una lista con las áreas de las ciencias de la computación que estarán disponibles, además, se debe de poder indicar cuales áreas son subáreas de otras áreas y una descripción de cada área} \\
\hline
\bf Datos de salida &\multicolumn{3}{|p{10cm}|}
{ El sistema proporcionará una nueva tabla en la base de datos que permitirá almacenar la información relacionada con las áreas de la ciencia de la computación.} \\
\hline
\bf Resultados esperados &\multicolumn{3}{|p{10cm}|}
{ El sistema se ve estimulado con un cambio  en la base de datos al tener nuevas  tablas para mantener información de las áreas de las ciencias de la computación..} \\
\hline
\bf Origen &\multicolumn{3}{|p{10cm}|}{Documento de descripción del problema.} \\
\hline
\bf Dirigido a  &\multicolumn{3}{|p{10cm}|}
{Catalogador, administrador y usuarios normales.} \\
\hline
\bf Prioridad &\multicolumn{3}{|p{10cm}|}{5} \\
\hline
\bf Requerimientos Asociados &\multicolumn{3}{|p{10cm}|}
{ \begin{itemize}
	\item R08
\end{itemize} } \\
\hline
\multicolumn{4}{|>{\columncolor[rgb]{0.8,0.8,0.8}}c|}{\bf Especificación}\\
\hline


\bf Precondiciones &\multicolumn{3}{|p{10cm}|}
{El sistema debe estar conectado a la base de datos y no tener previamente la área que se trata de crear. El usuario debe de tener como perfil administrador o catalogador.
} \\
\hline
\hline
\bf Poscondiciones &\multicolumn{3}{|p{10cm}|}
{El sistema en su base de datos tiene una descripción mas detallada y amplia de las áreas respecto a ciencias de la computación facilitando la recuperacion de documentos mediante este criterio de búsqueda.} \\
\hline
\bf Criterios de Aceptación &\multicolumn{3}{|p{10cm}|}
{Criterio de aceptación del requerimiento} \\
\hline

\end{longtable}
\end{center}

%\end{document} 
                %%\pagebreak
                
                %\documentclass[10pt,a4paper]{article}
%
%
%\usepackage[spanish]{babel}
%\usepackage[utf8]{inputenc}
%\usepackage{geometry}
%\usepackage{colortbl}
%\usepackage{longtable}
%
%\geometry{tmargin=1cm,bmargin=2cm,lmargin=2cm,rmargin=2cm}
%\begin{document}
\begin{center}


\begin{longtable}{|p{3cm}|p{3cm}|p{3cm}|p{3cm}|}

\hline
\multicolumn{2}{|p{6cm}|}{{\bf {Descripción del requerimiento:}}
    Restringir la modificación y la eliminación de áreas de interés a usuarios con perfil de administrador o catalogador.  } & 
     {\bf{ Estado:}} & Análisis \\
\hline
\bf {Creado por:} & 
	Maria Andrea Cruz   & \bf {Actualizado por:} & Felipe Vargas  \\
\hline
\bf {Fecha Creación } & Marzo 31 2011 & \bf {Fecha de  Actualización }& Abril 2 del 2011\\
\hline 
\multicolumn{2}{|p{6cm}|}{\bf Identificador} & \multicolumn{2}{|p{6cm}|}{R19} \\
\hline
\bf {Tipo de requerimiento:} & No Crítico &  \bf{Tipo de requerimiento:} & Funcional\\     
\hline
\bf Descripción &\multicolumn{3}{|p{10cm}|}
{ El sistema debe proporcionar la manera de poder realizar cambios en la lista de áreas de interés y sus subáreas y permitir la eliminación de áreas o subáreas existentes solo por usuarios que ingresen al sistema con perfil administrador o catalogador, los cambios  se refieren a la edición del nombre y/o la descripción del área, así como también las subáreas que tiene cada área.} \\
\hline
\bf Datos de salida &\multicolumn{3}{|p{10cm}|}
{ El sistema actualiza en la base de datos lo referente a los atributos de las áreas de interes si lo realizado fue una modificación o actualiza en la base de datos las tablas relacionadas eliminando de estas las tuplas a las que hace referencia la área de interes y se informa al usuario que este realizando la operación el éxito de la misma.} \\
\hline
\bf Resultados esperados &\multicolumn{3}{|p{10cm}|}
{ El sistema se ve estimulado con un cambio en la base de datos con nueva información actualizada y correcta del área de interes o la eliminación de esta según las especificaciones del usuario..} \\
\hline
\bf Origen &\multicolumn{3}{|p{10cm}|}{Documento de descripción del problema.} \\
\hline
\bf Dirigido a  &\multicolumn{3}{|p{10cm}|}
{Catalogador, administrador y usuarios normales.} \\
\hline
\bf Prioridad &\multicolumn{3}{|p{10cm}|}{5} \\
\hline
\bf Requerimientos Asociados &\multicolumn{3}{|p{10cm}|}
{ \begin{itemize}
	\item R08
	\item R18
	\item R20
\end{itemize} } \\
\hline
\multicolumn{4}{|>{\columncolor[rgb]{0.8,0.8,0.8}}c|}{\bf Especificación}\\
\hline


\bf Precondiciones &\multicolumn{3}{|p{10cm}|}
{El sistema debe de estar conectado a la base de datos permitiendo modificar o eliminar un área de interes de sus registros, el usuario debe estar bajo el perfil de catalogador o administrador.} \\
\hline
\hline
\bf Poscondiciones &\multicolumn{3}{|p{10cm}|}
{El sistema contiene ahora en su base de datos información mejorada que permite obtener documentos de mejor forma al realizar una consulta, o un documento faltante en caso de que se haya eliminado. } \\
\hline
\bf Criterios de Aceptación &\multicolumn{3}{|p{10cm}|}
{Criterio de aceptación del requerimiento} \\
\hline

\end{longtable}
\end{center}

%\end{document} 
                %%\pagebreak
                
                %%\documentclass[10pt,a4paper]{article}
%
%
%\usepackage[spanish]{babel}
%\usepackage[utf8]{inputenc}
%\usepackage{geometry}
%\usepackage{colortbl}
%\usepackage{longtable}
%
%\geometry{tmargin=1cm,bmargin=2cm,lmargin=2cm,rmargin=2cm}
%\begin{document}
\begin{center}


\begin{longtable}{|p{3cm}|p{3cm}|p{3cm}|p{3cm}|}

\hline
\multicolumn{2}{|p{6cm}|}{{\bf {Descripción del requerimiento:}}
  Permitir el registro de nuevas áreas de interés.    } & 
     {\bf{ Estado:}} & Análisis \\
\hline
\bf {Creado por:} & 
	Maria Andrea Cruz   & \bf {Actualizado por:} & Felipe Vargas  \\
\hline
\bf {Fecha Creación } & Marzo 31 2011 & \bf {Fecha de  Actualización }& Abril 2 del 2011\\
\hline 
\multicolumn{2}{|p{6cm}|}{\bf Identificador} & \multicolumn{2}{|p{6cm}|}{R20} \\
\hline
\bf {Tipo de requerimiento:} & No Crítico &  \bf{Tipo de requerimiento:} & Funcional\\     
\hline
\bf Descripción &\multicolumn{3}{|p{10cm}|}
{El sistema debe de permitir el registro en el sistema de nuevas áreas de interés, para poder realizar un nuevo registro en necesario suministrar el nombre del área, una descripción de esta y si es una subárea, indicar a que área pertenece. Este registro lo pueden hacer usuarios que tengan como perfil administrador o catalogador. } \\
\hline
\bf Datos de salida &\multicolumn{3}{|p{10cm}|}
{ El sistema actualiza en la base de datos lo referente a la tabla que mantiene los datos de las áreas de interés añadiendo una nueva tupla que representa una nueva área y se informa al usuario que este realizando la operación el éxito de la misma.} \\
\hline
\bf Resultados esperados &\multicolumn{3}{|p{10cm}|}
{ El sistema se ve estimulado con un cambio en la base de datos con nuevas tuplas agregada a la tabla que mantiene las áreas de interés.} \\
\hline
\bf Origen &\multicolumn{3}{|p{10cm}|}{Documento de descripción del problema.} \\
\hline
\bf Dirigido a  &\multicolumn{3}{|p{10cm}|}
{Catalogador, administrador.} \\
\hline
\bf Prioridad &\multicolumn{3}{|p{10cm}|}{5} \\
\hline
\bf Requerimientos Asociados &\multicolumn{3}{|p{10cm}|}
{ \begin{itemize}
	\item R08
	\item R18
\end{itemize} } \\
\hline
\multicolumn{4}{|>{\columncolor[rgb]{0.8,0.8,0.8}}c|}{\bf Especificación}\\
\hline


\bf Precondiciones &\multicolumn{3}{|p{10cm}|}
{El sistema debe de haberse iniciado, al igual que una conexión con la base de datos y estar bajo el perfil de administrador o catalogador.} \\
\hline
\hline
\bf Poscondiciones &\multicolumn{3}{|p{10cm}|}
{El sistema con nuevas áreas permitra catalogar otros documentos que no se habian tenido en cuenta ya que el área no existía generando mayor número de documentos.} \\
\hline
\bf Criterios de Aceptación &\multicolumn{3}{|p{10cm}|}
{Criterio de aceptación del requerimiento} \\
\hline

\end{longtable}
\end{center}

%\end{document} 
                %%\pagebreak
                
                %\documentclass[10pt,a4paper]{article}
%
%
%\usepackage[spanish]{babel}
%\usepackage[utf8]{inputenc}
%\usepackage{geometry}
%\usepackage{colortbl}
%\usepackage{longtable}
%
%\geometry{tmargin=1cm,bmargin=2cm,lmargin=2cm,rmargin=2cm}
%\begin{document}
\begin{center}


\begin{longtable}{|p{3cm}|p{3cm}|p{3cm}|p{3cm}|}

\hline
\multicolumn{2}{|p{6cm}|}{{\bf {Descripción del requerimiento:}}
  Permitir la consulta básica de documentos.    } & 
     {\bf{ Estado:}} & Análisis \\
\hline
\bf {Creado por:} & 
	Maria Andrea Cruz   & \bf {Actualizado por:} & Felipe Vargas  \\
\hline
\bf {Fecha Creación } & Marzo 31 2011 & \bf {Fecha de  Actualización }& Abril 2 del 2011\\
\hline 
\multicolumn{2}{|p{6cm}|}{\bf Identificador} & \multicolumn{2}{|p{6cm}|}{R21} \\
\hline
\bf {Tipo de requerimiento:} & Crítico &  \bf{Tipo de requerimiento:} & Funcional\\     
\hline
\bf Descripción &\multicolumn{3}{|p{10cm}|}
{ El sistema debe permitir la consulta básica, donde el usuario registrado o no registrado ingresa una frase o palabra para realizar su búsqueda.} \\
\hline
\bf Datos de salida &\multicolumn{3}{|p{10cm}|}
{ El sistema debe desplegar el resultado de la búsqueda en la pantalla.} \\
\hline
\bf Resultados esperados &\multicolumn{3}{|p{10cm}|}
{ El sistema se ve estimulado al realizar una consulta a la base de datos,  que si no es vacío despegla el resultado como una lista en la pantalla,  si la búsqueda es vacía muestra una notificación de documento no encontrado.} \\
\hline
\bf Origen &\multicolumn{3}{|p{10cm}|}{Marta Millán y Mauricio Gaona.} \\
\hline
\bf Dirigido a  &\multicolumn{3}{|p{10cm}|}
{Usuarios normales registrados o no registrados, catalogador y administrador.} \\
\hline
\bf Prioridad &\multicolumn{3}{|p{10cm}|}{5} \\
\hline
\bf Requerimientos Asociados &\multicolumn{3}{|p{10cm}|}
{
	\begin{itemize}
		\item R24
	\end{itemize}
} \\
\hline
\multicolumn{4}{|>{\columncolor[rgb]{0.8,0.8,0.8}}c|}{\bf Especificación}\\
\hline


\bf Precondiciones &\multicolumn{3}{|p{10cm}|}
{Haber iniciado el sistema} \\
\hline
\hline
\bf Poscondiciones &\multicolumn{3}{|p{10cm}|}
{Una lista de documentos en la pantalla o una notificación de no encontrado el documento.} \\
\hline
\bf Criterios de Aceptación &\multicolumn{3}{|p{10cm}|}
{Criterio de aceptación del requerimiento} \\
\hline

\end{longtable}
\end{center}

%\end{document} 
                %%\pagebreak
                
				%\documentclass[]{article}
%\usepackage[spanish]{babel}
%\usepackage[utf8]{inputenc}
%\usepackage{geometry}
%\usepackage{colortbl}
%\usepackage{longtable}
%\geometry{tmargin=3cm,bmargin=3cm,lmargin=3cm,rmargin=2cm}
%\begin{document}
%para incluir comentar hasta acá
\begin{center}
\begin{longtable}{|p{0.225\textwidth}|p{0.225\textwidth}|p{0.225\textwidth}|p{0.225\textwidth}|}
\hline
\multicolumn{2}{|p{0.45\textwidth}|}{{\bf {Función del requerimiento:}}
Permitir la consulta avanzada de documentos. } & {\bf{ Estado}} & Análisis \\
\hline
\multicolumn{2}{|p{0.45\textwidth}}{\bf Identificador} &
\multicolumn{2}{|p{0.45\textwidth}|}{R22} \\
\hline
\multicolumn{2}{|p{0.45\textwidth}}{\bf {Tipo de requerimiento}} &
\multicolumn{2}{|p{0.45\textwidth}|}{Funcional}\\
\hline
\bf {Creado por} & María Andrea Cruz & \bf {Fecha  } & Marzo 31 2011\\
\hline
\bf {Actualizado por} & María Andrea Cruz  & \bf {Fecha  }& Abril 28 2011\\
\hline
\bf {Actualizado por} & Cristian Ríos  & \bf {Fecha  }& Mayo 10 2011\\

\hline
\bf Descripción &\multicolumn{3}{p{0.675\textwidth}|}
{El sistema debe permitir la consulta avanzada, donde el usuario registrado o no registrado ingresa alguno de los siguientes datos: Titulo, autor, palabra clave, tipo de documento y/o área, y como datos adicionales idioma del documento, formato del archivo y fecha de publicación.} \\
\hline
\bf Datos de entrada &\multicolumn{3}{p{0.675\textwidth}|}{
El usuario que desee realizar la búsqueda deberá proporcionar alguno o todos de los siguientes datos: título del documento, autor, palabra clave, área a la que pertenece el documento, tipo de documento, idioma del documento, formato del archivo del documento y/o fecha de publicación. El dato que no sea proporcionando no será tenido en cuenta para realizar la búsqueda.}\\
\hline
\bf Datos de salida &\multicolumn{3}{p{0.675\textwidth}|}
{El sistema debe desplegar el resultado de la búsqueda en la pantalla.} \\
\hline
\bf Resultados esperados &\multicolumn{3}{p{0.675\textwidth}|}
{El sistema debe realizar una consulta a la base de datos, si la consulta no es vacía desplegar el resultado como una lista en la pantalla, de lo contrario mostrar una notificación de documentos no encontrado.} \\
\hline
\bf Origen &\multicolumn{3}{p{0.675\textwidth}|}
{Documento de descripción del problema.} \\
\hline
\bf Dirigido a &\multicolumn{3}{p{0.675\textwidth}|}
{Usuarios registrados(normal, administrador y catalogador) y no registrados.} \\
\hline
\bf Prioridad &\multicolumn{3}{p{0.675\textwidth}|}{5} \\
\hline
\bf Requerimientos Asociados &\multicolumn{3}{p{0.675\textwidth}|}
{\begin{itemize}
         \item R24
\end{itemize}} \\
\hline
\multicolumn{4}{|>{\columncolor[rgb]{0.8,0.8,0.8}}c|}{\bf Especificación}\\
\hline
\bf Precondiciones &\multicolumn{3}{p{0.675\textwidth}|}
{Haber iniciado el sistema.} \\
\hline
\hline
\bf Poscondicion &\multicolumn{3}{p{0.675\textwidth}|}
{Una lista de documentos en la pantalla o una notificación de no encontrado el documento. } \\
\hline
\bf Criterios de Aceptación &\multicolumn{3}{p{0.675\textwidth}|}
{El requerimiento es aceptado si cualquier usuario logra realizar consultas de documentos en la biblioteca digital especificando parámetros bien definidos.} \\
\hline
\end{longtable}
\end{center}
%\end{document} %comentar para inlcuir
                %%\pagebreak
                
                %\documentclass[]{article}
%\usepackage[spanish]{babel}
%\usepackage[utf8]{inputenc}
%\usepackage{geometry}
%\usepackage{colortbl}
%\usepackage{longtable}
%\geometry{tmargin=3cm,bmargin=3cm,lmargin=3cm,rmargin=2cm}
%\begin{document}
%para incluir comentar hasta acá
\begin{center}
\begin{longtable}{|p{0.225\textwidth}|p{0.225\textwidth}|p{0.225\textwidth}|p{0.225\textwidth}|}
\hline
\multicolumn{2}{|p{0.45\textwidth}|}{{\bf {Descripción del requerimiento:}}
Permitir la descarga de documentos. } & {\bf{ Estado}} & Análisis \\
\hline
\bf {Creado por} & Maria Andrea Cruz & \bf {Actualizado por} & María Andrea  \\
\hline
\bf {Fecha Creación } & Marzo 31 2011 & \bf {Fecha de Actualización }& Abril 28 2011\\
\hline
\multicolumn{2}{|p{0.45\textwidth}}{\bf Identificador} &
\multicolumn{2}{|p{0.45\textwidth}|}{R23} \\
\hline
\multicolumn{2}{|p{0.45\textwidth}}{\bf {Tipo de requerimiento}} &
\multicolumn{2}{|p{0.45\textwidth}|}{Funcional}\\
\hline
\bf Descripción &\multicolumn{3}{p{0.675\textwidth}|}
{El sistema debe permitir la descarga de documentos, esta descarga puede ser realizada solo por usuarios registrados.} \\
\hline
\bf Datos de salida &\multicolumn{3}{p{0.675\textwidth}|}
{El sistema muestra un cuadro de dialogo al usuario donde se pregunta por la dirección en el equipo cliente donde quedará el documento, y presenta una notificación de descarga. El usuario ahora tendrá una copia del documento} \\
\hline
\bf Resultados esperados &\multicolumn{3}{p{0.675\textwidth}|}
{El sistema debe verificar el documento que se descarga y el usuario que lo descarga para llevar el registro de la descarga en la base de datos. El sistema debe descargar en la ruta indicada el documento que solicita descargar el usuario.} \\
\hline
\bf Origen &\multicolumn{3}{p{0.675\textwidth}|}
{Documento de descripción del problema.} \\
\hline
\bf Dirigido a &\multicolumn{3}{p{0.675\textwidth}|}
{Usuarios Registrados, catalogadores, administrador y usuarios normales.} \\
\hline
\bf Prioridad &\multicolumn{3}{p{0.675\textwidth}|}{5} \\
\hline
\bf Requerimientos Asociados &\multicolumn{3}{p{0.675\textwidth}|}
{\begin{itemize}
\item R6
\item R11
\end{itemize}} \\
\hline
\multicolumn{4}{|>{\columncolor[rgb]{0.8,0.8,0.8}}c|}{\bf Especificación}\\
\hline
\bf Precondiciones &\multicolumn{3}{p{0.675\textwidth}|}
{Tener la ficha técnica del documento específico desplegada en pantalla y el usuario debe haberse logueado en el sistema.} \\
\hline
\hline
\bf Poscondicion &\multicolumn{3}{p{0.675\textwidth}|}
{Un cuadro de dialogo solicitando dirección donde se guardará el documento, una notificación de descarga y el usuario con una copia del documento. } \\
\hline
\bf Criterios de Aceptación &\multicolumn{3}{p{0.675\textwidth}|}
{El criterio es aceptado cuando se descarguen correctamente los documentos en la ruta indicada por el usuario registrado.} \\
\hline
\end{longtable}
\end{center}
%\end{document} %comentar para inlcuir
                %%\pagebreak
				
				%\documentclass[]{article}
%\usepackage[spanish]{babel}
%\usepackage[utf8]{inputenc}
%\usepackage{geometry}
%\usepackage{colortbl}
%\usepackage{longtable}
%\geometry{tmargin=3cm,bmargin=3cm,lmargin=3cm,rmargin=2cm}
%\begin{document}
%para incluir comentar hasta acá
\begin{center}
\begin{longtable}{|p{0.225\textwidth}|p{0.225\textwidth}|p{0.225\textwidth}|p{0.225\textwidth}|}
\hline
\multicolumn{2}{|p{0.45\textwidth}|}{{\bf {Descripción del requerimiento:}}
Listar nombre de documento y su correspondiente autor como resultado de una consulta. } & {\bf{ Estado}} & Análisis \\
\hline
\bf {Creado por} & María Andrea Cruz & \bf {Actualizado por} & María Andrea \\
\hline
\bf {Fecha Creación } & Marzo 31 2011 & \bf {Fecha de Actualización }& Abril  28 2011\\
\hline
\multicolumn{2}{|p{0.45\textwidth}}{\bf Identificador} &
\multicolumn{2}{|p{0.45\textwidth}|}{R24} \\
\hline
\multicolumn{2}{|p{0.45\textwidth}}{\bf {Tipo de requerimiento}} &
\multicolumn{2}{|p{0.45\textwidth}|}{Funcional}\\
\hline
\bf Descripción &\multicolumn{3}{p{0.675\textwidth}|}
{El sistema debe desplegar una lista con los resultados de la consulta, esta lista debe tener el nombre del documento y el nombre del autor correspondiente.} \\
\hline
\bf Datos de salida &\multicolumn{3}{p{0.675\textwidth}|}
{Una lista cuyos items tienen nombre de documento y nombre de autor.} \\
\hline
\bf Resultados esperados &\multicolumn{3}{p{0.675\textwidth}|}
{El sistema deberá despliega la lista con las condiciones mencionadas con anterioridad, el estado del sistema no se ve afectado} \\
\hline
\bf Origen &\multicolumn{3}{p{0.675\textwidth}|}
{Documento de descripción del problema.} \\
\hline
\bf Dirigido a &\multicolumn{3}{p{0.675\textwidth}|}
{Usuarios registrados normales,  catalogador, administrador y usuarios no registrados,.} \\
\hline
\bf Prioridad &\multicolumn{3}{p{0.675\textwidth}|}{3} \\
\hline
\bf Requerimientos Asociados &\multicolumn{3}{p{0.675\textwidth}|}
{\begin{itemize}
\item R21
\item R22
\end{itemize}} \\
\hline
\multicolumn{4}{|>{\columncolor[rgb]{0.8,0.8,0.8}}c|}{\bf Especificación}\\
\hline
\bf Precondiciones &\multicolumn{3}{p{0.675\textwidth}|}
{Haber realizado una consulta.} \\
\hline
\bf Poscondicion &\multicolumn{3}{p{0.675\textwidth}|}
{Una lista con items para escoger el documento que se crea conveniente. } \\
\hline
\bf Criterios de Aceptación &\multicolumn{3}{p{0.675\textwidth}|}
{Lista con nombre de documento y su autor.} \\
\hline
\end{longtable}
\end{center}
%\end{document} %comentar para inlcuir
                %\%pagebreak
				
				%\documentclass[]{article}
%\usepackage[spanish]{babel}
%\usepackage[utf8]{inputenc}
%\usepackage{geometry}
%\usepackage{colortbl}
%\usepackage{longtable}
%\geometry{tmargin=3cm,bmargin=3cm,lmargin=3cm,rmargin=2cm}
%\begin{document}
%para incluir comentar hasta acá
\begin{center}
\begin{longtable}{|p{0.225\textwidth}|p{0.225\textwidth}|p{0.225\textwidth}|p{0.225\textwidth}|}
\hline
\multicolumn{2}{|p{0.45\textwidth}|}{{\bf {Función del requerimiento:}}
Mostrar ficha técnica de un documento. } & {\bf{ Estado}} & Análisis \\
\hline
\multicolumn{2}{|p{0.45\textwidth}}{\bf Identificador} &
\multicolumn{2}{|p{0.45\textwidth}|}{R25} \\
\hline
\multicolumn{2}{|p{0.45\textwidth}}{\bf {Tipo de requerimiento}} &
\multicolumn{2}{|p{0.45\textwidth}|}{Funcional}\\
\hline
\bf {Creado por} & Maria Andrea Cruz & \bf {Fecha} & Marzo 31 2011\\
\hline
\bf {Actualizado por} & María Andrea Cruz & \bf {Fecha }& Abril 28 2011\\
\hline
\bf {Actualizado por} & Cristian Ríos  & \bf {Fecha }& Mayo 10 2011\\

\hline
\bf Descripción &\multicolumn{3}{p{0.675\textwidth}|}
{El sistema debe permitir mostrar la ficha técnica de un documento, esta debe tener todos los datos relacionados con el documento estos son: tipo de material, título principal, título secundario y/o traducido, editorial, fecha de publicación, Idioma, derechos de autor, resumen, autor, palabras clave, áreas a la que pertenece, formato del archivo.} \\
\hline
\bf Datos de entrada &\multicolumn{3}{p{0.675\textwidth}|}{
El usuario al solicitar una búsqueda se genera una lista de documento autor con los posibles documentos encontrados, cuando uno de los elementos de la lista es seleccionado esto proporcionará el identificador del documento al sistema con lo que será posible crear la ficha técnica del documento.}\\
\hline
\bf Datos de salida &\multicolumn{3}{p{0.675\textwidth}|}
{El sistema muestra en pantalla el conjunto de datos correspondientes al documento.} \\
\hline
\bf Resultados esperados &\multicolumn{3}{p{0.675\textwidth}|}
{El sistema deberá desplegar la ficha técnica del documento, consultando a la base de datos todos los datos relacionados al documento. El desplegar la ficha téctica se toma como consulta, el sistema actualizará la base de datos para llevar este registro} \\
\hline
\bf Origen &\multicolumn{3}{p{0.675\textwidth}|}
{Documento de descripción del problema.} \\
\hline
\bf Dirigido a &\multicolumn{3}{p{0.675\textwidth}|}
{Usuarios registrados normales, catalogador y administrador y usuarios no registrados,.} \\
\hline
\bf Prioridad &\multicolumn{3}{p{0.675\textwidth}|}{3} \\
\hline
\bf Requerimientos Asociados &\multicolumn{3}{p{0.675\textwidth}|}
{\begin{itemize}
\item R21
\item R22
\item R24
\item R10
\end{itemize}} \\
\hline
\multicolumn{4}{|>{\columncolor[rgb]{0.8,0.8,0.8}}c|}{\bf Especificación}\\
\hline
\bf Precondiciones &\multicolumn{3}{p{0.675\textwidth}|}
{Haber seleccionado un documento de la lista de documentos que despliega el resultado de una consulta.} \\
\hline
\bf Poscondicion &\multicolumn{3}{p{0.675\textwidth}|}
{Ficha técnica del documento desplegada en pantalla. } \\
\hline
\bf Criterios de Aceptación &\multicolumn{3}{p{0.675\textwidth}|}
{El requerimiento es aceptado si el usuario ve en pantalla una ficha técnica con datos del documento seleccionado.} \\
\hline
\end{longtable}
\end{center}
%\end{document} %comentar para inlcuir
                %\%pagebreak
				
				\input{requerimientos/R26.tex}
                %\%pagebreak
				
				%\documentclass[]{article}
%\usepackage[spanish]{babel}
%\usepackage[utf8]{inputenc}
%\usepackage{geometry}
%\usepackage{colortbl}
%\usepackage{longtable}
%\geometry{tmargin=3cm,bmargin=3cm,lmargin=3cm,rmargin=2cm}
%\begin{document}
%para incluir comentar hasta acá
\begin{center}
\begin{longtable}{|p{0.225\textwidth}|p{0.225\textwidth}|p{0.225\textwidth}|p{0.225\textwidth}|}
\hline
\multicolumn{2}{|p{0.45\textwidth}|}{{\bf {Descripción del requerimiento:}}
Generar reportes en formato PDF de los documentos descargados por área. } & {\bf{ Estado}} & Análisis \\
\hline
\bf {Creado por} & Maria Andrea Cruz & \bf {Actualizado por} & María Andrea Cruz\\
\hline
\bf {Fecha Creación } & Marzo 31 2011 & \bf {Fecha de Actualización }& Abril 28 2011\\
\hline
\multicolumn{2}{|p{0.45\textwidth}}{\bf Identificador} &
\multicolumn{2}{|p{0.45\textwidth}|}{R27} \\
\hline
\multicolumn{2}{|p{0.45\textwidth}}{\bf {Tipo de requerimiento}} &
\multicolumn{2}{|p{0.45\textwidth}|}{Funcional}\\
\hline
\bf Descripción &\multicolumn{3}{p{0.675\textwidth}|}
{El sistema debe proveer una interfaz para que el usuario con el perfil administrador pueda generar reportes en formato PDF de los documentos que han sido descargados por área.} \\
\hline
\bf Datos de salida &\multicolumn{3}{p{0.675\textwidth}|}
{El sistema pide los parámetros para generar el reporte, el usuario debe elegir documentos descargados ordenados por área, entonces, se generará un cuadro de dialogo pidiendo la dirección donde se guardará el reporte, y otro cuadro de dialogo notificando la entrega del reporte.} \\
\hline
\bf Resultados esperados &\multicolumn{3}{p{0.675\textwidth}|}
{El sistema deberá consultar a la base de datos todas las descargas agrupándolas por área,y organizando los resultados en un plantilla para generar el reporte en formato PDF.} \\
\hline
\bf Origen &\multicolumn{3}{p{0.675\textwidth}|}
{Documento de descripción del problema.} \\
\hline
\bf Dirigido a &\multicolumn{3}{p{0.675\textwidth}|}
{Administrador} \\
\hline
\bf Prioridad &\multicolumn{3}{p{0.675\textwidth}|}{3} \\
\hline
\bf Requerimientos Asociados &\multicolumn{3}{p{0.675\textwidth}|}
{\begin{itemize}
\item R23
\item R11
\end{itemize}} \\
\hline
\multicolumn{4}{|>{\columncolor[rgb]{0.8,0.8,0.8}}c|}{\bf Especificación}\\
\hline
\bf Precondiciones &\multicolumn{3}{p{0.675\textwidth}|}
{Haberse logueado como administrador.} \\
\hline
\bf Poscondicion &\multicolumn{3}{p{0.675\textwidth}|}
{Un archivo PDF que corresponde al informe elaborado por el sistema, donde organizado en tablas se muestran los documentos descargados por área.} \\
\hline
\bf Criterios de Aceptación &\multicolumn{3}{p{0.675\textwidth}|}
{El requerimiento es aceptado si se genera el reporte en formato PDF con la información pedida.} \\
\hline
\end{longtable}
\end{center}
%\end{document} %comentar para inlcuir
                %\%pagebreak
				
				\input{requerimientos/R28.tex}
                %%\pagebreak
				
				%\documentclass[]{article}
%\usepackage[spanish]{babel}
%\usepackage[utf8]{inputenc}
%\usepackage{geometry}
%\usepackage{colortbl}
%\usepackage{longtable}
%\geometry{tmargin=3cm,bmargin=3cm,lmargin=3cm,rmargin=2cm}
%\begin{document}
%para incluir comentar hasta acá
\begin{center}
\begin{longtable}{|p{0.225\textwidth}|p{0.225\textwidth}|p{0.225\textwidth}|p{0.225\textwidth}|}
\hline
\multicolumn{2}{|p{0.45\textwidth}|}{{\bf {Funcción del requerimiento:}}
Generar reportes en formato PDF de los documentos existentes por autor. } & {\bf{ Estado}} & Análisis \\
\hline
\multicolumn{2}{|p{0.45\textwidth}}{\bf Identificador} &
\multicolumn{2}{|p{0.45\textwidth}|}{R29} \\
\hline
\multicolumn{2}{|p{0.45\textwidth}}{\bf {Tipo de requerimiento}} &
\multicolumn{2}{|p{0.45\textwidth}|}{Funcional}\\
\hline
\bf {Creado por} & Maria Andrea Cruz & \bf {Fecha  } & Marzo 31 2011\\
\hline
\bf {Actualizado por} & María Andrea Cruz & \bf {Fecha  }&  Abril 28 2011\\
\hline
\bf {Actualizado por} & Cristian Ríos & \bf {Fecha  }&  Mayo 10 2011\\
\hline
\bf Descripción &\multicolumn{3}{p{0.675\textwidth}|}
{El sistema debe proveer una interfaz para que el usuario con  perfil administrador pueda generar reportes en formato PDF de los documentos existentes por autor.} \\
\hline
\bf Datos de entrada &\multicolumn{3}{p{0.675\textwidth}|}{
El usuario administrador que esté solicitando el reporte debe de proporcionar el nombre del reporte al momento en que el sistema pide los parámetros para generar el reporte, en este caso el usuario deberá elegir 'documentos existentes ordenados por autor'. Además, una vez se haya generado el reporte debera de proporcionar la dirección donde se guardará el reporte.}\\
\hline
\bf Datos de salida &\multicolumn{3}{p{0.675\textwidth}|}
{El sistema generará y mostrará un cuadro de dialogo notificando la generación y entrega del reporte.} \\
\hline
\bf Resultados esperados &\multicolumn{3}{p{0.675\textwidth}|}
{El sistema deberá consultar a la base de datos todos los documentos agrupándolos por autor, y organizar los resultados en una plantilla para generar el reporte en formato PDF.} \\
\hline
\bf Origen &\multicolumn{3}{p{0.675\textwidth}|}
{Documento de descripción del problema.} \\
\hline
\bf Dirigido a &\multicolumn{3}{p{0.675\textwidth}|}
{Administrador} \\
\hline
\bf Prioridad &\multicolumn{3}{p{0.675\textwidth}|}{3} \\
\hline
\bf Requerimientos Asociados &\multicolumn{3}{p{0.675\textwidth}|}
{\begin{itemize}
\item R07
\item R08
\item R09
\item R14
\end{itemize}} \\
\hline
\multicolumn{4}{|>{\columncolor[rgb]{0.8,0.8,0.8}}c|}{\bf Especificación}\\
\hline
\bf Precondiciones &\multicolumn{3}{p{0.675\textwidth}|}
{Haberse logueado como administrador.} \\
\hline
\bf Poscondicion &\multicolumn{3}{p{0.675\textwidth}|}
{Un archivo PDF que corresponde al informe elaborado por el sistema, donde organizado en tablas se muestran los documentos existentes por autor.} \\
\hline
\bf Criterios de Aceptación &\multicolumn{3}{p{0.675\textwidth}|}
{El requerimiento es aceptado si se genera el reporte en formato PDF con la información solicitada.} \\
\hline
\end{longtable}
\end{center}
%\end{document} %comentar para inlcuir
                %\%pagebreak
				
				%\documentclass[]{article}
%\usepackage[spanish]{babel}
%\usepackage[utf8]{inputenc}
%\usepackage{geometry}
%\usepackage{colortbl}
%\usepackage{longtable}
%\geometry{tmargin=3cm,bmargin=3cm,lmargin=3cm,rmargin=2cm}
%\begin{document}
%para incluir comentar hasta acá
\begin{center}
\begin{longtable}{|p{0.225\textwidth}|p{0.225\textwidth}|p{0.225\textwidth}|p{0.225\textwidth}|}
\hline
\multicolumn{2}{|p{0.45\textwidth}|}{{\bf {Función del requerimiento:}}
Generar reportes en formato PDF del total de usuarios registrados. } & {\bf{ Estado}} & Análisis \\
\hline
\multicolumn{2}{|p{0.45\textwidth}}{\bf Identificador} &
\multicolumn{2}{|p{0.45\textwidth}|}{R30} \\
\hline
\multicolumn{2}{|p{0.45\textwidth}}{\bf {Tipo de requerimiento}} &
\multicolumn{2}{|p{0.45\textwidth}|}{Funcional}\\
\hline
\bf {Creado por} & María Andrea Cruz & \bf {Fecha  } & Marzo 31 2011\\
\hline
\bf {Actualizado por} & María Andrea Cruz & \bf {Fecha  }& Abril 28 2011\\
\hline
\bf {Actualizado por} & Cristian Ríos & \bf {Fecha }& Mayo 10 2011\\

\hline
\bf Descripción &\multicolumn{3}{p{0.675\textwidth}|}
{El sistema debe proveer una interfaz para que el usuario con perfil administrador pueda generar reportes en formato PDF de todos los usuarios registrados en el sistema.} \\
\hline
\bf Datos de entrada &\multicolumn{3}{p{0.675\textwidth}|}{
El usuario administrador que esté solicitando el reporte debe de proporcionar el nombre del reporte al momento en que el sistema pide los parámetros para generar el reporte, en este caso el usuario deberá elegir 'total de usuarios registrados'. Además, una vez se haya generado el reporte deberá de proporcionar la dirección donde se guardará el reporte.}\\
\hline
\bf Datos de salida &\multicolumn{3}{p{0.675\textwidth}|}
{El sistema generará y mostrará un cuadro de dialogo notificando la generación y entrega del reporte.} \\
\hline
\bf Resultados esperados &\multicolumn{3}{p{0.675\textwidth}|}
{El sistema deberá consulta a la base de datos todos los usuarios registrados, y organizando los resultados por nombre en una plantilla generar el reporte en formato PDF.} \\
\hline
\bf Origen &\multicolumn{3}{p{0.675\textwidth}|}
{Documento de descripción del problema.} \\
\hline
\bf Dirigido a &\multicolumn{3}{p{0.675\textwidth}|}
{Administrador} \\
\hline
\bf Prioridad &\multicolumn{3}{p{0.675\textwidth}|}{3} \\
\hline
\bf Requerimientos Asociados &\multicolumn{3}{p{0.675\textwidth}|}
{\begin{itemize}
\item R01
\item R02
\end{itemize}} \\
\hline
\multicolumn{4}{|>{\columncolor[rgb]{0.8,0.8,0.8}}c|}{\bf Especificación}\\
\hline
\bf Precondiciones &\multicolumn{3}{p{0.675\textwidth}|}
{Haberse logueado como administrador.} \\
\hline
\bf Poscondicion &\multicolumn{3}{p{0.675\textwidth}|}
{Un archivo PDF que corresponde al informe elaborado por el sistema, onde organizado en tablas se muestran la información de todos los usuarios registrados, esto es: login, nombre1, apellido1, email, vinculo con Univalle, género, fecha de registro, fecha de nacimiento, fecha de registro, tipo de usuario y estado.} \\
\hline
\bf Criterios de Aceptación &\multicolumn{3}{p{0.675\textwidth}|}
{El requerimiento es aceptado si se genera el reporte en formato PDF y  con la información pedida.} \\
\hline
\end{longtable}
\end{center}
%\end{document} %comentar para inlcuir
                %\%pagebreak
				
				\input{requerimientos/R31.tex}
                %\%pagebreak
				
				\input{requerimientos/R32.tex}
                %\%pagebreak
				
				%\documentclass[]{article}
%\usepackage[spanish]{babel}
%\usepackage[utf8]{inputenc}
%\usepackage{geometry}
%\usepackage{colortbl}
%\usepackage{longtable}
%\geometry{tmargin=3cm,bmargin=3cm,lmargin=3cm,rmargin=2cm}
%\begin{document}
%para incluir comentar hasta acá
\begin{center}
\begin{longtable}{|p{0.225\textwidth}|p{0.225\textwidth}|p{0.225\textwidth}|p{0.225\textwidth}|}
\hline
\multicolumn{2}{|p{0.45\textwidth}|}{{\bf {Función del requerimiento:}}
Generar reportes en formato PDF de las consultas por área. } & {\bf{ Estado}} & Análisis \\
\hline
\bf {Creado por} & Maria Andrea Cruz & \bf {Actualizado por} & María Andrea \\
\hline
\bf {Fecha Creación } & Marzo 31 2011 & \bf {Fecha de Actualización }& 
Abril 28 2011\\
Mayo 10 2011\\
\hline
\multicolumn{2}{|p{0.45\textwidth}}{\bf Identificador} &
\multicolumn{2}{|p{0.45\textwidth}|}{R33} \\
\hline
\multicolumn{2}{|p{0.45\textwidth}}{\bf {Tipo de requerimiento}} &
\multicolumn{2}{|p{0.45\textwidth}|}{Funcional}\\
\hline
\bf Descripción &\multicolumn{3}{p{0.675\textwidth}|}
{El sistema debe proveer una interfaz para que el usuario con perfil administrador pueda generar reportes en formato PDF de todas las consultas de documentos por área.} \\
\hline
\bf Datos de entrada &\multicolumn{3}{p{0.675\textwidth}|}{
El usuario administrador que esté solicitando el reporte debe de proporcionar el nombre del reporte al momento en que el sistema pide los parámetros para generar el reporte, en este caso el usuario deberá elegir 'consultas a documentos ordenadas por área'. Además, una vez se haya generado el reporte debera de proporcionar la dirección donde se guardará el reporte.}\\
\hline
\bf Datos de salida &\multicolumn{3}{p{0.675\textwidth}|}
{El sistema generará y mostrará un cuadro de dialogo notificando la generación y entrega del reporte.} \\
\hline
\bf Resultados esperados &\multicolumn{3}{p{0.675\textwidth}|}
{El sistema deberá consultar a la base de datos todas las consultas de docuemtnos realizadas agrupándolas por área, y organizando los resultados en un plantilla para generar el reporte en formato PDF.} \\
\hline
\bf Origen &\multicolumn{3}{p{0.675\textwidth}|}
{Documento de descripción del problema.} \\
\hline
\bf Dirigido a &\multicolumn{3}{p{0.675\textwidth}|}
{Administrador} \\
\hline
\bf Prioridad &\multicolumn{3}{p{0.675\textwidth}|}{3} \\
\hline
\bf Requerimientos Asociados &\multicolumn{3}{p{0.675\textwidth}|}
{\begin{itemize}
\item R21
\item R22
\item R10
\end{itemize}} \\
\hline
\multicolumn{4}{|>{\columncolor[rgb]{0.8,0.8,0.8}}c|}{\bf Especificación}\\
\hline
\bf Precondiciones &\multicolumn{3}{p{0.675\textwidth}|}
{Haberse logueado como administrador.} \\
\hline
\bf Poscondicion &\multicolumn{3}{p{0.675\textwidth}|}
{Un archivo PDF que corresponde al informe elaborado por el sistema, donde organizado en tablas se muestran las consultas realizdas por área.} \\
\hline
\bf Criterios de Aceptación &\multicolumn{3}{p{0.675\textwidth}|}
{El requerimiento es aceptado si se genera el reporte en formato PDF y  con la información solicitada.} \\
\hline
\end{longtable}
\end{center}
%\end{document} %comentar para inlcuir
                \pagebreak
				
				%\documentclass[]{article}
%\usepackage[spanish]{babel}
%\usepackage[utf8]{inputenc}
%\usepackage{geometry}
%\usepackage{colortbl}
%\usepackage{longtable}
%\geometry{tmargin=3cm,bmargin=3cm,lmargin=3cm,rmargin=2cm}
%\begin{document}
%para incluir comentar hasta acá
\begin{center}
\begin{longtable}{|p{0.225\textwidth}|p{0.225\textwidth}|p{0.225\textwidth}|p{0.225\textwidth}|}
\hline
\multicolumn{2}{|p{0.45\textwidth}|}{{\bf {Función del requerimiento:}}
Generar reportes en formato PDF de los documentos catalogados por fecha. } & {\bf{ Estado}} & Análisis \\
\hline
\multicolumn{2}{|p{0.45\textwidth}}{\bf Identificador} &
\multicolumn{2}{|p{0.45\textwidth}|}{R34} \\
\hline
\multicolumn{2}{|p{0.45\textwidth}}{\bf {Tipo de requerimiento}} &
\multicolumn{2}{|p{0.45\textwidth}|}{Funcional}\\
\hline
\bf {Creado por} & Maria Andrea Cruz & \bf {Fecha } & Marzo 31 2011\\
\hline
\bf {Actualizado por} & María Andrea  Cruz & \bf {Fecha  }& Abril 28 2011\\
\hline
\bf {Actualizado por} & Cristian Ríos & \bf {Fecha  }& Mayo 10 2011\\


\hline
\bf Descripción &\multicolumn{3}{p{0.675\textwidth}|}
{El sistema debe proveer una interfaz para que el usuario con el perfil administrador pueda generar reportes en formato PDF de los documentos catalogados por fecha.} \\
\hline
\bf Datos de entrada &\multicolumn{3}{p{0.675\textwidth}|}{
El usuario administrador que esté solicitando el reporte debe de proporcionar el nombre del reporte al momento en que el sistema pide los parámetros para generar el reporte, en este caso el usuario deberá elegir 'documentos catalogados ordenados por fecha'. Además, una vez se haya generado el reporte debera de proporcionar la dirección donde se guardará el reporte.}\\
\hline
\bf Datos de salida &\multicolumn{3}{p{0.675\textwidth}|}
{El sistema generará y mostrará un cuadro de dialogo notificando la generación y entrega del reporte.} \\
\hline
\bf Resultados esperados &\multicolumn{3}{p{0.675\textwidth}|}
{El sistema deberá consultar a la base de datos todos los documentos catalogados agrupándolos por fecha, y organizando los resultados en un plantilla para generar el reporte en formato PDF.} \\
\hline
\bf Origen &\multicolumn{3}{p{0.675\textwidth}|}
{Documento de descripción del problema.} \\
\hline
\bf Dirigido a &\multicolumn{3}{p{0.675\textwidth}|}
{Administrador.} \\
\hline
\bf Prioridad &\multicolumn{3}{p{0.675\textwidth}|}{3} \\
\hline
\bf Requerimientos Asociados &\multicolumn{3}{p{0.675\textwidth}|}
{\begin{itemize}
\item R08
\item R09
\item R14
\item R16
\item R18
\item R20
\end{itemize}} \\
\hline
\multicolumn{4}{|>{\columncolor[rgb]{0.8,0.8,0.8}}c|}{\bf Especificación}\\
\hline
\bf Precondiciones &\multicolumn{3}{p{0.675\textwidth}|}
{Haberse logueado como administrador.} \\
\hline
\bf Poscondicion &\multicolumn{3}{p{0.675\textwidth}|}
{Un archivo PDF que corresponde al informe elaborado por el sistema, onde organizado en tablas se muestran los documentos catalogados por fecha.} \\
\hline
\bf Criterios de Aceptación &\multicolumn{3}{p{0.675\textwidth}|}
{El requerimiento es aceptado si se genera el reporte en formato PDF y  con la información solicitada.} \\
\hline
\end{longtable}
\end{center}
%\end{document} %comentar para inlcuir
                %\%pagebreak
				
				\input{requerimientos/R35.tex}
                %\%pagebreak
				
				%\documentclass[]{article}
%\usepackage[spanish]{babel}
%\usepackage[utf8]{inputenc}
%\usepackage{geometry}
%\usepackage{colortbl}
%\usepackage{longtable}
%\geometry{tmargin=3cm,bmargin=3cm,lmargin=3cm,rmargin=2cm}
%\begin{document}
%para incluir comentar hasta acá
\begin{center}
\begin{longtable}{|p{0.225\textwidth}|p{0.225\textwidth}|p{0.225\textwidth}|p{0.225\textwidth}|}
\hline
\multicolumn{2}{|p{0.45\textwidth}|}{{\bf {Descripción del requerimiento:}}
Generar reportes en formato PDF de los documentos existentes por tipo de documento. } & {\bf{ Estado}} & Análisis \\
\hline
\bf {Creado por} & Maria Andrea Cruz & \bf {Actualizado por} & María Andrea Cruz\\
\hline
\bf {Fecha Creación } & Marzo 31 2011 & \bf {Fecha de Actualización }& Abril 28 2011\\
\hline
\multicolumn{2}{|p{0.45\textwidth}}{\bf Identificador} &
\multicolumn{2}{|p{0.45\textwidth}|}{R36} \\
\hline
\multicolumn{2}{|p{0.45\textwidth}}{\bf {Tipo de requerimiento}} &
\multicolumn{2}{|p{0.45\textwidth}|}{Funcional}\\
\hline
\bf Descripción &\multicolumn{3}{p{0.675\textwidth}|}
{El sistema debe proveer una interfaz para que el usuario con perfil administrador pueda generar reportes en formato PDF de los documentos existentes por tipo de documento.} \\
\hline
\bf Datos de salida &\multicolumn{3}{p{0.675\textwidth}|}
{El sistema pide los parámetros para generar el reporte, el usuario debe elegir documentos existentes ordenados por tipo de documento, entonces se muestra un cuadro de dialogo pidiendo la dirección donde se guardará el reporte, y otro cuadro de dialogo notificando la entrega del reporte.} \\
\hline
\bf Resultados esperados &\multicolumn{3}{p{0.675\textwidth}|}
{El sistema se ve estimulado y consulta a la base de datos todos los documentos existentes agrupándolos por tipo de documento, y organizando los resultados en un plantilla para generar el reporte en formato PDF.} \\
\hline
\bf Origen &\multicolumn{3}{p{0.675\textwidth}|}
{Documento de descripción del problema.} \\
\hline
\bf Dirigido a &\multicolumn{3}{p{0.675\textwidth}|}
{Administrador.} \\
\hline
\bf Prioridad &\multicolumn{3}{p{0.675\textwidth}|}{3} \\
\hline
\bf Requerimientos Asociados &\multicolumn{3}{p{0.675\textwidth}|}
{\begin{itemize}
\item R07
\item R08
\item R09
\end{itemize}} \\
\hline
\multicolumn{4}{|>{\columncolor[rgb]{0.8,0.8,0.8}}c|}{\bf Especificación}\\
\hline
\bf Precondiciones &\multicolumn{3}{p{0.675\textwidth}|}
{Haberse logueado como administrador.} \\
\hline
\bf Poscondicion &\multicolumn{3}{p{0.675\textwidth}|}
{Un archivo PDF que corresponde al informe elaborado por el sistema, donde organizado en tablas se muestran los documentos existentes por tipo de documento.} \\
\hline
\bf Criterios de Aceptación &\multicolumn{3}{p{0.675\textwidth}|}
{El requerimiento es aceptado si se genera el reporte en formato PDF y  con la información solicitada.} \\
\hline
\end{longtable}
\end{center}
%\end{document} %comentar para inlcuir
                %\%pagebreak

				\input{requerimientos/R37.tex}
                %\%pagebreak        
				
				%\documentclass[]{article}
%\usepackage[spanish]{babel}
%\usepackage[utf8]{inputenc}
%\usepackage{geometry}
%\usepackage{colortbl}
%\usepackage{longtable}
%\geometry{tmargin=3cm,bmargin=3cm,lmargin=3cm,rmargin=2cm}
%\begin{document} %comentar hasta esta linea para incluir
\begin{center}
\begin{longtable}{|p{0.225\textwidth}|p{0.225\textwidth}|p{0.225\textwidth}|p{0.225\textwidth}|}
\hline
\multicolumn{2}{|p{0.45\textwidth}|}{{\bf {Función del requerimiento:}}
Permitir el logout del sistema. } & {\bf{ Estado}} & Análisis \\
\hline
\multicolumn{2}{|p{0.45\textwidth}}{\bf Identificador} &
\multicolumn{2}{|p{0.45\textwidth}|}{R38} \\
\hline
\multicolumn{2}{|p{0.45\textwidth}}{\bf {Tipo de requerimiento}} &
\multicolumn{2}{|p{0.45\textwidth}|}{Funcional}\\
\hline
\bf {Creado por} & María Andrea  Cruz& \bf {Fecha  } & Abril  28 2011\\
\hline
\bf {Actualizado por} & María Andrea Cruz & \bf {Fecha }& Abril 28 2011\\
\hline
\bf {Actualizado por} & Cristian Ríos & \bf {Fecha }& Mayo 10 2011\\


\hline
\bf Descripción &\multicolumn{3}{p{0.675\textwidth}|}
{El sistema debe permitirle a un usuario que halla hecho login, hacer  logout cuanto termine su sesión.} \\
\hline
\bf Datos de entrada &\multicolumn{3}{p{0.675\textwidth}|}{
Para salir del sistema no se debe de proporcionar ningún dato.}\\
\hline
\bf Datos de salida &\multicolumn{3}{p{0.675\textwidth}|}
{El sistema mostrará la interfaz inicial, es decir la de usuario no registrado.} \\
\hline
\bf Resultados esperados &\multicolumn{3}{p{0.675\textwidth}|}
{El sistema deberá dar por terminada la sesión del usuario que estaba logueado.} \\
\hline
\bf Origen &\multicolumn{3}{p{0.675\textwidth}|}
{Desarrolladores} \\
\hline
\bf Dirigido a &\multicolumn{3}{p{0.675\textwidth}|}
{Usuarios registrados del sistema, esto es usuarios normales, catalogadores y administradores.} \\
\hline
\bf Prioridad &\multicolumn{3}{p{0.675\textwidth}|}{3} \\
\hline
\bf Requerimientos Asociados &\multicolumn{3}{p{0.675\textwidth}|}
{\begin{itemize}
\item R01
\end{itemize}} \\
\hline
\multicolumn{4}{|>{\columncolor[rgb]{0.8,0.8,0.8}}c|}{\bf Especificación}\\
\hline
\bf Precondiciones &\multicolumn{3}{p{0.675\textwidth}|}
{El usuario debe haberse logueado en el sistema.} \\
\hline
\hline
\bf Poscondicion &\multicolumn{3}{p{0.675\textwidth}|}
{Ee cerrará la sesión del usuario que estaba logueado.} \\
\hline
\bf Criterios de Aceptación &\multicolumn{3}{p{0.675\textwidth}|}
{El requerimiento se acepta cuando cualquier usuario registrado que se encuentre logueado en el sistema pueda hacer logout de este} \\
\hline
\end{longtable}
\end{center}
%\end{document}
                %\%pagebreak
          %***********************************************************************

%\end{document}