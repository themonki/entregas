%\documentclass[]{article}
%\usepackage[spanish]{babel}
%\usepackage[utf8]{inputenc}
%\usepackage{geometry}
%\usepackage{colortbl}
%\usepackage{longtable}
%\usepackage{graphicx}
%\geometry{tmargin=3cm,bmargin=3cm,lmargin=3cm,rmargin=2cm}
%
%\begin{document}
\begin{center}
\begin{longtable}{|p{0.225\textwidth}|p{0.225\textwidth}|p{0.225\textwidth}|p{0.225\textwidth}|}
\hline
{\bf {Empresa:}} &
\multicolumn{2}{p{0.45\textwidth}|} { Escuela de Ingeniería de Sistemas y Computación } &
{\includegraphics[width=80.5pt]{LOGO}} \\
\hline
\bf {Nombre del caso de uso:}&\multicolumn{3}{l|}{
Ingresar tipos de material.
} \\
\hline
\bf Código: & 
CU07 &\bf Fecha: & 
Mayo 04 2011 \\
\hline
\bf Autor(es ): & 
Yerminson Gonzalez & 
 & 
 \\
\hline
\bf Descripción: &\multicolumn{3}{p{0.675\textwidth}|}
{
Permite la creación de nuevos tipos de material en caso de que se requiera para registro de un documento y que den información sobre la estructura del documento.
} \\
\hline
\bf Actores: &\multicolumn{3}{p{0.675\textwidth}|}{
Administrador y Catalogador. 
} \\
\hline
\bf Precondiciones: &\multicolumn{3}{p{0.675\textwidth}|}
{
Haber ingresado al sistema y tener perfil de catalogador o administrador.
} \\
\hline
\multicolumn{4}{|c|}{\bf {Flujo Normal}}\\
\hline
\multicolumn{2}{|c}{\bf Actor} & \multicolumn{2}{|c|}{\bf Sistema } \\
\hline
\multicolumn{2}{|p{0.45\textwidth}}
{
\begin{itemize}
\item[1. ]El caso de uso inicia cuando el catalogador solicita crear un tipo de material para algún documento.
\item[3.] El usuario ingresa datos en los campos proporcionado por la interfaz del sistema para creación de nuevos tipos de material.
\item[4. ]El usuario indica al sistema que ya a ingresado los datos y que desea crear el nuevo tipo de material.
\item[7. ]El usuario acepta el mensaje de confirmación generado por el sistema y el caso de uso finaliza.
\end{itemize}
} &
\multicolumn{2}{|p{0.45\textwidth}|}
{
\begin{itemize}
\item[2.] El sistema responde mostrando una interfaz con dos campos: un campo para indicar el nombre y un campo para indicar una descripción del tipo.
\item[5.] El sistema valida que el nombre del tipo de material que ha ingresado el usuario para el nuevo tipo de material no exista como nombre de otro tipo de material.
\item[6. ]El sistema crea un nuevo tipo de material para documentos de ciencias de la computación que serán almacenados en el sistema y responde con un mensaje al usuario indicando el éxito de la operación.
\end{itemize}
} \\
\hline
\multicolumn{4}{|c|}{\bf {Flujo Alternativo}}\\
\hline
\multicolumn{2}{|p{0.45\textwidth}}
{
\begin{itemize}
\item[7.1] El usuario acepta el mensaje de notificación del error generado por el sistema.
\end{itemize}
} &
\multicolumn{2}{|p{0.45\textwidth}|}
{
\begin{itemize}
\item[5.1] El sistema al realizar la validación del nombre se percata de que el nombre dado al nuevo tipo de material ya existe.
\item[6.1] El sistema genera un mensaje indicando que el nombre dado al tipo de material no se puede utilizar porque ya existe un tipo de material con ese nombre.
\end{itemize}
} \\
\hline
\bf Poscondiciones: &\multicolumn{3}{p{0.675\textwidth}|}
{
El sistema añade un registro correspondiente a los tipos de material a los que pueden pertenecer los documentos.
} \\
\hline
\bf Excepciones: &\multicolumn{3}{p{0.675\textwidth}|}
{
Fallo de conexionen la base de datos.	Falla en el sistema de suministro de energía.
} \\
\hline
\bf Aprobado por : & 
 & \bf Fecha & 
 \\
\hline
\end{longtable}
\end{center}
%\end{document}