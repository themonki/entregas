%\documentclass[]{article}
%\usepackage[spanish]{babel}
%\usepackage[utf8]{inputenc}
%\usepackage{geometry}
%\usepackage{colortbl}
%\usepackage{longtable}
%\geometry{tmargin=3cm,bmargin=3cm,lmargin=3cm,rmargin=2cm}
%\begin{document}
%para incluir comentar hasta acá
\begin{center}
\begin{longtable}{|p{0.225\textwidth}|p{0.225\textwidth}|p{0.225\textwidth}|p{0.225\textwidth}|}
\hline
\multicolumn{2}{|p{0.45\textwidth}|}{{\bf {Función del requerimiento:}}
Generar reportes en formato PDF de los documentos existentes por formato. } & {\bf{ Estado}} & Análisis \\
\hline
\multicolumn{2}{|p{0.45\textwidth}}{\bf Identificador} &
\multicolumn{2}{|p{0.45\textwidth}|}{R37} \\
\hline
\multicolumn{2}{|p{0.45\textwidth}}{\bf {Tipo de requerimiento}} &
\multicolumn{2}{|p{0.45\textwidth}|}{Funcional}\\
\hline
\bf {Creado por} & María Andrea Cruz & \bf {Fecha } & Marzo 31 2011\\
\hline
\bf {Actualizado por} & María Andrea Cruz & \bf {Fecha }& Abril 28 2011\\
\hline
\bf {Actualizado por} & Cristian Ríos & \bf {Fecha }& Mayo 10 2011\\

\hline
\bf Descripción &\multicolumn{3}{p{0.675\textwidth}|}
{El sistema debe proveer una interfaz para que el usuario con perfil administrador pueda generar reportes en formato PDF de los documentos existentes por formato.} \\
\hline
\bf Datos de entrada &\multicolumn{3}{p{0.675\textwidth}|}{
El usuario administrador que esté solicitando el reporte debe de proporcionar el nombre del reporte al momento en que el sistema pide los parámetros para generar el reporte, en este caso el usuario deberá elegir 'documentos existentes ordenados por formato'. Además, una vez se haya generado el reporte deberá de proporcionar la dirección donde se guardará el reporte.}\\
\hline
\bf Datos de salida &\multicolumn{3}{p{0.675\textwidth}|}
{El sistema generará y mostrará un cuadro de dialogo notificando la generación y entrega del reporte.} \\
\hline
\bf Resultados esperados &\multicolumn{3}{p{0.675\textwidth}|}
{El sistema deberá consultar a la base de datos todos los documentos existentes agrupándolos por formato, y organizando los resultados en un plantilla para generar el reporte en formato PDF.} \\
\hline
\bf Origen &\multicolumn{3}{p{0.675\textwidth}|}
{Documento de descripción del problema.} \\
\hline
\bf Dirigido a &\multicolumn{3}{p{0.675\textwidth}|}
{Administrador.} \\
\hline
\bf Prioridad &\multicolumn{3}{p{0.675\textwidth}|}{3} \\
\hline
\bf Requerimientos Asociados &\multicolumn{3}{p{0.675\textwidth}|}
{\begin{itemize}
\item R07
\item R08
\item R09
\end{itemize}} \\
\hline
\multicolumn{4}{|>{\columncolor[rgb]{0.8,0.8,0.8}}c|}{\bf Especificación}\\
\hline
\bf Precondiciones &\multicolumn{3}{p{0.675\textwidth}|}
{Haberse logueado como administrador.} \\
\hline
\bf Poscondicion &\multicolumn{3}{p{0.675\textwidth}|}
{Un archivo PDF que corresponde al informe elaborado por el sistema, donde organizado en tablas se muestran los documentos existentes por formato.} \\
\hline
\bf Criterios de Aceptación &\multicolumn{3}{p{0.675\textwidth}|}
{El requerimiento es aceptado si se genera el reporte en formato PDF con la información solicitada.} \\
\hline
\end{longtable}
\end{center}
%\end{document} %comentar para inlcuir