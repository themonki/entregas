%\documentclass[]{article}
%\usepackage[spanish]{babel}
%\usepackage[utf8]{inputenc}
%\usepackage{geometry}
%\usepackage{colortbl}
%\usepackage{longtable}
%\geometry{tmargin=3cm,bmargin=3cm,lmargin=3cm,rmargin=2cm}
%\begin{document}
%para incluir comentar hasta acá
\begin{center}
\begin{longtable}{|p{0.225\textwidth}|p{0.225\textwidth}|p{0.225\textwidth}|p{0.225\textwidth}|}
\hline
\multicolumn{2}{|p{0.45\textwidth}|}{{\bf {Descripción del requerimiento:}}
Asignar datos a documentos. } & {\bf{ Estado}} & Análisis \\
\hline
\bf {Creado por} & Maria Andrea Cruz & \bf {Actualizado por} & Maria Andrea Cruz \\
\hline
\bf {Fecha Creación } & Marzo 31 2011 & \bf {Fecha de Actualización }& Abril 28 2011\\
\hline
\multicolumn{2}{|p{0.45\textwidth}}{\bf Identificador} &
\multicolumn{2}{|p{0.45\textwidth}|}{R08} \\
\hline
\multicolumn{2}{|p{0.45\textwidth}}{\bf {Tipo de requerimiento}} &
\multicolumn{2}{|p{0.45\textwidth}|}{Funcional}\\
\hline
\bf Descripción &\multicolumn{3}{p{0.675\textwidth}|}
{El sistema debe de proporcionar la posibilidad a los usuarios que tengan como perfil catalogador o administrador de asignar datos a los documentos que se encuentran referenciados en la base de datos, estos datos son información sobre el documento como el autor, el titulo principal, el título secundario o traducido, la editorial, la fecha de publicación, la fecha de creación, el idioma en el que esta escrito, los derechos de autor, la descripción o resumen del material, las palabras claves relacionadas con este, el área de las ciencias de la computación a la que pertenece, el formato del archivo, el tamaño del archivo en bytes, la resolución en píxeles y el software recomendado para abrir el documento. Para saber que documentos existen y a saber a cuales de estos debe adicionar datos se debe realizar una consulta a la base de datos.} \\
\hline
\bf Datos de salida &\multicolumn{3}{p{0.675\textwidth}|}
{ Se informa al usuario que solicito la adición de los datos del documento si de la operación fue exitosa.} \\
\hline
\bf Resultados esperados &\multicolumn{3}{p{0.675\textwidth}|}
{ El sistema deberá realizar actualizaciones en la base de datos con nueva información actualizada y correcta del documento..} \\
\hline
\bf Origen &\multicolumn{3}{p{0.675\textwidth}|}
{Documento de descripción del problema.} \\
\hline
\bf Dirigido a &\multicolumn{3}{p{0.675\textwidth}|}
{Catalogador, administrador.} \\
\hline
\bf Prioridad &\multicolumn{3}{p{0.675\textwidth}|}{3} \\
\hline
\bf Requerimientos Asociados &\multicolumn{3}{p{0.675\textwidth}|}
{\begin{itemize}
        \item R06
        \item R07
        \item R17
        \item R14
        \item R18
\end{itemize}} \\\hline
\multicolumn{4}{|>{\columncolor[rgb]{0.8,0.8,0.8}}c|}{\bf Especificación}\\
\hline
\bf Precondiciones &\multicolumn{3}{p{0.675\textwidth}|}
{El sistema debe estar conectado a la base de datos y tener logueado a un usuario con los perfiles correspondientes a administrador o catalogador los cuales tienen acceso a una interfaz de catalogación, la cual muestra información correspondiente a los documentos y permite introducir una nueva para estos. Los autores, palabras clave y áreas de ciencias de la computación asociados con el documento deben de existir previamente en el sistema.} \\
\hline
\bf Poscondicion &\multicolumn{3}{p{0.675\textwidth}|}
{El sistema en su base de datos tiene ahora información de un documento agregado anteriormente, con esta información el documento podrá ser consultado y descargado posteriormente.} \\
\hline
\bf Criterios de Aceptación &\multicolumn{3}{p{0.675\textwidth}|}
{El requerimiento es aceptado si los usuarios con perfil administrador o catalogador puede adicionar datos a los documentos, esto es catalogar el documento.} \\
\hline
\end{longtable}
\end{center}
%\end{document} %comentar para inlcuir