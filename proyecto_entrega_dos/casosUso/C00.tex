%\documentclass[]{article}
%\usepackage[spanish]{babel}
%\usepackage[utf8]{inputenc}
%\usepackage{geometry}
%\usepackage{colortbl}
%\usepackage{longtable}
%\usepackage{graphicx}
%\geometry{tmargin=3cm,bmargin=3cm,lmargin=3cm,rmargin=2cm}
%\begin{document}
\begin{center}
\begin{longtable}{|p{0.225\textwidth}|p{0.225\textwidth}|p{0.225\textwidth}|p{0.225\textwidth}|}
\hline
{\bf {Empresa:}} &
\multicolumn{2}{p{0.45\textwidth}|} { Escuela de Ingeniería de Sistemas y Computación } &
{\includegraphics[width=80.5pt]{LOGO}} \\
\hline
\bf {Nombre del caso de uso:}&\multicolumn{3}{l|}{
Registro de Usuario.
} \\
\hline
\bf Codigo: & 
CU00 &\bf Fecha: & 
Abril 03 2011 \\
\hline
\bf Autor(es ): & 
Edgar Andrés Moncada & 
Yerminson Gonzalez & 
 \\
\hline
\bf Descripcion: &\multicolumn{3}{p{0.675\textwidth}|}
{
Permite el registro al sistema de nuevos usuarios.
} \\
\hline
\bf Actores: &\multicolumn{3}{p{0.675\textwidth}|}{
Usuario No Registrado. 
} \\
\hline
\bf Precondiciones: &\multicolumn{3}{p{0.675\textwidth}|}
{
El sistema debe de haberse iniciado.
} \\
\hline
\multicolumn{4}{|c|}{\bf {Flujo Normal}}\\
\hline
\multicolumn{2}{|c}{\bf Actor} & \multicolumn{2}{|c|}{\bf Sistema } \\
\hline
\multicolumn{2}{|p{0.45\textwidth}}
{
\begin{itemize}
\item [1.]El caso de uso inicia cuando el usuario no registrado selecciona la opción de registrarse.
\item [3.] El usuario digita los campos pedidos.
\item[4.] El usuario le indica al sistema que valide los datos.
\end{itemize}
} &
\multicolumn{2}{|p{0.45\textwidth}|}
{
\begin{itemize}
\item[2.] El sistema genera una interfaz que le pide al usuario los siguientes datos obligatorios: login, contraseña, verificación de la contraseña, pregunta y respuesta secreta, nombre. A demás pide los siguientes datos que no son obligatorios: apellidos, género, fecha de nacimiento, email, nivel de escolaridad, vinculo con Univalle (estudiante de pregrado, de posgrado, egresado, profesor activo, jubilado, ninguno). Se pide información sobre las aéreas de interés.
\item[5.] El Sistema valida que el login no exista.
\item[6.] El Sistema valida que los campos de contraseña y verificación de contraseña.
 El sistema valida los campos digitados.
\item[7.] El sistema crea una nueva instancia del usuario en la base de datos.
\item[8.] El sistema envía un mensaje de confirmación al usuario y el caso de uso termina.
\end{itemize}
} \\
\hline
\multicolumn{4}{|c|}{\bf {Flujo Alternativo}}\\
\hline
\multicolumn{2}{|p{0.45\textwidth}}
{
\begin{itemize}
\item[3.3.] El Usuario no digita ninguno de los campos obligatorios.
\item[4.3.] El Usuario le indica al Sistema que valide los datos.
\end{itemize}
} &
\multicolumn{2}{|p{0.45\textwidth}|}
{
\begin{itemize}
\item[5.1.] El Sistema al validar se da cuenta que el login digitado por el Usuario ya existe.
\item[6.1. ]El Sistema envía una notificación al usuario indicándole que ingrese un nuevo login.
\item[6.2. ]El Sistema al validar se da cuenta que el campo de contraseña y el campo de verificación de contraseña no son iguales.
\item[7.2. ]El Sistema envía una notificación al Usuario indicándole que las contraseñas no coinciden y que las vuelva a confirmar.
\item[5.3. ]El Sistema al validar se da cuenta que los campos obligatorios no tienen datos.
\item[6.3.] El Sistema envía una notificación al Usuario indicándole que vuelva a confirmar.
\end{itemize}
} \\
\hline
\bf Poscondiciones: &\multicolumn{3}{p{0.675\textwidth}|}
{
El sistema almacena los datos del usuario en la base de datos.
} \\
\hline
\bf Excepciones: &\multicolumn{3}{p{0.675\textwidth}|}
{
El sistema no puede acceder a la base de datos y no puede almacenar los datos.
} \\
\hline
\bf Aprobado por : & 
 & \bf Fecha & 
 \\
\hline
\end{longtable}
\end{center}
%\end{document}