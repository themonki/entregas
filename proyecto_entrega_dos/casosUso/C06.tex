%\documentclass[]{article}
%\usepackage[spanish]{babel}
%\usepackage[utf8]{inputenc}
%\usepackage{geometry}
%\usepackage{colortbl}
%\usepackage{longtable}
%\usepackage{graphicx}
%\geometry{tmargin=3cm,bmargin=3cm,lmargin=3cm,rmargin=2cm}
%
%\begin{document}
\begin{center}
\begin{longtable}{|p{0.225\textwidth}|p{0.225\textwidth}|p{0.225\textwidth}|p{0.225\textwidth}|}
\hline
{\bf {Empresa:}} &
\multicolumn{2}{p{0.45\textwidth}|} { Escuela de Ingeniería de Sistemas y Computación } &
{\includegraphics[width=80.5pt]{LOGO}} \\
\hline
\bf {Nombre del caso de uso:}&\multicolumn{3}{l|}{
Catalogar.
} \\
\hline
\bf Codigo: & 
CU06 &\bf Fecha: & 
Abril 02 2011 \\
\hline
\bf Autor(es ): & 
Yerminson Gonzalez & 
 & 
 \\
\hline
\bf Descripcion: &\multicolumn{3}{p{0.675\textwidth}|}
{
Permite la creación de un nuevo registro para un documento mediante su adecuada catalogacion es decir proporcionando los datos adecuados para su identificación y futura consulta.
} \\
\hline
\bf Actores: &\multicolumn{3}{p{0.675\textwidth}|}{
Catalogador. 
} \\
\hline
\bf Precondiciones: &\multicolumn{3}{p{0.675\textwidth}|}
{
Haber ingresado al sistema y tener perfil de catalogador.
} \\
\hline
\multicolumn{4}{|c|}{\bf {Flujo Normal}}\\
\hline
\multicolumn{2}{|c}{\bf Actor} & \multicolumn{2}{|c|}{\bf Sistema } \\
\hline
\multicolumn{2}{|p{0.45\textwidth}}
{
\begin{itemize}
\item[1.] El caso de uso inicia cuando el catalogador le solicita al sistema que permita catalogar un documento.
\item[3. ]El catalogador llena los campos correspondientes de acuerdo a la información que muestra el libro haciendo uso del listado que presenta algunos campos y otros que si se llenan mediante la especificación escrita del dato.  
\item[4. ]El catalogador solicita que sean verificados los datos al sistema al intentar guardar la información suministrada.
\item[7. ]El catalogador acepta el mensaje de éxito y así termina este caso de uso.
\end{itemize}
} &
\multicolumn{2}{|p{0.45\textwidth}|}
{
\begin{itemize}
\item[ 2.] El sistema le responde enviando una interfaz con los campos correspondientes a la información que se requiere para poder crear un nuevo registro de documento , los campos son: Tipo de material, número de identificación, título principal, título secundario y/o traducido, editorial, fecha publicación, fecha  catalogación, Fecha creación, Idioma, derechos de autor y una breve descripción o resumen del material.
\item[5.] El sistema valida los datos suministrados a través de la interfaz.
\item[6.] El sistema responde mostrando un mensaje de exito que indica que la operación de registro del nuevo documento se llevo con éxito.
\end{itemize}
} \\
\hline
\multicolumn{4}{|c|}{\bf {Flujo Alternativo}}\\
\hline
\multicolumn{2}{|p{0.45\textwidth}}
{
\begin{itemize}
\item[4.2.] El Catalogador le indica al Sistema que decea suspender el resgistro del documento (para registrar una nueva area y/o autor).
\item[6.2.1.] El Catalogador le indica al Sistema que si desea salir.
\item[6.2.2.] El Catalogador le indica al Sistema que no desea salir y se continua el caso de uso en el flujo normal en 3.
\end{itemize}
} &
\multicolumn{2}{|p{0.45\textwidth}|}
{
\begin{itemize}
 \item[5.1.] El Sistema al validar los datos encuentra errores.
\item[6.1. ]El Sistema responde enviando un mensaje de error informando que alguno de los campos no se ha llenado de manera correcta y que debe ser corregido.
\item[5.2. ]El Sistema envía un mensaje de confirmación que indica si desea salir del proceso de registro.
\item[7.2.1.] El Sistema le informa que se ha cancelado el proceso y el caso de uso termina. 
\end{itemize}
} \\
\hline
\bf Poscondiciones: &\multicolumn{3}{p{0.675\textwidth}|}
{
El sistema añade un registro correspondiente a un nuevo documento.
} \\
\hline
\bf Excepciones: &\multicolumn{3}{p{0.675\textwidth}|}
{
Fallo de conexionen la base de datos. Falla en el sistema de suministro de energía.
} \\
\hline
\bf Aprobado por : & 
 & \bf Fecha & 
 \\
\hline
\end{longtable}
\end{center}
%\end{document}