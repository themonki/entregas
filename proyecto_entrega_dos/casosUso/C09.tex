%\documentclass[]{article}
%\usepackage[spanish]{babel}
%\usepackage[utf8]{inputenc}
%\usepackage{geometry}
%\usepackage{colortbl}
%\usepackage{longtable}
%\usepackage{graphicx}
%\geometry{tmargin=3cm,bmargin=3cm,lmargin=3cm,rmargin=2cm}
%
%\begin{document}
\begin{center}
\begin{longtable}{|p{0.225\textwidth}|p{0.225\textwidth}|p{0.225\textwidth}|p{0.225\textwidth}|}
\hline
{\bf {Empresa:}} &
\multicolumn{2}{p{0.45\textwidth}|} { Escuela de Ingeniería de Sistemas y Computación } &
{\includegraphics[width=80.5pt]{LOGO}} \\
\hline
\bf {Nombre del caso de uso:}&\multicolumn{3}{l|}{
Ingresar Autor.
} \\
\hline
\bf Codigo: & 
CU09 &\bf Fecha: & 
Abril 02 2011 \\
\hline
\bf Autor(es ): & 
Yerminson Gonzalez & 
Cristian Ríos & 
 \\
\hline
\bf Descripcion: &\multicolumn{3}{p{0.675\textwidth}|}
{
Permite la creación de nuevos autores en el Sistema.
} \\
\hline
\bf Actores: &\multicolumn{3}{p{0.675\textwidth}|}{
Administrador, Catalogador. 
} \\
\hline
\bf Precondiciones: &\multicolumn{3}{p{0.675\textwidth}|}
{
Tener el perfil de usuario Administrador o catalogador.
} \\
\hline
\multicolumn{4}{|c|}{\bf {Flujo Normal}}\\
\hline
\multicolumn{2}{|c}{\bf Actor} & \multicolumn{2}{|c|}{\bf Sistema } \\
\hline
\multicolumn{2}{|p{0.45\textwidth}}
{
\begin{itemize}
\item[1. ]El caso de uso inicia cuando el Usuario solicita crear un nuevo autor
\item[3.] El Usuario ingresa datos en los campos proporcionado por la interfaz del sistema para creación de nuevos autores.
\item[4. ]El Usuario indica al Sistema que ya a ingresado los datos y que desea crear el nuevo autor.
\item[7.] El Usuario acepta el mensaje de confirmación generado por el sistema y el caso de uso finaliza.
\end{itemize}
} &
\multicolumn{2}{|p{0.45\textwidth}|}
{
\begin{itemize}
\item[2.]El Sistema responde mostrando una interfaz con cinco campos: identificación del autor, nombre, apellido, correo electrónico y el acrónimo.
\item[5.]El Sistema valida que la identificación del autor que a ingresado el usuario para el nuevo autor no exista como identificación de otro autor.
\item[6. ] El Sistema crea un nuevo autor en el sistema y responde con un mensaje al usuario indicando el éxito de la operación. 
\end{itemize}
} \\
\hline
\multicolumn{4}{|c|}{\bf {Flujo Alternativo}}\\
\hline
\multicolumn{2}{|p{0.45\textwidth}}
{
\begin{itemize}
\item[7.1.] El Usuario acepta el mensaje de notificación del error generado por el sistema.
\end{itemize}
} &
\multicolumn{2}{|p{0.45\textwidth}|}
{
\begin{itemize}
\item[5.1.] El sistema valida  los datos y encuenta un error en los campos que han sido diligenciados por el Usuario tales como: nombre, apellido, correo electrónico y el acrónimo.
\item[6.1.] El sistema envía un mensaje notificando del error a la hora de llenar los campos.
\end{itemize}
} \\
\hline
\bf Poscondiciones: &\multicolumn{3}{p{0.675\textwidth}|}
{
El Sistema crea una nueva tupla de la base de datos correspondiente a la nueva área de ciencias de la computación.
} \\
\hline
\bf Excepciones: &\multicolumn{3}{p{0.675\textwidth}|}
{
El Sistema no puede acceder a la base de datos y no puede crear la nueva tupla o realizar las consultas para las validaciones de los datos.
} \\
\hline
\bf Aprobado por : & 
 & \bf Fecha & 
 \\
\hline
\end{longtable}
\end{center}
%\end{document}